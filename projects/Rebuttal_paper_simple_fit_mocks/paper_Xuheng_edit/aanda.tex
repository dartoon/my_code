%                                                                 aa.dem
% AA vers. 9.1, LaTeX class for Astronomy & Astrophysics
% demonstration file
%                                                       (c) EDP Sciences
%-----------------------------------------------------------------------
%
%\documentclass[referee]{aa} % for a referee https://www.overleaf.com/2144999964jkhxrcqqgfjrversion
%\documentclass[onecolumn]{aa} % for a paper on 1 column  
%\documentclass[longauth]{aa} % for the long lists of affiliations 
%\documentclass[letter]{aa} % for the letters 
%\documentclass[bibyear]{aa} % if the references are not structured 
%                              according to the author-year natbib style

%
\documentclass[traditabstract, twocolumns, longauth]{aa}  

%
\usepackage{txfonts}
\usepackage{graphicx}	% Including figure files
\usepackage{amsmath}	% Advanced maths commands
\usepackage{amssymb}	% Extra maths symbols
\usepackage{xcolor}     % Text color
\usepackage{soul}       % for strikeout, can be removed after comments are treated
\usepackage{array}
\usepackage[colorlinks=true,linkcolor=blue,citecolor=blue, urlcolor=blue]{hyperref}
\bibpunct{(}{)}{;}{a}{}{,}


\newcommand{\cdfcom}[1]{\xspace\textcolor{magenta}{\textbf{CDF: #1}}}
\newcommand{\warning}[1]{\textcolor{red}{\textbf{#1}}}
\def\ksmpc{${\, \mathrm{km}\, \mathrm{s}^{-1}\, \mathrm{Mpc}^{-1}}$\xspace}
\def\ks{${\, \mathrm{km}\, \mathrm{s}^{-1}}$\xspace}
\newcommand{\lcdm}{$\mathrm{\Lambda CDM}$\xspace}
%\newcommand{\slope}{$\mathrm{dH_0/d \sigma_{v}}$ \xspace}
\newcommand{\slope}{$\xi$}
\newcommand{\Bsixteen}{$\mathrm{B1608+656}$\xspace}
\newcommand{\RXJ}{$\mathrm{RX\ J1131-1231}$\xspace}
% CDF: Changed location of the "J" to be consistent with the other lenses
\newcommand{\HEzero}{$\mathrm{HE\ 0435-1223}$\xspace}
\newcommand{\Jtwelve}{$\mathrm{SDSS\ J1206+4332}$\xspace}
\newcommand{\WFItwenty}{$\mathrm{WFI\ 2033-4723}$\xspace}
\newcommand{\PGeleven}{$\mathrm{PG\ 1115+080}$\xspace}
\newcommand{\DESzerofour}{$\mathrm{DES\ J0408-5354}$\xspace}
\newcommand{\LENSTRO}{{\tt LENSTRONOMY}}

\newcommand{\Hc}{\ensuremath{H_0}\xspace}
\newcommand{\zd}{\ensuremath{z_\mathrm{d}}\xspace}
\newcommand{\zs}{\ensuremath{z_\mathrm{s}}\xspace}
\newcommand{\Dd}{\ensuremath{D_\mathrm{d}}\xspace}
\newcommand{\Dds}{\ensuremath{D_\mathrm{ds}}\xspace}
\newcommand{\Ds}{\ensuremath{D_\mathrm{s}}\xspace}
\newcommand{\Ddt}{\ensuremath{D_{\Delta t}}\xspace}
\newcommand{\kext}{\ensuremath{\kappa_\mathrm{ext}}\xspace}
\newcommand{\raper}{\ensuremath{\theta_\mathrm{aperture}}\xspace}
\newcommand{\reff}{\ensuremath{\theta_\mathrm{eff}}\xspace}
\newcommand{\thetaE}{\ensuremath{\theta_\mathrm{E}}\xspace}


\begin{document} 


\title{
TDCOSMO. I. An exploration of systematic uncertainties in the inference of \Hc from time-delay cosmography 
%TDCOSMO. I. An exploration of systematic uncertainties in time-delay cosmography and the inference of \Hc
%Asserting the impact of galaxy kinematics on \Hc measurements with time-delays in gravitationally lensed quasars
}
\author{
M.~Millon\inst{\ref{epfl}} \and
A.~Galan\inst{\ref{epfl}} \and
F.~Courbin\inst{\ref{epfl}} \and
T.~Treu\inst{\ref{ucla}} \and
S.~H.~Suyu\inst{\ref{mpa},\ref{tum},\ref{asiaa}} \and
X.~Ding\inst{\ref{ucla}} \and
S.~Birrer\inst{\ref{stanford}} \and
G.~C.-F.~Chen \inst{\ref{ucdavis}} \and
A.~J.~Shajib \inst{\ref{ucla}} \and
K.~C.~Wong \inst{\ref{kavliipmu}} \and
A.~Agnello\inst{\ref{dark}} \and
M.~W.~Auger\inst{\ref{ucambridge},\ref{kavlicambridge}} \and
E.~J.~Buckley-Geer\inst{\ref{Fermilab}} \and
J.~H.~H.~Chan\inst{\ref{epfl}}\and
T.~Collett\inst{\ref{uportsmouth}} \and
C.~D.~Fassnacht \inst{\ref{ucdavis}} \and
S.~Hilbert \inst{\ref{mpa}} \and
L.~V.~E.~Koopmans\inst{\ref{unigroningen}}\and 
V.~Motta \inst{\ref{universidadvalparaiso}} \and
S.~Mukherjee \inst{\ref{star}}\and
C.~E.~Rusu \inst{\ref{naoj}} \and
D.~Sluse\inst{\ref{star}} \and
A.~Sonnenfeld \inst{\ref{leiden}} \and
C.~Spiniello \inst{\ref{OAC},\ref{ESO}} \and
L.~Van de Vyvere\inst{\ref{star}} 
}

\institute{
Institute of Physics, Laboratory of Astrophysics, Ecole Polytechnique 
F\'ed\'erale de Lausanne (EPFL), Observatoire de Sauverny, 1290 Versoix, 
Switzerland \label{epfl} \goodbreak 
\and
Department of Physics and Astronomy, University of California, Los Angeles CA 90095, USA \label{ucla} \goodbreak
\and
Max-Planck-Institut f{\"u}r Astrophysik, Karl-Schwarzschild-Str.~1, 85748 Garching, Germany \label{mpa} \goodbreak
\and
Physik-Department, Technische Universit\"at M\"unchen, James-Franck-Stra\ss{}e~1, 85748 Garching, Germany \label{tum} \goodbreak
\and
Academia Sinica Institute of Astronomy and Astrophysics (ASIAA), 11F of ASMAB, No.1, Section 4, Roosevelt Road, Taipei 10617, Taiwan \label{asiaa} \goodbreak
\and 
Kavli Institute for Particle Astrophysics and Cosmology and Department of Physics, Stanford University, Stanford, CA 94305, USA \label{stanford}\goodbreak
\and
 Department of Physics, University of California, Davis, CA 95616, USA \label{ucdavis}\goodbreak
 \and 
 Kavli IPMU (WPI), UTIAS, The University of Tokyo, Kashiwa, Chiba 277-8583, Japan\label{kavliipmu} \goodbreak
 \and
DARK, Niels Bohr Institute, Lyngbyvej 2, 2100 Copenhagen, Denmark \label{dark} \goodbreak
\and 
 Kapteyn Astronomical Institute, University of Groningen, P.O.Box 800, 9700AV Groningen, Netherlands \label{unigroningen} \goodbreak
 \and 
 Instituto de F\'{\i}sica y Astronom\'{\i}a, Facultad de Ciencias, Universidad de Valpara\'{\i}so, Avda. Gran Breta\~na 1111, Valpara\'{\i}so, Chile. \label{universidadvalparaiso} \goodbreak
 \and
 STAR Institute, Quartier Agora - All\'ee du six Ao\^ut, 19c B-4000 Li\`ege, Belgium \label{star} \goodbreak
 \and
 National Astronomical Observatory of Japan, 2-21-1 Osawa, Mitaka, Tokyo 181-8588, Japan \label{naoj}\goodbreak
\and
Institute of Astronomy, University of Cambridge, Madingley Road, Cambridge CB3 0HA, UK \label{ucambridge}\goodbreak
\and 
Kavli Institute for Cosmology, University of Cambridge, Madingley Road, Cambridge CB3 0HA, UK \label{kavlicambridge}\goodbreak
\and
Leiden Observatory, Leiden University, Niels Bohrweg 2, 2333 CA Leiden, the Netherlands
\label{leiden}\goodbreak
\and
University of Portsmouth, Institute of Cosmology and Gravitation, Portsmouth PO1 3FX, United Kingdom \label{uportsmouth} \goodbreak
\and
INAF - Osservatorio Astronomico di Capodimonte, Salita Moiariello, 16, 80131, Napoli, Italy
\label{OAC} \goodbreak
\and
European Southern Observatory, Karl-Schwarschild-Str. 2, 85748, Garching, Germany
\label{ESO} \goodbreak
\and 
Fermi National Accelerator Laboratory, P. O. Box 500, Batavia, IL 60510, USA \label{Fermilab}
} 


%\datesmooth{\today}
\abstract{
}
\keywords{methods: data analysis – gravitational lensing: strong}

\titlerunning{Uncertainties in time-delay cosmography}
\maketitle
%\begin{comment}
%Low-redshift and pre-recombinaiton measurements of the Hubble constant \Hc are at 3-5$\sigma$ tension, which requires new physics beyond $\Lambda$CDM or significant systematics.
%Time-delay cosmography is a single-step method, independent of any distance-ladder calibration, and has resulted in 2.4\% precision on \Hc. As the need to control systematics becomes more pressing, we analyse three sources of systematic uncertainties: stellar kinematics, line-of-sight effects, and lens model degeneracies.
%Simplified toy models cannot provide quantitative insights in this regime, so we closely mimic the H0LiCOW/SHARP/STRIDES (TDCOSMO) procedure.
%First, given current uncertainties, stellar kinematics is not a dominant source of bias. Second, there is no evidence of any bias arising from incorrect estimation of the line-of-sight effects. Third, we show that arbitrary elliptical composite (stars+halo), power-law, and cored power-law mass profiles have the flexibility to yield a range of \Hc estimates.
%Whenever two lens models disagree, the TDCOSMO procedure can discriminate them based on goodness-of-fit, and/or account for the discrepancy in the error bars once models are marginalised over.
%This conclusion is consistent with a reanalysis of the six TDCOSMO lenses: the composite and power-law models yield \Hc=$74.0^{+1.6}_{-1.7}$kms$^{-1}$Mpc$^{-1}$ and  $73.8^{+1.7}_{-1.9}$ kms$^{-1}$Mpc$^{-1}$ respectively.
%In conclusion, we find no evidence of bias larger than the current statistical uncertainties. %Future papers by the TDCOSMO collaboration will continue to investigate potential sources of bias as precision increases.
%\end{comment}

%
%-------------------------------------------------------------------
\section{Introduction}
\label{sec:introduction}


\section{Background \label{sec:msd}}

%%%%%%%%%%%%%%
\section{Inference procedure and limitations of toy models}
\label{sec:toysareforbabies}
%%%%%%%%%%%%%%

\section{Influence of kinematics data on \Hc measurement \label{sec:kinematics_impact}}
\label{sec:kinem_sensitivity}

%%%%%%%%%%%%%%%%%%%%%%
\section{Is there any evidence for correlation between \Hc and physically independent observables?}
%%%%%%%%%%%%%%%%%%%%%%
\label{sec:corr}

%%%%%%%%%%%%%%
\section{Impact of the choice of families of mass model}
\label{sec:6}
%%%%%%%%%%%%%%

%%%%%%%%%%%%%%%%%%
\subsection{\Hc inference per model family}
\label{ssec:powervscomposite}
%%%%%%%%%%%%%%%%%%

%%%%%%%%%%%%
\subsection{Simulations}
\label{ssec:sim}
%%%%%%%%%%%%


\subsection{Results}
\label{ssec:results}

\subsection{Discussion}
\label{ssec:disc}

({\color{red}{{\bf Xuheng:} Currently, let's consider to put these sentences somewhere in section~\ref{ssec:disc}:}})

As mentioned in Section~\ref{sec:toysareforbabies}, neglecting the other information but only considering the point-source information and lensing galaxy position, is not sufficient to constrain all the lens model parameters. As an additional test to address this point, we model the simulated data using only the four lensed image positions and their time delays. A reduced $\chi^2 <$ 1 can be obtained for all the mocks using a power law model, except for mock \#6 for which the best reduced $\chi^2 \sim 1.9$. Even when the true mass distribution is a power law (e.g. mock \#1 and \#2), the maximum likelihood models are associated with power-law indices substantially different from the input one, yielding a bias on \Hc\, that can reach 90\% (see Appendix~\ref{app:simple_model} for details). This is well understood as the multipole components of the lens potential can compensate for large changes in the monopole structure which are only poorly constrained by the few image positions. This test highlights the necessity of using the full information provided by the high resolution images to better constrain the lens potential. In particular, the multiple images of the lensed host galaxy are critical to pin down the uncertainty on the average mass density at the image positions, which is critical for \Hc inference (add a ref on CSJK2001 on "the importance of Einstein rings  \url{http://adsabs.harvard.edu/abs/2001ApJ...547...50K} which basically make this point).

%%%%%%%%%%%%%%%%%
\section{Conclusion}
\label{conclusion}
%%%%%%%%%%%%%%%%%

\begin{acknowledgements}
\end{acknowledgements}

\bibliographystyle{aa}
%\bibliography{biblio}

\begin{appendix}
\onecolumn
\section{Properties of simulated lenses}
...

\section{Model only based on lensed quasar position}\label{app:simple_model}

% To perform a consistent comparison to the modeling process as in Section~\ref{ssec:results},
We model the simulated dataset as generated in Section~\ref{ssec:sim}, using only the lensed quasar positions, the lensing galaxy position and the relative time delays between the lensed images. We assume an uncertainty $\sigma = 0\farcs{004}$ on the point-source positions, $\sigma = 0\farcs{01}$ on the lensing galaxy centroid, and the same uncertainty on the time-delay as in Sect.~\ref{ssec:results}. Similarly to extended source modeling performed (Sect.~\ref{ssec:results}), we employ the lens modeling package \LENSTRO. We adopt both the singular isothermal ellipsoid (SIE) model (i.e., fix slope value $\gamma=2.0$) and the Power-law model in this test. For the SIE, an independent modeling has been carried out with \texttt{lensmodel} \citep{Keeton2001, Keeton2011}, as well as for modeling mock ID \#1 with a SPEMD. We obtained similar inference with both packages and therefore only report hereafter on results obtained with \LENSTRO.  

We use the true parameters as the input values to start performing the modeling. A careful choice of the likelihood and sampling options has to be carried out to ensure that image position constraints arise from the same source. While the details of the fitting options were not critical for SIE models, they were for the SPEMD modeling. In practice, we sample the source plane and evaluate the positional likelihood in the image plane, but adding a source plane likelihood term to ensure that each image arise from the same source within $\sigma = 0\farcs{001}$  [need to be clarified based on what was finally implemented - cf source likelihood term if and ]. A notebook implementing our fitting strategy is available at {\url{}}. \textcolor{red}{[To do: Xuheng will create a notebook for this]}

In Figure~\ref{fig:corner_pos_only}, we show the corner plots of the inference based on the mock system ID \#1. As shown, when using a SIE model where the mass slope value fixed to the truth, the fitting could obtain an unbiased \Hc\ with uncertainty level at $\sim10\%$ level. However, when the slope is a free parameter, the inference broadens significantly as the data are not sufficient to constrain that parameter. In particular, the uncertainty level of \Hc\ increase by a factor of $\sim3$ and the maximum likelihood deviates from the truth.
\textcolor{red}{[Xuheng: A corner plot of Figure~\ref{fig:corner_pos_only} might be enough to clarify our points. We can also show the H0 posterior histogram for the overall six systems if the other people feel needed.]}

\begin{figure*}[t!]
    \centering
    \includegraphics[width=0.49\textwidth]{corner_plot_SIE_ID1.pdf}
     \includegraphics[width=0.49\textwidth]{corner_plot_SPEMD_ID1.pdf}
    \caption{The corner plot of the inference of modeling mock system ID \#1 using only the lensed quasar position and time delay. The SIE model and Power-law model are adopted on the left and right, separately. The blue lines indicate the true values in the simulation. }
    \label{fig:corner_pos_only}
\end{figure*}


\end{appendix}
\end{document}