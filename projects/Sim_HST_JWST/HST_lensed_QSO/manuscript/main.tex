\documentclass[useAMS,usenatbib,usegraphicx]{mn2e}
\usepackage{graphicx}% Include figure files
\usepackage{subfigure}% Include subfigure files
\usepackage{xcolor}% Change colors of text using \textcolor{red}{...}
\usepackage{mathtext,bm,bbm,amsmath,amsfonts,amssymb,indentfirst,syntonly,graphicx,epsf}
\usepackage{mathtools}
\usepackage[english]{babel}
\usepackage{calc}
\usepackage{tikz}
\usepackage[T1]{fontenc}
\usepackage{ae,aecompl}

\newcommand{\hc}{$H_0$}


\title[Lens modelling improvements in lensed transients]{Lens modelling improvements in strongly lensed explosive transient systems}
\author[Ding et al.]
{Xuheng Ding$^{1}$,
and Kai Liao$^{2}$\thanks{e-mail: liaokai@whut.edu.cn}
and TBD\\
$^{1}$Department of Physics and Astronomy, University of California, Los Angeles, CA, 90095-1547, USA\\
$^{2}$School of Science, Wuhan University of Technology, Wuhan 430070, China}
\begin{document}

\date{Accepted xxxx; Received xxxx; in original form xxxx}

\pagerange{\pageref{firstpage}--\pageref{lastpage}} \pubyear{2018}

\maketitle

\label{firstpage}

\begin{abstract}
  See the review by Oguri M. \citep{Oguri2019}. Strongly lensed explosive transients like supernovae, gamma ray bursts, fast radio bursts and gravitational waves are very promising to determine the Hubble constant ($H_0$) in the next stage. Compared to the traditional targets: lensed quasars, they can give much more precise time delay measurements due to either the transient nature or well-known light curves. In this work, we quantitatively study how lens modelling improves without the contamination from bright point source, i.e., the AGN, since the complete host arcs can be observed before or after the transient's appearance. We simulate 8 lensing systems with and without adding point sources and compare the inferred Fermat potential difference.
  Generally, the lens modelling can be improved by 3?? times. Moreover, we incorporate
  the information on the magnifications and find lens modelling can be further improved by ??.
\end{abstract}

\begin{keywords}
gravitational lensing: strong \--- Hubble constant
\end{keywords}


\section{Introduction}

The Hubble constant ($H_0$) is currently attracting the eyes of the cosmology community. The direct measurement based on distance ladders is different from that from cosmic microwave background (CMB) by $4.4\sigma$. Note that $H_0$ in the latter is treated as a free parameter. This discrepancy manifests that
either unknown systematic errors existing in the data or new physics.

Strong gravitational lensing provides an independent and one-step way to determine $H_0$.

Lensed transient systems resulting in precise time-delay cosmology
include supernovae in all types,
GW/SGRB+kilonava/afterglow association~\citep{Liao2017}, LGRB+SN Ic/afterglow association and repeated FRB.


\section{Time delay cosmology}
Time delay between multiple images was proposed to measure $H_0$.
Strong lensing gives: 
\begin{equation}
D_{\Delta t}=\frac{c\Delta t}{\Delta\phi(\boldsymbol{\xi}_{lens})}\frac{1}{1-\kappa_\mathrm{ext}},
\end{equation}
where the time delay distance:
\begin{equation}
D_{\Delta t}=(1+z_l)\frac{D_lD_s}{D_{ls}}
\end{equation}
is primarily proportional to $1/H_0$ and weakly depends on other cosmological parameters.
To determine $D_{\Delta t}$, three uncertainty sources need to be considered. First of all,
the time delay $\Delta t$ between any two images of AGN is measured with long-term light curves.
Secondly, the Fermat potential difference can be determined by the high-resolution imaging of the host arcs, the dispersion velocity of the lens galaxy and the AGN image positions. At last,
the perturbers along the line of sight (LOS) can contribute to the lens potential as well, causing additional (de)focusing of the light rays and affecting the observed time delay. They
can be approximated by an external convergence $\kappa_{\rm{ext}}$, resulting in a scaled inferred $D_{\Delta t}$ by $1-\kappa_{\rm{ext}}$.

\section{Data simulation and modelling test}
%structure:
%1. describe the goal -- simulate two contract sample (with and without AGN) to test how H0 measurement improves when the brightness AGN effect disappear.
It is well-known that one of the major difficulties in the lensed AGN modelling is that the bright point source dominates the central regions in the host, which significantly reduce the number of pixels with available lensed arc information making the lensing modelling more challengeable.  
To investigate at which level the bright AGN affects the inference of the lens model and the inference of \hc, we make a group of two samples and perform a control experiment. The first group is a sample of 50 mock HST observed lensed AGN systems, the data quality is based on the real observation as used by the TDCOSMO collaboration. As a contrast sample, the second group are generated based on the same parameters and only that the AGN image is not added. We then adopt the lens modelling software lenstronomy to model these two samples to make direct comparison of the inference using the two samples. In addition, the kinematics information of the deflector would be helpful to get extra constrains on the lens model [ref]. However, the aim of this work is to test the lens modelling by comparing the fittings when with/without AGN, and using lens velocity dispersion information to give extra constrains on the lens model is beyond the scope of this paper. 

\subsection{Mock data simulations}
The simulations of the 50 lensed AGN systems are based on the pipeline introduced by [Ding et al.].
This approach has also been used in TDLMC and TDCOSMO-1 [ref]. The mock data are assumed to be observed by HST through WFC/F160W. %We adopt the same simulating strategy as used TDLMC to generate the mock sample, which are summarized as below.

The simulating image is composed of the light from the deflector galaxy in the image plane and active galaxies in the source plane. The light profile of the galaxy is assumed as S\'ersic profile is parameterized by:
\begin{eqnarray}
   \label{eq:sersic}
   &I(R) = A \exp\left[-k\left(\left(\frac{R}{R_{\mathrm{eff}}}\right)^{1/n}-1\right)\right] ,\\
   &R(x,y,q) = \sqrt{qx^2+y^2/q}.
\end{eqnarray}
%
where $A$ is the amplitude and S\'ersic index $n$ controls the shape of the radial
surface brightness profile; a larger $n$ corresponds to a steeper
inner profile and a highly extended outer wing. 
 $k$ is a constant which
depends on $n$ so as to ensure that the isophote at $R=R_{\mathrm{eff}}$
encloses half of the total light \citep{C+B99} and
$q$ denotes the axis ratio. The light of the active nuclei is assumed as a non-resolved point source.

The mass of the deflector is assumed as a elliptical power-law model, whose surface mass density is given by:
%
\begin{equation}
 \label{massmodel}
 \Sigma(x,y)=\Sigma_{cr}\frac{3-\gamma^{\prime}}{2}\left(\frac{\sqrt{q_m x^{2}+y^{2}/q_m}}{R_{\rm E}}\right)^{1-\gamma^{\prime}},
\end{equation}
%
$q_m$ described the projected axis ratio.
The so-called Einstein radius $R_{\rm{E}}$ is chosen such
that, when $q_m=1$ (i.e. spherical limit), it encloses a mean surface
density equal to $\Sigma_{cr}.$
%This is also the radius of a ring traced by the host of the AGN when this is exactly aligned with the lens galaxy.
The exponent $\gamma'$ is the slope of the power-law profile,
for massive elliptical galaxies $\gamma' \approx2$  \citep{T+K02a,T+K04,Koo++09}.
We refer the reader to the reviews by \citet{Sch06, Bar10, Tre10} for more details.
An external shear component is also considered.


The assumed lensing parameters are randomly generated using the based on the same strategy as in Table 2 in TDLMC-2. The position of the source AGN is manually input so that all the lensed image are quads. The noise includes the Gaussian background noise and the Poisson noise which is based on the realistic HST condition with exposure time as $\sim$9,600 s. The PSF is generated by tinytim. The pixel scale is drizzled from 0\farcs{13} to 0\farcs{08} into the frame size with $99\times99$ pixels. The 50 lensed non-AGN systems are also simulated using the identical parameters as the 50 lensed AGN sample, expect that the point-source flux is muted. 

The time delays are calculated based on the lens parameters based on a standard flat $\Lambda$CDM model, with Om = 0.27 and Om$_\Lambda$ = 0.73. The true H0 value is assumed as 74. Note that since the two samples (i.e., with AGN and without AGN) share the same parameters, their time delays are the identical. For the uncertainties level, we assume a non-biased error with rms level as the larger on between 1\% level and 0.25 days.

In figure~, we illustrate the simulated lensed image that based on the same lensing parameters.

[fig]

\subsection{Model }
%3. Modelling. Based on Lenstronomy, start with PSO and then MCMC. For lensed AGN, we adopt the PSF iteration. For lensed galaxy, the position of the invisible point source are taken from the lensed AGNs. We use a sample of 40 systems to estimate the scatter of the H0 inference. The kinematic information is not considered, since the goal of this paper is to make a direct comparisons.
We adopt the strong lensing modeling tool lenstronomy to perform the modelling.

For the lensed AGN systems, the typical modelling approaching is adopted. We manually boost the Poisson noise level to mitigate the drizzling effects. A mask region is defined to bracket the pixels with sufficient SNR with would be used to calculate the likelihood. To account for the fact that the information of the PSF is unknown, we assumed a 0.1 variance uncertainty on the input PSF image when modelling the point source. The true values used in the simulation are adopted as the initial values to allow a fast convergence of the particles. The lensing image are combined with the time delay data to achieve a joint-inference for the time delay distance. We adopt the particle swarm optimization (PSO) to derive the minimization best-fit result for each system. The resulting H0 is derived with Om fixed to the truth value 0.27.
Finally, we consider the 50 lensed systems are considered as 50 times realizations to investigate the uncertainty and the accuracy.

The same approach is adopted to fit the lensed non-AGN system. Note that when calculate the time delay, the positions of the point source in the image plane and source plane are needed (i.e., alpha and beta in eq). However, for the non-AGN case the point source position in the image plane does not appear. We use the twin lensed AGN system and adopt the fitted point source position as the astrometry information. This strategy is to mimic the situation when a SN is explosion  with point source position captured.  

Figure illustrate the modelling result for a twin systems.
[ref]


\section{Results}
[] The lensed AGN case is non-biased, but the scatter is larger. This scatter represent the precision level when modeling the lensed without taken the kinematic information. 
[]When AGN disappear, the non-biased H0 can be still inferred and the scatter is improved by a factor of $\sim$3. 

\section{What does this result mean?}
[] Lens modeling alone would improved by a factor of 3 in the future lensed SN case.
[] What about the improvement of the H0? a combination discussion of the time delay (precision and micro-lensing), the lens modelling. The kinematic measurement of the deflector should also be improved. What about the kappa external? probably kext would dominate the uncertainty. 

\section{Conclusions and discussions}



\section*{Acknowledgments}
This work was supported by the National Natural Science Foundation of China (NSFC) No. 11973034.





\begin{thebibliography}{}
\bibitem[\protect\citeauthoryear{Oguri}{2019}]{Oguri2019} Oguri M. 2019, Reports on Progress in Physics, 82, 12
\bibitem[\protect\citeauthoryear{Liao et al.}{2017}]{Liao2017} Liao K., Fan X.-L., Ding X., Biesiada M., Zhu Z.-H. 2017, Nature Communications, 8, 1148
\end{thebibliography}

\label{lastpage}

\end{document}
