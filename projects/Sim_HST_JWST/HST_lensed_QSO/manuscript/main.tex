\documentclass[useAMS,usenatbib,usegraphicx]{mn2e}
\usepackage{graphicx}% Include figure files
\usepackage{subfigure}% Include subfigure files
\usepackage{xcolor}% Change colors of text using \textcolor{red}{...}
\usepackage{mathtext,bm,bbm,amsmath,amsfonts,amssymb,indentfirst,syntonly,graphicx,epsf}
\usepackage{mathtools}
\usepackage[english]{babel}
\usepackage{calc}
\usepackage{tikz}
\usepackage[T1]{fontenc}
\usepackage{ae,aecompl}

\newcommand{\hc}{$H_0$}


\title[Lens modelling improvements in lensed transients]{Lens modelling improvements in strongly lensed explosive transient systems}
\author[Ding et al.]
{Xuheng Ding$^{1}$,
and Kai Liao$^{2}$\thanks{e-mail: liaokai@whut.edu.cn}
and TBD\\
$^{1}$Department of Physics and Astronomy, University of California, Los Angeles, CA, 90095-1547, USA\\
$^{2}$School of Science, Wuhan University of Technology, Wuhan 430070, China}
\begin{document}

\date{Accepted xxxx; Received xxxx; in original form xxxx}

\pagerange{\pageref{firstpage}--\pageref{lastpage}} \pubyear{2018}

\maketitle

\label{firstpage}

\begin{abstract}
  Strongly lensed explosive transients like supernovae, gamma ray bursts, fast radio bursts and gravitational waves are very promising to determine the Hubble constant ($H_0$) in the next stage. Compared to the traditional targets: lensed quasars, they can give much more precise time delay measurements due to either the transient nature or well-known light curves. In this work, we quantitatively study how lens modelling improves without the contamination from bright point source, i.e., the AGN, since the complete host arcs can be observed before or after the transient's appearance. We simulate 8 lensing systems with and without adding point sources and compare the inferred Fermat potential difference.
  Generally, the lens modelling can be improved by 3?? times. Moreover, we incorporate
  the information on the magnifications and find lens modelling can be further improved by ??.
\end{abstract}

\begin{keywords}
gravitational lensing: strong \--- Hubble constant
\end{keywords}


\section{Introduction}

The Hubble constant ($H_0$) is currently attracting the eyes of the cosmology community. The direct measurement based on distance ladders is different from that from cosmic microwave background (CMB) by $4.4\sigma$. Note that $H_0$ in the latter is treated as a free parameter. This discrepancy manifests that
either unknown systematic errors existing in the data or new physics.

Strong gravitational lensing provides an independent and one-step way to determine $H_0$.

Lensed transient systems resulting in precise time-delay cosmology
include supernovae in all types,
GW/SGRB+kilonava/afterglow association, LGRB+SN Ic/afterglow association and repeated FRB.


\section{Time delay cosmology}
Time delay between multiple images was proposed to measure $H_0$.
Strong lensing gives: 
\begin{equation}
D_{\Delta t}=\frac{c\Delta t}{\Delta\phi(\boldsymbol{\xi}_{lens})}\frac{1}{1-\kappa_\mathrm{ext}},
\end{equation}
where the time delay distance:
\begin{equation}
D_{\Delta t}=(1+z_l)\frac{D_lD_s}{D_{ls}}
\end{equation}
is primarily proportional to $1/H_0$ and weakly depends on other cosmological parameters.
To determine $D_{\Delta t}$, three uncertainty sources need to be considered. First of all,
the time delay $\Delta t$ between any two images of AGN is measured with long-term light curves.
Secondly, the Fermat potential difference can be determined by the high-resolution imaging of the host arcs, the dispersion velocity of the lens galaxy and the AGN image positions. At last,
the perturbers along the line of sight (LOS) can contribute to the lens potential as well, causing additional (de)focusing of the light rays and affecting the observed time delay. They
can be approximated by an external convergence $\kappa_{\rm{ext}}$, resulting in a scaled inferred $D_{\Delta t}$ by $1-\kappa_{\rm{ext}}$.

\section{Data simulation and modelling test}
%structure:
%1. describe the goal -- simulate two contract sample (with and without AGN) to test how H0 measurement improves when the brightness AGN effect disappear.
It is well-known that one of the major difficulties in the lensed AGN modelling is that the bright point source dominates the central regions in the host, which significantly reduce the number of pixels with available lensed arc information making the lensing modelling more challengeable.  
To investigate at which level the bright AGN affects the inference of the lens model and the inference of \hc, we make a group of two samples and perform a control experiment. The first group is a sample of 50 mock HST observed lensed AGN systems, the data quality is based on the real observation as used by the TDCOSMO collaboration. As a contrast sample, the second group are generated based on the same parameters and only that the AGN image is not added. We then adopt the lens modelling software lenstronomy to model these two samples to make direct comparison of the inference using the two samples. In addition, the kinematics information of the deflector would be helpful to get extra constrains on the lens model [ref]. However, the aim of this work is to test the lens modelling by comparing the fittings when with/without AGN, and using lens velocity dispersion information to give extra constrains on the lens model is beyond the scope of this paper. 

\subsection{Mock data simulations}
The simulations of the 50 lensed AGN systems are based on the pipeline introduced by [Ding et al.].  The mock data are based on the observations using HST WFC/F160W.
This pipeline has also been used in TDLMC and TDCOSMO-1 [ref]. We adopt the same simulating strategy as used TDLMC to generate the mock sample, which are summarized as below.

The simulating ingredients includes the light from the deflector and active galaxies in the source plane. 

The light profile of the galaxy is assumed as S\'ersic profile is parameterized by:
\begin{eqnarray}
   \label{eq:sersic}
   &I(R) = A \exp\left[-k\left(\left(\frac{R}{R_{\mathrm{eff}}}\right)^{1/n}-1\right)\right] ,\\
   &R(x,y,q) = \sqrt{qx^2+y^2/q}.
\end{eqnarray}
%
where $A$ is the amplitude and S\'ersic index $n$ controls the shape of the radial
surface brightness profile; a larger $n$ corresponds to a steeper
inner profile and a highly extended outer wing. 
 $k$ is a constant which
depends on $n$ so as to ensure that the isophote at $R=R_{\mathrm{eff}}$
encloses half of the total light \citep{C+B99} and
$q$ denotes the axis ratio.

The mass of the deflector is assumed as a elliptical power-law model, whose surface mass density is given by:
%
\begin{equation}
 \label{massmodel}
 \Sigma(x,y)=\Sigma_{cr}\frac{3-\gamma^{\prime}}{2}\left(\frac{\sqrt{q_m x^{2}+y^{2}/q_m}}{R_{\rm E}}\right)^{1-\gamma^{\prime}},
\end{equation}
%
$q_m$ described the projected axis ratio.
The so-called Einstein radius $R_{\rm{E}}$ is chosen such
that, when $q_m=1$ (i.e. spherical limit), it encloses a mean surface
density equal to $\Sigma_{cr}.$
%This is also the radius of a ring traced by the host of the AGN when this is exactly aligned with the lens galaxy.
The exponent $\gamma'$ is the slope of the power-law profile,
for massive elliptical galaxies $\gamma' \approx2$  \citep{T+K02a,T+K04,Koo++09}.
We refer the reader to the reviews by \citet{Sch06, Bar10, Tre10} for more details.
An external shear component is also considered.


The assumed lensing parameters are randomly generated using the Table 2 in TDLMC-2. The position of the source AGN is manually input so that all the lensed image are quads. The noise includes the Gaussian background noise and the Poisson noise which is based on the realistic HST condition with exposure time as 9,584 s. The PSF is generated by tinytim. The pixel scale is drizzled from 0\farcs{13} to 0\farcs{08} into the frame size with $99\times99$ pixels. The 50 lensed non-AGN systems are also simulated using the identical parameters as the 50 lensed AGN sample, expect that the point-source flux is muted. 

The time delays are calculated based on the lens parameters based on a standard flat $\Lambda$CDM model, with Om = 0.27 and Om$_\Lambda$ = 0.73. The true H0 value is assumed as 74. Note that since the two samples share the same parameters, their time delays are the same. For the uncertainties level, we assume a non-biased error with rms level as the larger on between 1\% level and 0.25 days.

In figure~, we illustrate the simulated lensed image that based on the same lensing parameters.

[fig]

\subsection{Model test}
%3. Modelling. Based on Lenstronomy, start with PSO and then MCMC. For lensed AGN, we adopt the PSF iteration. For lensed galaxy, the position of the invisible point source are taken from the lensed AGNs. We use a sample of 40 systems to estimate the scatter of the H0 inference. The kinematic information is not considered, since the goal of this paper is to make a direct comparisons.
We use the strong lensing modeling tool lenstronomy to perform the modelling. For the lensed AGN systems, the typical modelling approaching is adopted. We manually boost the Poisson noise level to mitigate the drizzling effects. 
[] The PSF uncertainty 0.1. 
[] The lens images are combining with the time delay data to joint infer the time delay distance, hence the h0.
[] Mask was added.
[] We input the true parameters as the initial values to perform the Minimization. Finally, the H0 inferred from the overall 50 systems are considered as 50 times realizations to investigate the uncertainty and the accuracy.

We adopt the same approach to model the lensed non-AGN systems. Since the point source are not appeared in the image, we applied the astrometry information inferred from lensed AGN systems to mimic the situation when a SN is blooming with point source available.  


\section{Comparison results}
[] The lensed AGN case is non-biased, but the scatter is larger. This scatter represent the precision level when modeling the lensed without taken the kinematic information. 
[]When AGN disappear, the non-biased H0 can be still inferred and the scatter is improved by a factor of $\sim$3. 

\section{What does this result mean?}
[] Lens modeling alone would improved by a factor of 3 in the future lensed SN case.
[] What about the improvement of the H0? a combination discussion of the time delay (precision and micro-lensing), the lens modelling. The kinematic measurement of the deflector should also be improved. What about the kappa external? probably kext would dominate the uncertainty. 

\section{Conclusions and discussions}



\section*{Acknowledgments}
This work was supported by the National Natural Science Foundation of China (NSFC) No. 11973034.





\begin{thebibliography}{}
\bibitem[\protect\citeauthoryear{Abbott et al.}{2016}]{Abbott2016} Abbott B. P., Abbott R., Abbott T. D., et al. 2016, Phys. Rev. Lett., 116, 061102
\bibitem[\protect\citeauthoryear{Abbott et al.}{2019}]{Abbott2019b} Abbott B. P., Abbott T. D., Abraham S., et al. 2019, ApJL, 882, L24
\bibitem[\protect\citeauthoryear{Alcock et al.}{2000}]{Alcock2000} Alcock C., et al. [MACHO Collaboration], 2000, ApJ, 542, 281
\bibitem[\protect\citeauthoryear{Ali-Haimoud et al.}{2017}]{Ali-Haimoud2017}  Ali-Ha\"{\i}moud Y., Kamionkowski M. 2017, Phys. Rev. D, 95, 043534
\bibitem[\protect\citeauthoryear{Benton et al.}{2007}]{Benton2007} Benton Metcalf R., Silk J. 2007, Phys. Rev. Lett., 98, 071302
\bibitem[\protect\citeauthoryear{Bird et al.}{2016}]{Bird2016} Bird S., Cholis I., Mu\~noz J. B., et al. 2016, Phys. Rev. Lett., 116, 201301
\bibitem[\protect\citeauthoryear{Brandt et al.}{2016}]{Brandt2016} Brandt T. D. 2016, ApJ, 824, L31
\bibitem[\protect\citeauthoryear{Carr et al.}{1974}]{Carr1974} Carr B. J., Hawking S. W. 1974, MNRAS, 168, 399
\bibitem[\protect\citeauthoryear{Carr et al.}{1975}]{Carr1975} Carr B. J. 1975, ApJ, 201, 1
\bibitem[\protect\citeauthoryear{Calchi Novati et al.}{2013}]{Calchi Novati2013} Calchi Novati S., Mirzoyan S., Jetzer P., Scarpetta G. 2013, MNRAS, 435, 1582
\bibitem[\protect\citeauthoryear{Cao et al.}{2014}]{Cao2014} Cao Z., Li L.-F., Wang Y. 2014, Phys. Rev. D, 90, 062003
\bibitem[\protect\citeauthoryear{Collett et al.}{2017}]{Collett2017} Collett T. E., Bacon D. 2017, Phys. Rev. Lett., 118, 091101
\bibitem[\protect\citeauthoryear{Christian et al.}{2018}]{Christian2018} Christian P., Vitale S., Loeb A. 2018, Phys. Rev. D, 98, 103022
\bibitem[\protect\citeauthoryear{Dai et al.}{2018}]{Dai2018} Dai L., Li S.-S., Zackay B., Mao S., Lu Y. 2018, Phys. Rev. D, 98, 104029
\bibitem[\protect\citeauthoryear{Ding et al.}{2015}]{Ding2015} Ding X., Biesiada M., Zhu Z.-H. 2015, JCAP, 12, 006
\bibitem[\protect\citeauthoryear{Ding et al.}{2020}]{Ding2020} Ding X., Liao K., Biesiada M., Zhu Z.-H. 2020, ApJ, 891, 76
\bibitem[\protect\citeauthoryear{Dominik et al.}{2013}]{Dominik2013} Dominik M., Belczynski K., Fryer C., et al. 2013, ApJ, 779, 72
\bibitem[\protect\citeauthoryear{Fan et al.}{2017}]{Fan2017} Fan X.-L., Liao K., Biesiada M.,  et al. 2017, Phys. Rev. Lett., 118, 091102
\bibitem[\protect\citeauthoryear{Finn et al.}{1996}]{Finn1996} Finn L. S. 1996, Phys. Rev. D, 53, 2878
\bibitem[\protect\citeauthoryear{Fishbach et al.}{2018}]{Fishbach2018}Fishbach M., Holz D. E., Farr W. M. 2018, ApJL, 863, L41
\bibitem[\protect\citeauthoryear{Griest et al.}{1991}]{Griest1991} Griest K. 1991, ApJ, 366, 412
\bibitem[\protect\citeauthoryear{Hardy et al.}{2017}]{Hardy2017} Hardy E. 2017, JHEP, 02, 046
\bibitem[\protect\citeauthoryear{Hou et al.}{2019}]{Hou2019} Hou S., Fan X.-L., Liao K., Zhu Z.-H. 2019, arXiv: 1911.02798
\bibitem[\protect\citeauthoryear{Ji et al.}{2018}]{Ji2018} Ji L., Kovetz E. D., Kamionkowski M. 2018, Phys. Rev. D, 98, 123523
\bibitem[\protect\citeauthoryear{Jung et al.}{2019}]{Jung2019} Jung S., Shin C. S. 2019, Phys. Rev. Lett., 122, 041103
\bibitem[\protect\citeauthoryear{Kovetz et al.}{2017}]{Kovetz2017}Kovetz E. D., Cholis I., Breysse P. C.,  Kamionkowski M. 2017, Phys. Rev. D, 95, 103010
\bibitem[\protect\citeauthoryear{Lai et al.}{2018}]{Lai2018} Lai K.-H., Hannuksela O. A., Herrera-Martin A.,  et al. 2018, Phys. Rev. D, 98, 083005
\bibitem[\protect\citeauthoryear{Liao et al.}{2017}]{Liao2017} Liao K., Fan X.-L., Ding X., Biesiada M., Zhu Z.-H. 2017, Nature Communications, 8, 1148
\bibitem[\protect\citeauthoryear{Liao et al.}{2019}]{Liao2019} Liao K., Biesiada M., Fan X.-L. 2019, ApJ, 875. 139
\bibitem[\protect\citeauthoryear{Li et al.}{2018}]{Li2018} Li S.-S., Mao S., Zhao Y., Lu Y. 2018, MNRAS, 476, 2220
\bibitem[\protect\citeauthoryear{Mediavilla et al.}{2009}]{Mediavilla2009} Mediavilla E., Mu\~{n}oz J. A., Falco E., et al. 2009, ApJ, 706, 1451
\bibitem[\protect\citeauthoryear{Monroy-Rodriguez et al.}{2014}]{Monroy-Rodriguez2014} Monroy-Rodr\'{\i}guez M. A., Allen C. 2014, ApJ, 790, 159
\bibitem[\protect\citeauthoryear{Munoz et al. 2016}{}]{Munoz2016} Mu\~{n}oz J. B., Kovetz E. D., Dai L., Kamionkowski M. 2016, Phys. Rev. Lett., 117, 091301
\bibitem[\protect\citeauthoryear{Mishra et al.}{2010}]{Mishra2010} Mishra C. K., Arun K. G., Iyer B. R., Sathyaprakash B. S. 2010, Phys. Rev. D, 82, 064010
\bibitem[\protect\citeauthoryear{Nakamura et al.}{1998}]{Nakamura1998} Nakamura, T. T. 1998, Phys. Rev. Lett., 80, 6
\bibitem[\protect\citeauthoryear{Niikura et al.}{2017}]{Niikura2017} Niikura H., Takada M., Yasuda N.,  et al. 2017, Nature Astronomy, 3, 524
\bibitem[\protect\citeauthoryear{Oguri et al.}{2018}]{Oguri2018} Oguri M., Diego J. M., Kaiser N., Kelly P. L., Broadhurst T. 2018, Phys. Rev. D, 97, 023518
\bibitem[\protect\citeauthoryear{Oguri}{2018}]{Oguri2018a} Oguri, M. 2018, MNRAS, 480, 3842
\bibitem[\protect\citeauthoryear{Pooley et al.}{2009}]{Pooley2009} Pooley D., Rappaport S., Blackburne J., et al. 2009, ApJ, 697, 1892
\bibitem[\protect\citeauthoryear{Quinn et al.}{2009}]{Quinn2009} Quinn D. P., Wilkinson M. I., Irwin M. J.,  et al. 2009, MNRAS, 396, 11
\bibitem[\protect\citeauthoryear{Ricotti et al.}{2009}]{Ricotti2009} Ricotti M., Gould A. 2009, ApJ, 707, 979
\bibitem[\protect\citeauthoryear{Ricotti et al.}{2008}]{Ricotti2008} Ricotti M., Ostriker J. P., Mack K. J. 2008, ApJ, 680, 829
\bibitem[\protect\citeauthoryear{Sasaki et al.}{2016}]{Sasaki2016} Sasaki M., Suyama T., Tanaka T., Yokoyama S. 2016, Phys. Rev. Lett., 117, 061101
\bibitem[\protect\citeauthoryear{Sereno et al.}{2010}]{Sereno2010} Sereno M., Bleuler A., Jetzer P.,  et al. 2010, Phys. Rev. Lett., 105, 251101
\bibitem[\protect\citeauthoryear{Smith et al.}{2018}]{Smith2018} Smith G. P., Jauzac M., Veitch J.,  et al. 2018, MNRAS, 475, 3823
\bibitem[\protect\citeauthoryear{Takahashi et al.}{2003}]{Takahashi2003} Takahashi R., Nakamura, T. 2003, ApJ, 595, 1039
\bibitem[\protect\citeauthoryear{Takahashi et al.}{2017}]{Takahashi2017} Takahashi R. 2017, ApJ, 835, 103
\bibitem[\protect\citeauthoryear{Tisserand et al.}{2007}]{Tisserand2007} Tisserand P., et al. [EROS-2 Collaboration], 2007, A\&A, 469, 387
\bibitem[\protect\citeauthoryear{Udalski et al.}{2015}]{Udalski2015} Udalski A., Szyma\'{n}ski M. K., Szyma\'{n}ski G. 2015, Acta Astromonica, 65, 1
\bibitem[\protect\citeauthoryear{Wang et al.}{1996}]{Wang1996} Wang Y., Stebbins A., Turner E. L. 1996, Phys. Rev. Lett., 77, 2875
\bibitem[\protect\citeauthoryear{Wei et al.}{2017}]{Wei2017} Wei J.-J., Wu X.-F. 2017, MNRAS, 472, 2906
\bibitem[\protect\citeauthoryear{Wilkinson et al.}{2001}]{Wilkinson2001} Wilkinson P. N., Henstock D. R., Browne I. W. A.,  et al. 2001, Phys. Rev. Lett., 86, 584
\bibitem[\protect\citeauthoryear{Wyrzykowski et al.}{2011}]{Wyrzykowski2011}  Wyrzykowski L., et al. 2011, MNRAS, 416, 2949
\bibitem[\protect\citeauthoryear{Yang et al.}{2019}]{Yang2019}Yang L., Ding X., Biesiada M., Liao K., Zhu Z.-H. 2019, ApJ, 874, 139Y


\end{thebibliography}

\label{lastpage}

\end{document}
