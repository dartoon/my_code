%% Beginning of file 'sample631.tex'
%%
%% Modified 2021 March
%%
%% This is a sample manuscript marked up using the
%% AASTeX v6.31 LaTeX 2e macros.
%%
%% AASTeX is now based on Alexey Vikhlinin's emulateapj.cls 
%% (Copyright 2000-2015).  See the classfile for details.

%% AASTeX requires revtex4-1.cls and other external packages such as
%% latexsym, graphicx, amssymb, longtable, and epsf.  Note that as of 
%% Oct 2020, APS now uses revtex4.2e for its journals but remember that 
%% AASTeX v6+ still uses v4.1. All of these external packages should 
%% already be present in the modern TeX distributions but not always.
%% For example, revtex4.1 seems to be missing in the linux version of
%% TexLive 2020. One should be able to get all packages from www.ctan.org.
%% In particular, revtex v4.1 can be found at 
%% https://www.ctan.org/pkg/revtex4-1.

%% The first piece of markup in an AASTeX v6.x document is the \documentclass
%% command. LaTeX will ignore any data that comes before this command. The 
%% documentclass can take an optional argument to modify the output style.
%% The command below calls the preprint style which will produce a tightly 
%% typeset, one-column, single-spaced document.  It is the default and thus
%% does not need to be explicitly stated.
%%
%% using aastex version 6.3
%\documentclass[linenumbers]{aastex631}
\documentclass[twocolumn]{aastex631}

\newcommand{\vdag}{(v)^\dagger}
\newcommand\aastex{AAS\TeX}
\newcommand\latex{La\TeX}
\newcommand{\todo}[1]{\textcolor{red}{[{\bf TODO}: #1]}}
\newcommand{\ding}[1]{\textcolor{red}{[{\bf Xuheng}: #1]}}

\def\smass{{$M_*$}}
\def\sersic{S\'ersic}
\def\halpha{${\rm H}\alpha$}
\def\hbeta{${\rm H}\beta$}
\def\mbh{$\mathcal M_{\rm BH}$}
\def\lcdm{$\Lambda$CDM}
\def\hst{{\it HST}}
\def\kms{km~s$^{\rm -1}$}

%%%%%%%%%%%%%%%%%%%%%%%%%%%%%%%%%%%%%%%%%%%%%%%%%%%%%%%%%%%%%%%%%%%%%%%%%%%%%%%%
%%
%% The following section outlines numerous optional output that
%% can be displayed in the front matter or as running meta-data.
%%
%% If you wish, you may supply running head information, although
%% this information may be modified by the editorial offices.
\shorttitle{Simulation comparison}
\shortauthors{Ding et al.}
%%%%%%%%%%%%%%%%%%%%%%%%%%%%%%%%%%%%%%%%%%%%%%%%%%%%%%%%%%%%%%%%%%%%%%%%%%%%%%%%
\graphicspath{{./}{figures/}}
%% This is the end of the preamble.  Indicate the beginning of the
%% manuscript itself with \begin{document}.

\begin{document}
\title{Testing the Fidelity of Simulations of Black Hole -- Galaxy Coevolution}

\author[0000-0001-8917-2148]{Xuheng Ding}
\affiliation{Kavli Institute for the Physics and Mathematics of the Universe, The University of Tokyo, Kashiwa, Japan 277-8583 (Kavli IPMU, WPI)}

\author[0000-0002-1605-915X]{Junyao Li}
\affiliation{Kavli Institute for the Physics and Mathematics of the Universe, The University of Tokyo, Kashiwa, Japan 277-8583 (Kavli IPMU, WPI)}

\author[0000-0002-0000-6977]{John D.Silverman}
\affiliation{Kavli Institute for the Physics and Mathematics of the Universe, The University of Tokyo, Kashiwa, Japan 277-8583 (Kavli IPMU, WPI)}


\author[0000-0002-8460-0390]{Tommaso Treu}
\affiliation{Department of Physics and Astronomy, University of California, Los Angeles, CA, 90095-1547, USA}

\author{Friends}
\affiliation{Kavli Institute for the Physics and Mathematics of the Universe, The University of Tokyo, Kashiwa, Japan 277-8583 (Kavli IPMU, WPI)}

%% Mark off the abstract in the ``abstract'' environment. 
\begin{abstract}
We investigate the evolution of the scaling relations between the supermassive black hole (BH) and its host galaxy stellar mass (\smass-\mbh) by making comparisons between observations and simulations at $0.2<z<1.7$. The measurements of the observational data are consistent of 612 uniformly-selected SDSS quasars at $0.2 < z < 0.8$ that imaged by Hyper Suprime-Cam (HSC) and 32 X-ray-selected, broad-line AGNs imaged by the Hubble Space telescope (HST) at $1.2<z<1.7$. We make direct comparisons by first adding realistic observational error to the simulation data and then making selection to the sample follows real observation condition. Our results show that all the simulations predict consistent scatter of the scaling relations to both HSC and HST sample, across board redshift range. %We also find that the selection criteria for the $z<1$ do not produce strong effects on the simulated relation. 
The scatter of the scaling relation is dominated by the observational error. Thus, the original relations (i.e., before adding noise) predicted by the simulation are supposed to have consistent intrinsic scatter compared with the observational data. Considering that the tight relation of the scaling relation is highly relay on the AGN feedback, our results support the hypothesis of the AGN feedback being responsible for a causal link between the BH and its host galaxy, resulting in a tight correlation between their respective masses.
%\todo{Add comments on the center positions. KS test. Estimation of the intrinsic scatter.}
\end{abstract}
\keywords{Galaxy evolution (594); Active galaxies (17); Active galactic nuclei (16)}

\section{Introduction} \label{sec:intro}
The close correlations between the supermassive black holes and their host galaxies properties (stellar mass in particular) indicates a physical coupling during their coevolution~\citep{Mag++98, F+M00, M+H03, H+R04, Gul++09}. To understand the nature of this connection, considerable efforts have been focused to measure the correlation using board-line active galactic nucleus (AGN) to measure them as a function of redshift, intending to see how and when the correlation emerge and evolve over cosmic time \citep[e.g.,][]{Peng2006a, Tre++07, Woo++08, Jahnke2009, Bennert11, Schramm2013, Park15, Mechtley2016, Ding2020, 2021arXiv210902751L}. While, an {\it apparent} evolution has been found in which the growth of BHs predates that of the host, equally as many studies claim a non-evolution when considering the host total stellar mass. To understand the significance {\it intrinsic} evolution, it is important to take into account the systematic uncertainty of the measurement and the selection effect~\citep{Tre++07, Lauer2007, Schulze2014}. 

In theory, various models have been proposed to explain the origin of the scaling relation. For instance, the AGN feedback is considered as one of the possible physical link. During this process, the fraction of the AGN energy is injected into its surrounding gas, which regulates the mass growth of the BH and its host. In this scene, the star formation is inhibited by the heating and unbinding of a significant fraction of the gas by the AGN activity. Alternatively, the scaling relation can be explained through a more indirect connection in which AGN accretion and host star formation are fed through a common gas supply~\citep{Cen2015, Menci2016}. Without any physical mechanisms, the statistical convergence from galaxy assembly alone (i.e., dry mergers) could regulate the observed correlations~\citep{Peng2007, Jahnke2011, Hirschmann2010}. From this central limit theorem, higher dispersion would appear in the scaling relations at high-$z$ as observed today. 

Numerical simulation provides good opportunity to understand the origin of the connection. For example, the comparison of the scaling relations has been made between the predictions by the state-of-the-art cosmological hydrodynamical simulation of structure formation ({\tt MassiveBlackII}) and the HST observational measurements at  $0.3<z<1$, which show a positive evolution that BH growth predates than its host~\citep{DeG++15}. The efforts have been taking to use large-volume simulations to investigate the scaling relations and find good agreement with the local relation with redshift evolution, including the Magneticum Pathfinder smooth particle hydrodynamics (SPH) Simulations~\citep{Steinborn2015}, the Evolution and Assembly of Galaxies and their Environments suite of SPH simulations~\citep{Schaye2015}, Illustris moving mesh simulation~\citep{Sijacki2015, Vogelsberger2014, Li2019} and SIMBA simulation~\citep{Thomas2019}. In recent work, the BH populations from the six large scale cosmological simulations have been adopted to compare the \mbh-\smass\ with observation of the local universe~\citep{Habouzit2021}. However, these comparison works are limited by the observation data in terms of the sample size ($<$100) and the redshift range ($z<1$).

In the meanwhile, it is crucial to consider the systematic uncertainties and selection biases in the comparison. A direct way to take them in to account is to apply the same effects to the simulation products and make a forward comparison in the observational plane. 
In~\citet{Ding2020b}, this direct comparison has been performed using 32 X-ray-selected AGN at $1.2<z<1.7$ and direct compare with two state-of-the-art simulation efforts, including {\tt MassiveBlackII} (MBII) and Semianalytic Model (SAM). The dispersion in the mass ratio between black hole mass and stellar mass is significantly more consistent with the MBII prediction ($\sim0.3$~dex) confirming the hypothesis of AGN feedback being responsible for a causal link between the BH and its host galaxy.

In this study, we extend our previous comparison work by adding the latest measurements of hundreds SDSS quasar $0.2<z<0.8$ using the latest HSC observation. We adopt more simulated quasar populations including MBII, SAM, Illustris, TNG100, TNG300 and Horizon. This paper is structured as follows. In section~\ref{sec:sample}, we describe our observation and simulation sample. The direct comparison is performed and the result is presented in section~\ref{sec:result}. The concluding remarks are presented in section~\ref{sec:con}.

\section{Comparison Sample: Observations and Simulations} \label{sec:sample}
%The correlations of \mbh-\smass\ between the observed data and the numerical simulations are compared in this work. 
We introduce the comparison samples in this section, including the observational data and the numerical simulations.

\subsection{Observation Data}
The observational data are consistent of 900 HSC-observed uniformly-selected SDSS quasars at $0.2<z<0.8$~\citep{Li2021a} and 32 \hst-observed quasars analyzed in \citep[][hereafter D20]{Ding2020} which are selected from three deep-survey fields, namely COSMOS~\citep{Civano2016}, (E)-CDFS-S~\citep{Lehmer2005, Xue2011}, and SXDS~\citep{Ueda2008}. The details of the observational data and their measurements are introduced in following subsections. 

\subsubsection{HSC}\label{sec:hsc}
A sample of $\sim$5000 type-1 SDSS DR14 quasar catalog~\citep{Paris2018} at $0.2<z<1$ has been imaged through high-resolution Subaru Strategic Program (SSP) wide survey~\citep{Aihara2019} by Hyper Suprime-Cam~\citep{Miyazaki2018}. With accurate PSF model and {\it grizy} bands, the two-dimensional quasar-host decomposition has been performed in~\citet[][hereafter L21a]{Li2021a} to obtain the quasar host galaxies flux and the color. The stare-of-the-art image modeling software {\tt lenstronomy}\citep{Birrer2015, Birrer2018, Birrer2021} is adopted to perform the modeling task. This approach is first developed by~\citet{Ding2020}, which is used to decompose the HST sample (see next section). Having inferred the host light, the stellar mass of host galaxy is derived using spectral energy distribution (SED) fitting by CIGALE~\citep{Boquien2019}. A simulation tests are also performed to verify fidelity of the \smass\ measurements. The measurement error of \smass\ is dominated by the Spectral Energy Distribution (SED) fitting which is at a 0.2~dex level. The \mbh\ are taken from~\citet{Rakshit2020} which are estimated by the \hbeta-based measurements using virial method~\citep{Peterson2004, Vestergaard2006}. The typical error of \mbh\ are estimated to be 0.4 dex. We refer the reader to~\citet{Li2021a} in the Section~4.2 for more details.

To avoid any potential biases related to the selection of the quasars, \citet[][hereafter, L21b]{2021arXiv210902751L} adopted the 877 sources which are uniformly selected based on PSF-magnitude, color cuts using single-epoch SDSS photometries and the value of the measured \smass. Specifically, we use the {\it ugri} color-selected sample (228 sources) from SDSS I/II~\citep{Richards2002}, and the CORE sample from SDSS BOSS (408 sources) and eBOSS (241 sources) surveys~\citep{Ross2013, Myers2015} (hereafter the uniform sample). These samples are initially selected based on PSF-magnitude cuts of $15 < i < 19.1$ (for {\it ugri}) and $i > 17.8 \& g/r < 22.0$ (for CORE). Finally, considering that the accuracy of the measured \smass\ is higher when \smass\ is increasing, a cut on \smass\ is set to assure the measurement accuracy. These selections will be adopted to the simulation sample to erase the selection effect and allow the fair comparison.

\subsubsection{HST}
A sample of 32 HST-observed AGN systems across the redshift range $1.2<z<1.7$ are selected from three deep-survey fields (COSMOS, (E)-CDFS-S, and SXDS). HSC/WFC3 is used to obtain the high-resolution imaging data (HST program GO-15115, PI: John Silverman) with six dither pattern and a total exposure time $\sim$2348~s. The filters F125W ($1.2<z<1.44$) and F140W ($1.44<z<1.7$) were employed, according to the redshift of each target.  The AGN images are analyzed and decomposed to infer the host galaxies fluxes using the approached developed by D20 based on {\tt lenstronomy}. The HST ACS/F814W imaging data for 21/32 of the AGNs is also used to infer the host color. The results show that the 1 and 0.625~Gyr stellar templates could well match the sample color at $z<1.44$ and $z>1.44$, respectively (see Figure 5 in D20), which are adopted to estimate the stellar masses of the host galaxies. We refer the interested reader to D20 for more detailed description. 
The \mbh\ is determined using published near-infrared spectroscopic observations of the broad \halpha\ emission line with recipe provided by~\citep{Vestergaard2006}, the recipe is consistent with the one adopted by HSC sample \mbh. 

Based on the consistent approach, the measurements of the \mbh-\smass\ relations for the HST sample and the HSC sample are obtained. Thus, we expect the measurement error of these two samples to be at the same level (i.e., $\Delta$\mbh$=0.4~$dex, $\Delta$\smass$=0.2~$dex). 
Indeed, the HST and HSC sample demonstrate consistent evolution distribution in L20b Figure 6, though the sample selection is slightly different.

\subsection{Simulation}
We introduce the simulation sample that adopted in this section. All the simulation are based on the larger-scale cosmological simulations, except the Semianalytic Model (SAM) simulation.
%\todo{Maybe we can also summarize the importance of AGN feedback somewhere here?}

\subsubsection{{\tt MassiveBlackII} (MBII)}
MBII is a high-resolution cosmological hydrodynamic simulation that has a box size of $100~\mathrm{cMpc/h}$ and $2\times1792^3$ particles. The simulation is based on smooth particle hydrodynamic (SPH) code \texttt{P-GADGET}, a hybrid version of the parallel code {\tt GADGET}~\citep{2005MNRAS.364.1105S}. For dark matter and gas, the elements of the masses resolution are $1.1\times 10^7~M_{\odot}/h$ and $2.2\times 10^6~M_{\odot}/h$, respectively. The simulation includes a full modeling of gravity plus gas hydrodynamics, with a wide range of subgrid recipes to model the star formation~\citep{2003MNRAS.339..289S}, BH growth, and the feedback process. Halos were identified using a friends-of-friends (FOF) group finder~\citep{1985ApJ...292..371D}. Galaxies are identified with the stellar matter components of subhalos; these subhalos are identified using {\tt SUBFIND} within the halos~\citep{2005MNRAS.364.1105S}. The base cosmology parameters are based on the WMAP7 results~\citep{2011ApJS..192...18K}, i.e., $\Omega_0=0.275$, $\Omega_l=0.725$, $\Omega_b=0.046$, $\sigma_8=0.816$, $h = 0.701$, $n_s=0.968$.

To model the growth of BH, a feedback prescription is adopted as detailed in the literature~\citep{2005Natur.433..604D, 2005MNRAS.361..776S}. In particular, the BH seed with mass $5\times 10^{5}~M_{\odot}/h$ are embedded into halos of mass $\gtrsim 5\times 10^{10}~M_{\odot}/h$. Once seeded, BH growth via gas accretion is assigned at a rate of $\dot{M}_{BH}={4\pi G^2 M_{BH}^2 \rho}/{(c_s^2+v_{BH}^2)^{3/2}}$ where $\rho$ and $c_s$ are the density and sound speed of the interstellar medium (ISM) gas at cold phase; $v_{BH}$ is the relative velocity between the BH and its surrounding gas. The accreted gas is released as radiation at a radiative efficiency of 10\%. A fraction of 5\% of the radiated energy couples to the surrounding gas as black hole (or AGN) feedback~\citep{2005Natur.433..604D}. A mildly super-Eddington (two times Eddington rate) is allowed; however, unlike some previous work, the accretion rate in MBII adopt the prescription in~\citet{Pelupessy2007}, which does not include any artificial boost factor. Two BH are considered to be merged when their separation distance is below the simulation spatial resolution (i.e., the SPH smoothing length) and their relative speeds are lower than the local sound speed of the medium.

For the galaxy photometry, the initial SEDs of the BH host galaxies were determined by combining all the contributions from the overall start particles. The stellar SEDs were derived using the PEGASE-2~\citep{1999astro.ph.12179F} stellar population synthesis code with a Salpeter IMF. Finally, the galaxy SEDs are  cooperated with the desired filter response to obtain the broadband photometry (e.g., SDSS $r$-band magnitude).

As a common practice, the stellar mass is obtained by using a 3D spherical aperture of 30~kpc to represent the observed stellar mass. By this definition, it has been tested that the corresponding stellar mass function is consistent with the observation measurement~\citep[e.g.,][]{Pillepich2018}. Even more, the stellar mass using this 3D aperture can achieve good agreement to those measured within Petrosian radii in observational studies~\citep{Schaye2015}. 

For further details of MBII simulation, we refer the reader to~\citet{Khandai2015}

\subsubsection{Semianalytic Model (SAM)}
We highlight here the main points of the simulation with respect to our study; for more detail, a full descriptions of the SAM simulation sample can be found in~\citet{Menci2016} which is based on semianalytic model introduced in~\citet{Menci2014}. For dark matter halos that merge with a larger halo, the impact of dynamical friction is assessed to define whether the halo will survive as a satellite or sink to the center of the dominant galaxy to increase its mass. The binary interactions (fly-by and merging), among satellite subhalos, are  also described by the model. In each halo, we compute the fraction of gas which cools because of the atomic processes and settles into a disk~\citep{Mo1998}. The stars are converted from the gas through three channels: (1) quiescent star formation with long time scales: $\sim1$~Gyr; (2) starbursts following galaxy interactions with timescales $\lesssim 100$~Myr, according to BH feeding; (3) the loss of angular momentum triggered by the internal disk instabilities causing the gas inflows to the center, resulting in stimulating star formation (as well as BH accretion). The stellar feedback is also considered by calculating the energy released by the supernovae associated with the total star formation which returns a fraction of the disk gas into a hot phase. A Salpeter IMF is adopted in the SAM simulation.

The BHs primordial seeds are assumed to originate from PopIII stars with a mass $M_{seed}=100\,M_{\odot}$~\citep{Madau2001}, and to be initially present in all galaxy progenitors. The BH accretion is based on interactions feeding mode and disk instabilities feeding mode.
%\newline(1)~{\it triggered by interactions.} The interaction rate $\tau_r^{-1}=n_T\,\Sigma (r_t,v_c,V)\,V_{rel} (V)$ for galaxies with relative velocity $V_{rel}$ and number density $n_T$ in a common DM halo determines the probability for encounters, either fly-by or merging, through the corresponding cross sections $\Sigma$ given in~\citet{Menci2014}. The fraction of gas destabilized in each interaction corresponds to the loss $\Delta j$ of orbital angular momentum $j$, and depends on the mass ratio of the merging partners $M'/M$ and on the impact factor $b$.
%\newline(2)~{\it induced by disc instabilities.} We assume these to arise  in  galaxies with disc mass exceeding~\citep{Efstathiou1982} $M_{crit} =  {v_{max}^2 R_{d}/ G \epsilon}$ with $\epsilon=0.75$, where $v_{max}$ is the maximum circular velocity associated to each halo ~\citep{Mo1998}.  Such a criterion strongly suppresses the probability for disc instabilities to occur not only in massive, gas-poor galaxies, but also in dwarf galaxies characterized by small values of the gas-to-DM mass ratios. The instabilities induce loss of angular momentum resulting into strong inflows that we compute following the description in~\citet{Hopkins2011}, recast and extended as in~\citet{Menci2014}. 
\todo{This part should be updated for the new SAM version?}
A detailed treatment of AGN feedback is included, which is introduced in~\citet{Menci2008}. The assumption stems from the fast winds with velocity up to 
$10^{-1}c$ observed in the central regions of AGNs~\citep{Chartas2002, Pounds2003}. The supersonic outflows compress the gas into a blast wave terminated by a leading shock front, which moves outward with a lower but still supersonic speed, and sweeps out the surrounding medium. The model follows in detail the expansion of the blast wave through the galaxy disk, and computes the fraction of gas expelled from the galaxy. ~\citep[see, i.e.,][for the details.]{Menci2016}. 


\subsubsection{Illustris}
The Illustris Project is a serise of larger-scale hydrodynamical simulations of galaxy formation which was introduced in~\citet{2014MNRAS.444.1518V, 2014Natur.509..177V}. The simulation consist of a volume of (106.5 cMpc)$^3$, and was run with the moving Voronoi mesh code {\tt Arepo}~\citep{2010MNRAS.401..791S}. Besides gravitational force and hydrodynamical fluxes, the simulation calculates the astrophysical process known to be crucial for galaxy formation~\citep{2013MNRAS.436.3031V, 2014MNRAS.438.1985T}, which includes gas cooling, a subresolution ISM model, stochastic SF with a density threshold of 0.13 cm$^{-3}$, stellar evolution, gas recycling, chemical enrichment, kinetic stellar feedback by SNe activity, BH activity (accretion and merging), and AGN feedback. A cosmological background consistent with WMAP9~\citep{2013ApJS..208...19H} and a Chabrier IMF is adopted.

The BH seeding in the simulation is based on DM halo mass -- when a halo reaching a mass of $7.10 \times 10^{10} M_{\odot}$ is inserted with a BH with initial mass of $1.42 \times 10^5 M_{\odot}$. Then, the BH accretion follows the Bondi-Hoyle-Lyttleton formalism ($\alpha4\pi G^2M_{BH}^2 \rho/c_s^3$), but capped at the Eddington limit. The boost factor $\alpha$ is introduced to compensate that the ISM multiphase structure is unresolved~\citep{Springel2005, 2009MNRAS.398...53B}, since simulation usually underestimate the density around a BH.

The AGN feedback consists of three components: thermal quasar-mode feedback; thermal-mechanical radio-mode feedback; and radiative feedback. Through these process, the AGNs are able to deposit thermal energy into their surroundings with a coupling efficiency of 0.05 in for BHs with low Eddington ratio (i.e., $f_{\rm Edd} < 0.05$). For BHs with $f_{\rm Edd} > 0.05$, the thermal energy is released in hot bubbles with a radius of $\sim$ 100~kpc and couple to the gas with an efficiency of 0.35. It is highlighted in ~\citet{2014MNRAS.444.1518V} that the AGN feedback is the key process in quenching massive galaxy which is crucial to reproduce a tight stellar mass and BH mass relation. 

\subsubsection{IllustrisTNG}
{\it The Next Generation Illustris Simulations} (IllustrisTNG)~\citep{2018MNRAS.475..676S, Pillepich2018} are an ambitious suite of hydrodynamical simulations of galaxy formation in large cosmological volumes. It builds upon the scientific achievements of the Illustris simulation with improvements upon Illustris by 1) extending the mass range of the simulated galaxies and haloes, 2) adopting an improved numerical and astrophysical modelling, and 3) addressing the identified shortcomings of the previous generation simulations.

The TNG100 and TNG300 have a volume of (111~cMpc)$^3$ and (302~cMpc)$^3$, respectively. They use the same initial condition with the previous Illustris simulation. The adopted cosmological parameters are updates by the Planck result~\citep{2016A&A...594A..13P}.
Dark matter halos with a mass exceeding $7.38 \times 10^{10} M_{\odot}$ are seeded in their center using BHs with initial mass as $1.18 \times 10^6 M_{\odot}$, one order of magnitude higher than the assumptions in the previous Illustris simulation. The BHs accretion also follows the Bondi-Hoyle-Lyttleton formalism, but without any boost factor. The assumption of the magnetic fields can affect the relationship between the BHs and their host galaxies properties; the \mbh-\smass\ mean relation is higher with magnetic fields~\citep{2018MNRAS.473.4077P}. The TNG simulation also employs a kernel-weighted accretion rate over about 256 neighboring cells. Thus, the TNG BH can build up correlation with the gas properties in the galaxy central region.

TNG includes a two-mode AGN feedback: for high accretion rates -- injection of thermal energy in the surroundings of the BHs accreting; for low accretion rates -- directional injection of momentum with random direction~\citep{2017MNRAS.465.3291W, 2018MNRAS.479.4056W}. The transition between two modes is set by the minimum value of 0.1 and $2 \times 10 ^{-3} \times (M_{BH} / 10^8/M_{\odot})$.

\subsubsection{Horizon-AGN}
The simulation of Horizon-AGN~\citep{2014MNRAS.444.1453D, 2016MNRAS.463.3948D} has a volume of (142 cMpc)$^3$ which was generated based on the adaptive mesh refinement code {\tt Ramses}~\citep{2002A&A...385..337T} with a $\Lambda$CDM model based on WMAP7~\citep{2011ApJS..192...18K} cosmological results. The total volume contains $1024^3$ DM particles, corresponding to a DM mass resolution of $8 \times 10^7  M_{\odot}$  and initial gas resolution of  $1\times10^7M_{\odot}$. A Salpeter IMF is assumed in Horizon-AGN simulation.

Different from the previous simulations, Horizon-AGN uses a fixed threshold in the dark matter halo mass. Star formation occurs in regions with gas number density exceeds the star formation threshold $n_0 = 0.1$. The BHs also form in this dense gas cells with stellar velocity dispersion larger than 100 km/s. BH seeds are not formed when they are closer than 50 ckpc of an existing BH. After $z = 1.5$, BHs are prevented from forming. At that time, all the progenitors of the \smass$>10^{10} M_{\odot}$ galaxies at $z = 0$ should be formed and seeded with BHs~\citep{2016MNRAS.460.2979V}. Horizon-AGN simulation also include the feedback from stellar winds, SNe Type Ia and Type II with mass, energy and metal release.

The BH accretion is computed using the Bondi-Hoyle-Lyttleton formalism with a boost factor $\alpha = (\rho/\rho_0)^2$ when the density $\rho$ is higher than the resolution-dependent threshold $\rho_0$. Otherwise, the boost factor is fixed as unity~\citep{2009MNRAS.398...53B}.

Horizon-AGN includes a two mode AGN feedback. In the high state mode ($f_{\rm Edd}<0.01$), thermal energy is isotropically released within a sphere of radius a few resolution elements. The energy deposition rate is $\dot{E}_{\rm AGN} = 0.015 \dot{M}_{\rm BH} c^2$. In the low state mode, energy is injected into a bipolar  outflow  with  a  velocity  of  10$^4$ km/s,  to  mimic the  formation  of  a  jet.  The  energy  rate  in  this  mode is $\dot{E}_{\rm AGN} = 0.1 \dot{M}_{\rm BH} c^2$.  The  technical  details  of  BH  formation,  growth  and  feedback  modeling  of  Horizon-AGN  can be found in~\citet{2012MNRAS.420.2662D} 

%Horizon-AGN  employs  a  kinetic  SN  feedback,  including momentum, mechanical energy and metals from type~II, Type  Ia  SNe,  and  stellar  winds  (details  in  Kaviraj  et  al.2017).  The  feedback  is  modelled  as  kinetic  release  of  energy on timescale below 50 Myr, and a thermal energy after 50 Myr. The feedback is also pulsed, meaning that energy is accumulated until sufficient to propagate the blast wave to  at  least  two  cells.  The  energy  released  depends  on  theSSP modeled assumed and the metallicity of the gas, and is aboute SN $\sim10^{49}$ erg/$M_{\odot}$.


\section{Comparison and Results} \label{sec:result}
\begin{figure*}
\centering
\begin{tabular}{c c}
{\includegraphics[width=0.45\textwidth]{HSC_selection_MBII.png}} &
{\includegraphics[width=0.45\textwidth]{HST_selection_MBII.png}}
\end{tabular}
\caption{\label{fig:selection} 
{\it left}: The correlation between the \mbh\ and $L_{\rm bol}$ of the HSC sample is used to set the selection window for the simulated samples. This region roughly bracket the type-1 AGN sample. Taking MBII as example, the light green background cloud shows the simulated number density after added random uncertainty of the overall sample. 
{\it right}: Similar to the top plane -- equivalent selection window adopted for the HST-observed and MBII simulated samples.
}
\end{figure*} 

To make direct comparison, we adopt the measurement error and the selection effect and apply to the simulation sample. To this end, we first inject random noise to the simulation catalog to mimic the scatter that caused by the measurement error. As mentioned in pervious section, the \smass\ and \mbh\ for HSC and HST samples are estimated by using the similar approach, and thus their uncertainty level is expected to be the same. 
We assume the following measurement uncertainties to add the random noise: $\Delta$\mbh$ = 0.4~$dex, $\Delta$\smass$ = 0.17~$dex, and $\Delta L_{\rm bol} = 0.03~$dex. 

We then make selection to the noise-injected simulation data to consider the selection effect as in the realistic data. Since the HSC sample and the HST sample have different selection function, we adopt different selection criteria to the simulation as follows.
\begin{itemize}
 \item{}To consider the same selection effect in the HSC sample, the following measures are taken. 1) The observed sample is consist of type-1 AGN, and thus the simulated sample should follow consistent \mbh-$L_{\rm bol}$ relation to HSC sample to ensure the fact that all the selected AGN have board line feature. We take the MBII as an example to demonstrate this sample selection, shown in Figure~\ref{fig:selection}--{\it top}. 2) The $i$-band magnitude of the AGN are bright (see Section~\ref{sec:hsc}). The specific selection is made as following; for systems at $z<0.5$ and $z>0.5$, the AGN $i$-band magnitude are required to be brighter than 20.5 mag and 22.0 mag, respectively. Since the simulations do not provide the AGN observed magnitude, we adopt the simulated rest-frame magnitude or L$_{\rm 5100}$ and assume the quasar continuum as a single power-law with an index of $\alpha_\nu=-0.44$~\citep{2001AJ....122..549V} to calculate the observed $i$-band magnitude.
 3)~Following the HSC selection, we require the \smass\ value to be above a certain level (according to their redshift) to assure the accurate of them measurement. Finally, the HSC sample is split into three redshift bins which are $0.2<z<0.4$, $0.4<z<0.6$, and $0.6<z<0.8$.
 
\item{}For HST sample, the selection is relatively simpler. The simulated AGN systems are selected only when they match the  \mbh-$L_{\rm bol}$ targeting window which is same as observation selection, see Figure~\ref{fig:selection}--{\it bottom} taken MBII as example. Note that the sample selection of the HST sample has a hard cut on the \mbh\ values (i.e., between [7.7, 8.6] $M_{\odot}$). The HST sample covers the higher redshift range $1.2<z<1.7$, which is considered as a single redshift bin to make the comparison with the simulations at $z=1.5$.
\end{itemize}

Having selected the sample, the comparison is performed as follows. We use the local relation \mbh$=0.98$\smass$-2.56$ adopted by D20 (Chabrier based\footnote{Since different simulations adopt either the Chabrier or the Salpeter IMF, we will adopt the local relation and the \smass\ of the observation data based on the same IMF to ensure that the comparison with the observation data are self-consistent.}) as the fiducial to estimate the sample offset and scatter.  In Figure~\ref{fig:comparsion}, we present the comparison of the scaling relation \mbh-\smass\ between the observation and simulations. Different simulations are presented in each row and the redshift evolution is shown from left to right. 
For both observation and simulation sample, the central offsets and scatters of the scaling relation are estimated by the mean and the standard deviation of the \mbh\ residuals for each system\footnote{The slope value of the local sample is close to 1, and thus if taking the \smass\ to calculate the residual  for each system, the value remains the same.} (i.e., the offset to the local relation along the y-axis in Figure~\ref{fig:comparsion}).
A collection of these central offset and scatter values is presented in Table~\ref{tab:sum}. The simulation sample has been added with random error; we use Monte Carlo approach to infer their uncertainties of the the central offsets and scatters for the simulation sample. The comparison process including the noise injection, the sample selection and the offsets estimation are repeated by up to 1,000 times until the result is stable. To visually inspect the comparison results, we show the offsets to the local relation (in terms of the $\Delta{\rm log}$\mbh) as a function of redshift using sample both observation and simulations in Figure~\ref{fig:offsets_vz}. We also summarize the histogram of these offsets distributions based on each redshift bins, as shown in Figure~\ref{fig:offsets}.

The absolute value of stellar masses in both observation and theory have significant uncertainty (up to factor of two), which depends on the assumption of initial mass function, and possibly on the implementation of star formation in the models. In contrast, the scatter around the mean correlation is a relative quantity, which is less affected by this uncertainty. Thus, in this work, we take the comparison and mainly focus on the scatter as a direct diagnostic criteria. The results show that, almost all the simulations predict consistent scatter value to the observation. For simulation sample at $z<1$, this scatter level is $\sim0.5$~dex, while for $z>1$, this scatter level is $\sim0.3$~dex. Note that the HST sample $z>1$ has narrow selection window based on the \mbh (see Figure~\ref{fig:selection}~bottom), leading to the result that the observed scatter is smaller than HSC sample at $z<1$.
\ding{In the simulation, is there any observational data that be used to give extra constrain on the recipe? Xuheng and John are surprised to see that all of them have very good scatter matches!}

We also investigate the central offset to understand how the simulations predict the {\it observed} evolution of the scaling relation. According to the results in Table~\ref{tab:sum} and Figure~\ref{fig:offsets_vz}, at low redshift range $z<0.6$, the Illustris is the only simulation that can predict the {\it observed} \mbh\ offset at a level of $<0.1$ dex. At higher redshift $0.6<z<1.5$, the simulations SAM, TNG100, TNG300 and Horizon can predict the {\it observed} evolution well. These results are consistent with the Kolmogorov-Smirnov (KS) test performed using the offset distributions between the simulation and the observation. The inference of the p-values are collected in Table~\ref{tab:pvalue}, showing that only the Illustris presents good match of the scaling relation at $z<0.6$ (i.e., p-value $> 0.1$), while the TNG100, TNG300 and Horizon simulation predict well at $z>0.6$. 

We recognized that the observed scatters are dominated by the measurement uncertainties in the data. Understand how much of the scatters derive from random noise can help us to extrapolate the intrinsic scatter in the scaling relation. To this end, we measure the scatter of the simulation sample without injecting the random noise but adopting the same selecting window for both $z<1$ and $z>1$ sample to infer the central offset and scatter value. We find that the scatters by this measure are at a level of $\sim0.15-0.2$ dex for both $z<1$ and $z>1$ (see Table~\ref{tab:sum_no_noise}). These levels are consistent with the {\it intrinsic} scatter as estimated using observation data by~\citet{Ding2020, 2021arXiv210902751L}. More important, these scatters values are redshift-independent, suggesting that the tight scaling relation could not be explained by a pure stochastic process, i.e., random mergers. Still, these scatters are affected by the sample selection, and thus these levels can only be taken as an approximation of the intrinsic scatter.
%their physical meaning should not be considered as the intrinsic scatter of the overall sample.

%Taking \mbh\ and \smass\ as independent measurements, we also summarize their distributions and make comparison between the observation and simulation in the Appendix~\ref{app:function}.

\begin{figure*}
\centering
\begin{tabular}{c c c c}
%\vspace*{-0.4cm} 
{\includegraphics[trim = 0mm 30mm 80mm 25mm, clip, height=0.25\textwidth]{MM_SAM_zs_03.png}}&
\hspace*{-0.5cm} 
{\includegraphics[trim = 44mm 30mm 80mm 25mm, clip, height=0.25\textwidth]{MM_SAM_zs_05.png}}&
\hspace*{-0.5cm} 
{\includegraphics[trim = 44mm 30mm 80mm 25mm, clip, height=0.25\textwidth]{MM_SAM_zs_07.png}}&
\hspace*{-0.5cm} 
{\includegraphics[trim = 30mm 30mm 80mm 25mm, clip, height=0.25\textwidth]{MM_SAM_zs_15.png}}\\
{\includegraphics[trim = 0mm 30mm 80mm 25mm, clip, height=0.25\textwidth]{MM_MBII_zs_03.png}}&
\hspace*{-0.5cm} 
{\includegraphics[trim = 44mm 30mm 80mm 25mm, clip, height=0.25\textwidth]{MM_MBII_zs_06.png}}&
\hspace*{-0.5cm} &
\hspace*{-0.5cm} 
{\includegraphics[trim = 30mm 29mm 80mm 25mm, clip, height=0.25\textwidth]{MM_MBII_zs_15.png}}\\
{\includegraphics[trim = 0mm 30mm 80mm 25mm, clip, height=0.25\textwidth]{MM_Illustris_zs_03.png}}&
\hspace*{-0.5cm} 
{\includegraphics[trim = 44mm 30mm 80mm 25mm, clip, height=0.25\textwidth]{MM_Illustris_zs_05.png}}&
\hspace*{-0.5cm} 
{\includegraphics[trim = 44mm 30mm 80mm 25mm, clip, height=0.25\textwidth]{MM_Illustris_zs_07.png}}&
\hspace*{-0.5cm} 
{\includegraphics[trim = 30mm 30mm 80mm 25mm, clip, height=0.25\textwidth]{MM_Illustris_zs_15.png}}
\end{tabular}
\caption{\label{fig:comparsion} 
The comparison of observed black hole mass vs. stellar mass relation from observation data and simulation products (by each row). The panels from left to right are based on different redshift bin. The black line in the panel indicates the local relation adopted by~\citet{Ding2020}. The background cloud shows the simulated number density with random noise that before selection.
\todo{Xuheng will zoom in the figures and enlarger the text to show the them more clear.}
}
\end{figure*} 
\renewcommand\thefigure{\arabic{figure}}
\setcounter{figure}{1}  
\begin{figure*}
\centering
\begin{tabular}{c c c c}
%\vspace*{-0.4cm} 
{\includegraphics[trim = 0mm 30mm 80mm 25mm, clip, height=0.25\textwidth]{MM_TNG_zs_03.png}}&
\hspace*{-0.5cm} 
{\includegraphics[trim = 44mm 30mm 80mm 25mm, clip, height=0.25\textwidth]{MM_TNG_zs_05.png}}&
\hspace*{-0.5cm} 
{\includegraphics[trim = 44mm 30mm 80mm 25mm, clip, height=0.25\textwidth]{MM_TNG_zs_07.png}}&
\hspace*{-0.5cm} 
{\includegraphics[trim = 30mm 30mm 80mm 25mm, clip, height=0.25\textwidth]{MM_TNG_zs_15.png}}\\
{\includegraphics[trim = 0mm 30mm 80mm 25mm, clip, height=0.25\textwidth]{MM_TNG300_zs_03.png}}&
\hspace*{-0.5cm} 
{\includegraphics[trim = 44mm 30mm 80mm 25mm, clip, height=0.25\textwidth]{MM_TNG300_zs_05.png}}&
\hspace*{-0.5cm} 
{\includegraphics[trim = 44mm 30mm 80mm 25mm, clip, height=0.25\textwidth]{MM_TNG300_zs_07.png}}&
\hspace*{-0.5cm} 
{\includegraphics[trim = 30mm 30mm 80mm 25mm, clip, height=0.25\textwidth]{MM_TNG300_zs_15.png}}\\
{\includegraphics[trim = 0mm 30mm 80mm 25mm, clip, height=0.25\textwidth]{MM_Horizon_zs_03.png}}&
\hspace*{-0.5cm} 
{\includegraphics[trim = 44mm 30mm 80mm 25mm, clip, height=0.25\textwidth]{MM_Horizon_zs_05.png}}&
\hspace*{-0.5cm} 
{\includegraphics[trim = 44mm 30mm 80mm 25mm, clip, height=0.25\textwidth]{MM_Horizon_zs_07.png}}&
\hspace*{-0.5cm} 
{\includegraphics[trim = 30mm 30mm 80mm 25mm, clip, height=0.25\textwidth]{MM_Horizon_zs_15.png}}\\
\end{tabular}
\caption{
continued.
}
\end{figure*} 

\begin{deluxetable*}{lccccc}
%\tablenum{1}
\tablecaption{Summary of the central offsets and scatters\label{tab:sum}}
\tablewidth{0pt}
\tablehead{
\colhead{Simulation} &  \multicolumn{3}{c}{HSC comparison (offset, scatter)} & \colhead{HST comparison (offset, scatter)} \\
\cline{2-4} 
%  \cline{2-4}  \cline{5} \\
\colhead{} &  \colhead{$0.2<z<0.4$} & \colhead{$0.4<z<0.6$} & \colhead{$0.6<z<0.8$} & \colhead{$1.2<z<1.7$} & IMF
}
\startdata
Observation & (0.07, 0.58) & (0.17, 0.52)  & (0.18, 0.54)  & (0.43, 0.31) \\
SAM & (0.73$\pm$0.01, 0.49$\pm$0.01) & (0.65$\pm$0.01, 0.46$\pm$0.01)  & (0.51$\pm$0.02, 0.45$\pm$0.02)  & (0.51$\pm$0.01, 0.36$\pm$0.01) & Salpeter \\
MBII & (-0.15$\pm$0.02, 0.48$\pm$0.01) & \multicolumn{2}{c}{(-0.16$\pm$0.02, 0.48$\pm$0.01) [$z=0.6$]}  & (0.14$\pm$0.02, 0.31$\pm$0.01) & Salpeter\\
Illustris & (0.01$\pm$0.03, 0.52$\pm$0.02) & (0.08$\pm$0.04, 0.53$\pm$0.03)  & (0.06$\pm$0.03, 0.54$\pm$0.02)  & (0.07$\pm$0.03, 0.32$\pm$0.06) &  Chabrier \\
TNG100 & (0.27$\pm$0.02, 0.48$\pm$0.02) & (0.24$\pm$0.05, 0.46$\pm$0.03)  & (0.24$\pm$0.01, 0.45$\pm$0.01)  & (0.38$\pm$0.02, 0.33$\pm$0.01) & Chabrier \\
TNG300 & (0.26$\pm$0.01, 0.48$\pm$0.01) & (0.20$\pm$0.01, 0.48$\pm$0.01)  & (0.17$\pm$0.01, 0.48$\pm$0.01)  & (0.41$\pm$0.01, 0.34$\pm$0.01) & Chabrier \\
Horizon & (0.26$\pm$0.11, 0.45$\pm$0.08) & (0.25$\pm$0.04, 0.46$\pm$0.03)  & (0.26$\pm$0.02, 0.47$\pm$0.01)  & (0.47$\pm$0.02, 0.35$\pm$0.02) & Salpeter\\
\enddata
\tablecomments{This table collects the  comparison results between different simulation at different redshift. The value shows the central position offset to the local relation and the scatters of the sample of the \smass-\mbh\ correlations. A positive offset means the \mbh\ value predicted by the simulation is higher than the compared measurement at fixed \smass\ value. The last column shows the corresponding IMF that adopted to the local anchor to make a fair comparison with the observation. For the MBII sample, the simulation does not produce the sample at $z=0.5$ or $z=0.7$, but rather at $z=0.6$.
}
\end{deluxetable*}

\begin{figure}
\centering
\vspace*{-0.2cm} 
{\includegraphics[height=0.45\textwidth]{offset_summary_vz.pdf}}
\caption{\label{fig:offsets_vz} 
The apparent evolution of $\Delta{\rm log}$\mbh\ as a function of redshift using both observation and simulation data. The black line shows the evolution by fitting the offset as a function of redshift. The predictions from the numerical simulations are presented by different color regions, using the values inferred in Table~\ref{tab:sum}.}
\end{figure} 

\begin{figure*}
\centering
\begin{tabular}{c c}
%\vspace*{-0.2cm} 
{\includegraphics[height=0.35\textwidth]{offset_dis_z03.pdf}}&
%\hspace*{-0.4cm} 
{\includegraphics[height=0.35\textwidth]{offset_dis_z05.pdf}}\\
%\hspace*{-0.4cm} 
{\includegraphics[height=0.35\textwidth]{offset_dis_z07.pdf}}&
%\hspace*{-0.4cm} 
{\includegraphics[height=0.35\textwidth]{offset_dis_z15.pdf}}\\
\end{tabular}
\caption{\label{fig:offsets} 
Illustration of the histogram of the offset distribution for all the sample. The mean value and the standard derivation of the histogram are summarized in Table~\ref{tab:sum}.
For MBII simulation, the sample at redshift 0.6 is used to compare with other samples at $z=0.5$ and $z=0.7$.
}
\end{figure*} 



\begin{deluxetable*}{lcccc}
%\tablenum{1}
\tablecaption{Summary of the p-value using KS test \label{tab:pvalue}}
\tablewidth{0pt}
\tablehead{
\colhead{Simulation} &  \multicolumn{3}{c}{HSC comparison} & \colhead{HST comparison} \\
\cline{2-4} 
%  \cline{2-4}  \cline{5} \\
\colhead{} &  \colhead{$z\sim0.3$} & \colhead{$z\sim0.5$} & \colhead{$z\sim0.7$} & \colhead{$z\sim1.5$}
}
%\decimalcolnumbers
\startdata
SAM &  $<$1e-10 & $<$1e-10  & $<$1e-10  & 7.456907e-03  \\
MBII & 2.006930e-08 & \multicolumn{2}{c}{$<$1e-10 [$z=0.6$]}  & 1.420860e-04  \\
Illustris & 3.436676e-01 & 1.254032e-01  & 6.389386e-02  & 2.055473e-04  \\
TNG100 & 3.541028e-09 & 1.821151e-02  & 4.076073e-01  & 2.307911e-01  \\
TNG300 & 2.180858e-07 & 2.618639e-01  & 1.186972e-02  & 2.114181e-01  \\
Horizon & 3.928276e-04 & 2.891120e-04  & 1.289359e-02  & 1.953735e-01  \\
\enddata
\tablecomments{The p-value of the KS test between the simulation and the observation.}
\end{deluxetable*}


\begin{deluxetable*}{lcccc}
%\tablenum{1}
\tablecaption{Same as Figure~\ref{tab:sum} but without inject random noise to the simulation.\label{tab:sum_no_noise}}
\tablewidth{0pt}
\tablehead{
\colhead{Simulation} &  \multicolumn{3}{c}{HSC comparison (offset, scatter)} & \colhead{HST comparison (offset, scatter)} \\
\cline{2-4} 
%  \cline{2-4}  \cline{5} \\
\colhead{} &  \colhead{$0.2<z<0.4$} & \colhead{$0.4<z<0.6$} & \colhead{$0.6<z<0.8$} & \colhead{$1.2<z<1.7$}
}
\startdata
%Observation & (0.07, 0.58) & (0.17, 0.52)  & (0.18, 0.54)  & (0.43, 0.31) \\
SAM & (0.72, 0.20) & (0.64, 0.18) & (0.56, 0.16)  & (0.08, 0.18) \\
MBII & (-0.15, 0.21) & \multicolumn{2}{c}{(-0.15, 0.22) [$z=0.6$]}  & (0.08, 0.19)\\
Illustris & (0.03, 0.32) & (0.10, 0.36) & (0.08, 0.36)  & (0.04, 0.19) \\
TNG100 &  (0.27, 0.20) & (0.27, 0.15) & (0.26, 0.16)  & (0.36, 0.15) \\
TNG300 & (0.25, 0.21) & (0.19, 0.23) & (0.20, 0.22)  & (0.32, 0.16) \\
Horizon & (0.24, 0.21) & (0.23, 0.22) & (0.29, 0.19)  & (0.37, 0.13) \\
\enddata
%\tablecomments{Same as Figure~\ref{tab:sum} but without inject noise to the simulation.}
\end{deluxetable*}


\section{Conclusion} \label{sec:con}
%A brief introduce of the result. Scatter, central, which is best...
We compared the scaling relation of \mbh-\smass\ using both the observation data and the predictions from numerical simulations. The observation data are composed of 612 quasars at $0.2 < z < 0.8$ imaged by HSC and 32 X-ray-selected quasars at $1.2 < z < 1.7$ that imaged by HST. These observational measurements are compared with simulations including a semianalytic model (SAM) and five hydrodynamic simulation, i.e., MBII, Illustris, TNG100, TNG300 and Horizon. We designed the direct comparisons that were made in the observational plane. To this end, we first injected the random errors with same observational uncertainty to the simulation, and then adopt the same selection condition to the simulation data. Finally, we adopted the scaling relation from the local universe and performed the comparisons using the scatters of the measurements and their central offset to the local relation.

%We also estimate the scatter of the sample without inject the noise.
Our main results are presented in Figure~\ref{fig:offsets_vz} and the Table~\ref{tab:sum} in which we summary the scatter and the offset for each sample. These results show that the scatters predicted by the simulations are all consistent to the observational measurements, i.e., $\sim0.5$~dex at $z<1$ and $\sim0.3$~dex at $z>1$. We note that the scatters are {\it apparent} rather than {\it intrinsic}, which are affected by the observational error and the selection effects. To understand how much the random observational error dominates the scatter, we re-run the estimation but not injecting noise to the simulations. The obtained scatter for both $z<1$ and $z>1$ are at a level of $\sim$0.15$-$0.2 dex, which remain unchanged. This result suggests that the tightness of the scaling relations have been formed up to redshift 1.7, which is against the scenario of the central limit theorem~\citep{Peng2007, Jahnke2011, Hirschmann2010} in which the scaling relation is a consequence of a stochastic cloud in the early universe with random merger. Otherwise, the scatter of the scaling relations should increase at higher redshift. In fact, the scatter level in the simulation without adding random noise are consistent with the {\it intrinsic} scatter estimations reported in~\citet{Ding2020, 2021arXiv210902751L}, which is also not larger than the typical scatter of the local relations reported in the literature~\citep[][i.e., $\gtrsim0.35$~dex]{Kormendy13, Gul++09, Reines2015}.

The results also show that only the Illustris simulation could predict well the scaling relation central offset (related to the local) at $z<0.6$, while SAM, TNG100, TNG300 and Horizon can predict the {\it observed} evolution at $z>0.6$. \todo{As mentioned in the result section, we need provide some details on how the scaling relations in the simulations are calibrated. For example, why Illustris and TNG sample predict different center for the scaling relation. }
Interestingly, TNG100, TNG300 and Horizon simulations could well predict the evolution trend of the scaling relation from $z\sim0.7$ to $z\sim1.5$ --- the  {\it observed} offset of the $\Delta$\mbh\ (to the local relation at a given \smass) is larger at higher redshift, which are clearly shown in Figure~\ref{fig:offsets_vz} and Table~\ref{tab:sum}.

The simulations studied in this work have adopted completely different numerical technique. However, all of them can provide good agreement with the observed dispersion in the scaling relation. In fact, the tightness of the scaling relation is the result of the AGN feedback adopted in the recipe. Thus, the consistency of the scatter level between simulation and observations support the hypothesis that the causal link (i.e., AGN feedback) between BH and their hosts plays a dominate role in the scaling relation. 
\ding{The fact that in previous SAM version with a different AGN feedback recipe, the apparent scatter in the simulation is much larger up to 0.7 dex. Nicola, could you add more details here and example the different of AGN feedback in SAM compared to previous version?}
So far, a control experiment is still not performed yet in which one numerical model provide the simulation and altering the AGN feedback prescription, while fixing all other conditions.
%\todo{We need a paragraph or section to highlight that SAM simulation has been updated. The comparison of the HST in old version shows that the scatter can be up to 0.7 dex.}

\todo{Some discussion of the comparison with Habouzit et al to be added. Also, do we want to take ENGLE simulaiton?}

Extending this study to even higher redshift would be very beneficial, probing closer to the epoch of formation of massive galaxies and SMBHs. The understanding of how and when the tight scaling relation emerged are crucial to test the theory. On the observational side, the James Webb Space Telescope will provide high-quality imaging data of AGNs at redshift up to $z\sim7$ (i.e., JWST cycle 1 program GO-1967 and GO-1727). These upcoming measurements will provide severe constraints to the theories such as the initial conditions in the simulations.

%Look at the further.

\begin{acknowledgments}
We thank all the people that give us comments and useful suggestions. \todo{More to be added.}

\end{acknowledgments}


\vspace{5mm}
\facilities{HST, HSC}

%% Similar to \facility{}, there is the optional \software command to allow 
%% authors a place to specify which programs were used during the creation of 
%% the manuscript. Authors should list each code and include either a
%% citation or url to the code inside ()s when available.

%\software{astropy \citep{2013A&A...558A..33A,2018AJ....156..123A},  }

%% Appendix material should be preceded with a single \appendix command.
%% There should be a \section command for each appendix. Mark appendix
%% subsections with the same markup you use in the main body of the paper.

%% Each Appendix (indicated with \section) will be lettered A, B, C, etc.
%% The equation counter will reset when it encounters the \appendix
%% command and will number appendix equations (A1), (A2), etc. The
%% Figure and Table counter will not reset.

%\appendix
%\section{Comparison of the observed mass function}\label{app:function}
%Our comparison also provides the opportunity to compare the {\it apparent} mass distribution of the BH mass and the stellar mass. The histogram distribution of the {\it observed} \mbh\ and \smass\ are shown in Figure~\ref{fig:MB_fun} and Figure~\ref{fig:Mstar_fun}, respectively. All the simulations have good agreements to the observed mass function, in terms of the standard derivation. However, focus on the central point of the distribution, the TNG100 and TNG300 give the best prediction, indicating that their parent sample are most similar to the HSC sample. \todo{Xuheng will talk with Jingjing and add more discussion to this part.}

%\begin{figure*}
%\centering
%\begin{tabular}{c c}
%%\vspace*{-0.2cm} 
%{\includegraphics[height=0.35\textwidth]{MBH_dis_z03.pdf}}&
%%\hspace*{-0.4cm} 
%{\includegraphics[height=0.35\textwidth]{MBH_dis_z05.pdf}}\\
%%\hspace*{-0.4cm} 
%{\includegraphics[height=0.35\textwidth]{MBH_dis_z07.pdf}}&
%%\hspace*{-0.4cm} 
%{\includegraphics[height=0.35\textwidth]{MBH_dis_z15.pdf}}\\
%\end{tabular}
%\caption{\label{fig:MB_fun} 
%Illustration of the histogram of the observed \mbh\ function. For MBII simulation, the sample at redshift 0.6 is used to compare with other samples at $z=0.5$ and $z=0.7$. Note that for sample at $z\sim1.5$, the selection cut on \mbh\ is used in the simulation, thus the distribution is almost flat. \todo{update hist format as Figure 4}
%}
%\end{figure*} 
%
%\begin{figure*}
%\centering
%\begin{tabular}{c c}
%%\vspace*{-0.2cm} 
%{\includegraphics[height=0.35\textwidth]{Mstar_dis_z03.pdf}}&
%%\hspace*{-0.4cm} 
%{\includegraphics[height=0.35\textwidth]{Mstar_dis_z05.pdf}}\\
%%\hspace*{-0.4cm} 
%{\includegraphics[height=0.35\textwidth]{Mstar_dis_z07.pdf}}&
%%\hspace*{-0.4cm} 
%{\includegraphics[height=0.35\textwidth]{Mstar_dis_z15.pdf}}\\
%\end{tabular}
%\caption{\label{fig:Mstar_fun} 
%Illustration of the histogram of the observed MBH function. For MBII simulation, the sample at redshift 0.6 is used to compare with other samples at $z=0.5$ and $z=0.7$.
%}
%\end{figure*} 


\bibliography{reference}{}
\bibliographystyle{aasjournal}

%% This command is needed to show the entire author+affiliation list when
%% the collaboration and author truncation commands are used.  It has to
%% go at the end of the manuscript.
%\allauthors

%% Include this line if you are using the \added, \replaced, \deleted
%% commands to see a summary list of all changes at the end of the article.
%\listofchanges

\end{document}

% End of file `sample631.tex'.
