%% Beginning of file 'sample631.tex'
%%
%% Modified 2021 March
%%
%% This is a sample manuscript marked up using the
%% AASTeX v6.31 LaTeX 2e macros.
%%
%% AASTeX is now based on Alexey Vikhlinin's emulateapj.cls 
%% (Copyright 2000-2015).  See the classfile for details.

%% AASTeX requires revtex4-1.cls and other external packages such as
%% latexsym, graphicx, amssymb, longtable, and epsf.  Note that as of 
%% Oct 2020, APS now uses revtex4.2e for its journals but remember that 
%% AASTeX v6+ still uses v4.1. All of these external packages should 
%% already be present in the modern TeX distributions but not always.
%% For example, revtex4.1 seems to be missing in the linux version of
%% TexLive 2020. One should be able to get all packages from www.ctan.org.
%% In particular, revtex v4.1 can be found at 
%% https://www.ctan.org/pkg/revtex4-1.

%% The first piece of markup in an AASTeX v6.x document is the \documentclass
%% command. LaTeX will ignore any data that comes before this command. The 
%% documentclass can take an optional argument to modify the output style.
%% The command below calls the preprint style which will produce a tightly 
%% typeset, one-column, single-spaced document.  It is the default and thus
%% does not need to be explicitly stated.
%%
%% using aastex version 6.3
%\documentclass[linenumbers]{aastex631}
\documentclass[twocolumn]{aastex631}

\newcommand{\vdag}{(v)^\dagger}
\newcommand\aastex{AAS\TeX}
\newcommand\latex{La\TeX}
\newcommand{\todo}[1]{\textcolor{red}{[{\bf TODO}: #1]}}
\newcommand{\ding}[1]{\textcolor{red}{[{\bf Xuheng}: #1]}}
\newcommand{\jshi}[1]{\textcolor{orange}{#1}}
\newcommand{\blue}[1]{\textcolor{blue}{#1}}
\newcommand{\yo}[1]{\textcolor{purple}{[{\bf Yohan}: #1]}}
\newcommand{\red}[1]{\textcolor{purple}{#1}}

\def\smass{{$M_*$}}
\def\sersic{S\'ersic}
\def\halpha{${\rm H}\alpha$}
\def\hbeta{${\rm H}\beta$}
\def\mbh{$\mathcal M_{\rm BH}$}
\def\lcdm{$\Lambda$CDM}
\def\hst{{\it HST}}
\def\kms{km~s$^{\rm -1}$}


\newcommand{\aklant}[1]{\textcolor{blue}{#1}}
%%%%%%%%%%%%%%%%%%%%%%%%%%%%%%%%%%%%%%%%%%%%%%%%%%%%%%%%%%%%%%%%%%%%%%%%%%%%%%%%
%%
%% The following section outlines numerous optional output that
%% can be displayed in the front matter or as running meta-data.
%%
%% If you wish, you may supply running head information, although
%% this information may be modified by the editorial offices.
\shorttitle{Comparing simulations and observations of black hole - galaxy relations}
\shortauthors{Ding et al.}
%%%%%%%%%%%%%%%%%%%%%%%%%%%%%%%%%%%%%%%%%%%%%%%%%%%%%%%%%%%%%%%%%%%%%%%%%%%%%%%%
\graphicspath{{./}{figures/}}
%% This is the end of the preamble.  Indicate the beginning of the
%% manuscript itself with \begin{document}.

\begin{document}

\title{Concordance between observations and simulations in the evolution of the mass relation between supermassive black holes and their host galaxies}

\author[0000-0001-8917-2148]{Xuheng Ding}
\affiliation{Kavli Institute for the Physics and Mathematics of the Universe, The University of Tokyo, Kashiwa, Japan 277-8583 (Kavli IPMU, WPI)}

\author[0000-0002-0000-6977]{John D.Silverman}
\affiliation{Kavli Institute for the Physics and Mathematics of the Universe, The University of Tokyo, Kashiwa, Japan 277-8583 (Kavli IPMU, WPI)}

\author[0000-0002-8460-0390]{Tommaso Treu}
\affiliation{Department of Physics and Astronomy, University of California, Los Angeles, CA, 90095-1547, USA}

\author[0000-0002-1605-915X]{Junyao Li}
\affiliation{CAS Key Laboratory for Research in Galaxies and Cosmology, Department of Astronomy, University of Science and Technology of China, Hefei 230026, China}
\affiliation{School of Astronomy and Space Science, University of Science and Technology of China, Hefei 230026, China}
\affiliation{Kavli Institute for the Physics and Mathematics of the Universe, The University of Tokyo, Kashiwa, Japan 277-8583 (Kavli IPMU, WPI)}

\author[0000-0002-7080-2864]{Aklant K. Bhowmick}
\affiliation{Dept. of Physics, University of Florida, Gainesville, FL 32611, USA}

\author[0000-0002-4096-2680]{Nicola Menci}
\affiliation{INAF Osservatorio Astronomico di Roma, via Frascati 33, I-00078 Monteporzio, Italy}

\author[0000-0002-3216-1322]{Marta Volonteri}
\affiliation{Institut d'Astrophysique de Paris, Sorbonne Universit\'e, CNRS, UMR 7095, 98 bis bd Arago, 75014 Paris, France}

\author[0000-0002-2183-1087]{Laura Blecha}
\affiliation{Dept. of Physics, University of Florida, Gainesville, FL 32611, USA}

\author[0000-0002-6462-5734]{Tiziana Di Matteo}
\affiliation{McWilliams Center for Cosmology, Dept. of Physics, Carnegie Mellon University, Pittsburgh, PA 15213, USA}

\author[0000-0003-0225-6387]{Yohan Dubois}
\affiliation{Institut d'Astrophysique de Paris, Sorbonne Universit\'e, CNRS, UMR 7095, 98 bis bd Arago, 75014 Paris, France}


\begin{abstract}
We carry out a comparative analysis of the relation between the mass of supermassive black holes (BHs) and the stellar mass of their host galaxies (\mbh -- \smass) at $0.2<z<1.7$ using well-matched observations and multiple state-of-the-art simulations (e.g., Massive Black II, Horizon-AGN, Illustris, TNG and a semi-analytic model). The observed sample consist of 646 uniformly-selected SDSS quasars ($0.2 < z < 0.8$) and 32 broad-line AGNs ($1.2<z<1.7$) with imaging from Hyper Suprime-Cam (HSC) for the former and Hubble Space Telescope (HST) for the latter. To make fair comparisons, we first add realistic observational uncertainties to the simulation data and then construct a simulated sample in the same manner as the observations. Over the full redshift range, our analysis demonstrates that all simulations predict a level of intrinsic scatter of the scaling relations comparable to the observations which appear to agree with the dispersion of the local relation. 
Regarding the mean scaling relation, Horizon-AGN and TNG are in closest agreement with the observational data at both low and high redshift ($z\sim$ 0.2 and 1.5, respectively) while the other simulations show subtle differences within the uncertainties. 
For insight into the physics involved, the scatter of the scaling relation, seen in the SAM, is reduced by a factor of two and thus closer to the observations after adopting a new feedback model which considers the geometry of the AGN outflow. The consistency in the dispersion with redshift in our analysis lends support to the importance of both quasar and radio mode feedback prescriptions in the simulations. Finally, we investigate evolution as a function of stellar and black hole mass, and argue that the most direct way to probe the co-evolution of galaxies and black holes is to increase the sensitivity, e.g., with JWST, thereby pushing to lower masses and minimizing possible biases arising from selection effects.

\end{abstract}
\keywords{Galaxy evolution (594); Active galaxies (17); Active galactic nuclei (16)}

\section{Introduction} \label{sec:intro}

The close correlations between supermassive black holes and the properties of their host galaxies (e.g., stellar mass) indicate a physical coupling during their joint evolution~\citep{Mag++98, F+M00, M+H03, H+R04, Gul++09}. To understand the nature of this connection, considerable efforts have been focused on measuring such correlations using broad-line active galactic nucleus (AGN) over a range of redshifts with the intention to determine how and when the correlation emerges and evolves over cosmic time \citep[e.g.,][]{Tre++04,Peng2006a, Tre++07, Woo++08, Jahnke2009, Bennert11, Schramm2013, Park15, Mechtley2016, Ding2020, 2021arXiv210902751L}. While, an {\it observed} evolution has been found in many studies in which the growth of BHs predates that of the host, other studies predict that BHs grow commensurately with galaxies.
However, to understand the significance {\it intrinsic} evolution, it is necessary to take into account systematic uncertainties and the selection effects~\citep{Tre++07, Lauer2007, Schulze2014, Park15}. 

Various theoretical models have been proposed to explain the origin of the scaling relations. For example, AGN feedback is considered as one of the possible viable mechanisms. During this process, a fraction of the AGN energy is injected into its surrounding gas, which can then regulate the mass growth of the BH and its host. In this scenario, star formation is inhibited by the heating and unbinding of a significant amount of gas. Alternatively, the mass relations can be explained through an indirect connection in which AGN accretion and star formation are fed through a common gas supply~\citep{Cen2015, Menci2016}. Actually, even without any physical mechanisms, statistical convergence from galaxy assembly alone (i.e., dry mergers) could instill the observed correlations~\citep{Peng2007, Jahnke2011, Hirschmann2010}. However, as expected from the central limit theorem, a higher dispersion would appear in the scaling relations at high-$z$ compared to what is observed today. 

Numerical simulations provide an opportunity to further understand the connection between BHs and their host galaxies. For example, a comparison of scaling relations has been made using state-of-the-art cosmological hydrodynamical simulation of structure formation ({\tt MassiveBlackII}) and observational measurements at  $0.3<z<1$ \citep[e.g., ][]{DeG++15}, which show a positive evolution (i.e., the mass growth of the BH growth predates that of its host). Further efforts are using large-volume simulations to investigate the scaling relations and find good agreement with the local relation with redshift evolution, 
including the Magneticum Pathfinder smooth-particle hydrodynamics (SPH) Simulations~\citep{Steinborn2015}, the Evolution and Assembly of Galaxies and their Environments~(EAGLE) suite of SPH simulations~\citep{Schaye2015}, Illustris moving-mesh simulation~\citep{Sijacki2015, 2014MNRAS.444.1518V, Li2019} and the SIMBA simulation~\citep{Thomas2019}. 
In particular, the \mbh-\smass\ relation using BH populations using six large-scale cosmological simulations (i.e, Illustris, TNG100, TNG300, Horizon-AGN, EAGLE, and SIMBA) has been compared with observations in the local universe in a recent study~\citep{Habouzit2021}. However, these comparison works are limited by the observation data in terms of the sample size ($<$100) and redshift range (i.e., limited to the local universe).

For such comparisons using simulations, it is crucial to consider the systematic uncertainties and selection biases. A direct means to account for these is to apply the same effects and selection to the simulation products and make a forward comparison in the observational plane. In~\citet{Ding2020b}, a direct comparison has been performed using 32 X-ray-selected AGN at $1.2<z<1.7$ and a direct comparison with two state-of-the-art simulation efforts, including {\tt MassiveBlackII} (MBII) and a Semi-analytic Model \citep[SAM,][]{Menci2014, Menci2016}. The dispersion in the mass ratio between black hole mass and stellar mass is significantly more consistent with the MBII prediction ($\sim0.3$~dex) favoring the hypothesis of AGN feedback being responsible for a causal link between the BH and its host galaxy.

In this study, we extend our previous work by adding recent measurements of hundreds of SDSS quasars at $0.2<z<0.8$ based on wide and deep HSC imaging from the Strategic Subaru Program, and comparing the observational measurements with that from simulations. Furthermore, we extend the simulated quasar populations by including MBII, SAM, Illustris, TNG100, TNG300 and Horizon-AGN. This paper is structured as follows. In Sections~\ref{sec:observations} and~\ref{sec:simulations}, we describe our observed and simulated samples. A direct comparison is performed and the result is presented in Section~\ref{sec:result}. The concluding remarks are presented in Section~\ref{sec:con}.

\section{Observational data set}
\label{sec:observations}
The observed sample consists of 646 uniformly-selected SDSS quasars at $0.2<z<0.8$, imaged by Subaru/HSC \citep{Li2021a}, and 32 quasars at $1.2<z<1.7$ as imaged by~\hst~\citep[][hereafter D20]{Ding2020}. The latter are selected from three deep-survey fields, namely COSMOS~\citep{Civano2016}, (E)-CDFS-S~\citep{Lehmer2005, Xue2011}, and SXDS~\citep{Ueda2008}. Further details of these two samples and their measurements are given below. 

\subsection{SDSS/HSC sample}\label{sec:hsc}
A sample of $\sim$5000 type-1 SDSS quasars from the DR14 catalog~\citep{Paris2018} at $0.2<z<1$ has been imaged by the high-resolution Subaru Strategic Program (SSP) wide area survey~\citep{Aihara2019} using Hyper Suprime-Cam~\citep{Miyazaki2018}. With accurate PSF models in five optical bands {\it grizy}, two-dimensional quasar-host decompositions have been performed \citep[][hereafter L21a]{Li2021a} to obtain the flux and color of each quasar's host galaxy. The stare-of-the-art image modeling software {\tt lenstronomy} \citep{Birrer2015, Birrer2018, Birrer2021} is adopted to perform the modeling task. This approach is first developed by~\citet{Ding2020} and used to decompose the near-infrared emission of the HST sample (see next section). Having measured the host light in each band, the stellar mass of host galaxy is derived using spectral energy distribution (SED) fitting with CIGALE~\citep{Boquien2019}. Simulation tests are also performed to verify the fidelity of the \smass\ measurements. The statistical measurement error on \smass\ is at the $\sim$0.2~dex level. The values of \mbh\ are determined by~\citet{Rakshit2020} which are estimated based on the \hbeta-based measurements using the virial method~\citep{Peterson2004, Vestergaard2006}. The typical error of \mbh\ are estimated to be 0.4 dex. We refer the reader to~\citet{Li2021a} in the Section~4.2 for more details.

To avoid any potential biases related to the selection of the quasars, \citet{2021arXiv210902751L} isolated 877 sources which are uniformly selected based on their PSF-magnitudes, color cuts using single-epoch SDSS photometry and the value of the measured \smass. Specifically, we use the {\it ugri} color-selected sample (228 sources) from SDSS I/II~\citep{Richards2002}, and the CORE sample from SDSS BOSS (408 sources) and eBOSS (241 sources) surveys~\citep{Ross2013, Myers2015} (hereafter the uniform sample). These samples are initially selected based on PSF-magnitude cuts of $15 < i < 19.1$ (for {\it ugri}), and $i > 17.8$ and $g, r < 22.0$ (for CORE). Furthermore, a limit on \smass\ is set to assure the detection of the host, especially since the rate and accuracy of detection is higher when \smass\ is increasing, resulting in a final sample of 646 quasars. These selections will be adopted in an equivalent manner to the simulated samples to mitigate selection effects thus allowing the fair comparison.

\subsection{HST sample}

A sample of 32 HST-observed AGN systems across the redshift range $1.2<z<1.7$ are selected from three deep-survey fields (COSMOS, (E)-CDFS-S, and SXDS). HSC/WFC3 IR camera is used to obtain the high-resolution imaging data (HST program GO-15115, PI: John Silverman) with six position dither pattern and a total exposure time $\sim$2348~s. The filters F125W ($1.2<z<1.44$) and F140W ($1.44<z<1.7$) were employed, according to the redshift of each target to bracket the 4000~\AA~break.  The AGN images are analyzed and decomposed to infer the host galaxies fluxes using the approach developed by D20 based on {\tt lenstronomy}. The HST ACS/F814W imaging data for 21/32 of the AGNs is also used to infer the host color. The results show that stellar templates of 1 and 0.625~Gyr can match the sample color at $z<1.44$ and $z>1.44$, respectively (see Figure 5 in D20). These best-fit models are used to estimate the stellar masses of the host galaxies. \mbh\ is determined by \citet{Schulze2018} using near-infrared spectroscopic observations of the broad \halpha\ emission line with the recipe provided by~\citet{Vestergaard2006}, in a consistent manner to that adopted for HSC sample. We refer the reader to D20 for a more detailed description of the analysis. 

The measurements of the \mbh-\smass\ relations for both the HST and HSC samples are obtained with a consistent approach. Thus, we expect the measurement errors of these two samples to be at a comparable level (i.e., $\Delta$\mbh$=0.4~$dex, $\Delta$\smass$=0.2~$dex). 
Indeed, the two samples are consistent with a lack of evolution in the mass ratio~\citep[see Figure 6 of][]{2021arXiv210902751L}, even though the sample selection is slightly different.


\begin{deluxetable*}{lccccc}
\tablecaption{Key characteristics of hydrodynamic simulations used in this study.\label{tab:sim_sum}}
\tablewidth{0pt}
\tablehead{
\colhead{Simulation} & MBII & Illustris &  TNG100 & TNG300 & Horizon-AGN
}
\startdata
box sizes $(\mathrm{cMpc})^3$  & $142.7^3$ & $(106.5)^3$ & $(111)^3$ & $(302)^3$ & $(142)^3$ \\
particles  & $2\times1792^3$ & $2\times1820^3$ & $2\times1820^3$ & $2\times2500^3$ & $\sim2\times1024^3$ \\
\cline{1-6}
{\bf mass resolution} \\
 dark matter &$1.57\times10^7$ & $6.26\times10^6$ & $7.5\times10^6$ & $5.9\times10^7$ & $8\times10^7$ \\
 baryonic matter &$3.14\times10^6$ & $1.26\times10^6$ & $1.4\times10^6$ & $1.1\times10^7$ & $2\times10^6$ \\
\cline{1-6}
{\bf AGN feedback} & \multicolumn{2}{c}{(feedback efficiency $\times$ radiative efficiency)} \\
High acc. mode & 0.05 $\times$ 0.1 & 0.05 $\times$ 0.2 & 0.1 $\times$ 0.2 & 0.1 $\times$ 0.2 & 0.15 $\times$ 0.1 \\
Low acc. mode  & -- & 0.35 $\times$ 0.2 & $\leq0.2 \times$ 0.2 & $\leq0.2 \times$ 0.2 & 1 $\times$ 0.1 \\
Transitions btw. modes & -- & 0.05 & \multicolumn{2}{c}{min[$2\times10^{-3}(\frac{M_{\rm BH}}{10^8M_{\odot}})^2$, 10\%]} & 0.01 \\
\cline{1-6}
AGN fueling mechanism   & ${4\pi G^2 M_{BH}^2 \rho}/{(c_s^2+v_{BH}^2)^{3/2}}$ & $\alpha \dot{M}_{Bondi}$ & $\dot{M}_{Bondi}$&$\dot{M}_{Bondi}$ & $\alpha\dot{M}_{Bondi}$ \\
maximum accretion rate & 2 $\times$Edd. acc. rate & Edd. acc. rate & Edd. acc. rate & Edd. acc. rate & Edd. acc. rate\\
\enddata
\tablecomments{In the penultimate row, $\alpha$ is the boost factor. For Illustris, $\alpha = 100$; for Horizon-AGN, $\alpha =$ max$[(\rho/\rho_0)^2, 1]$. $\dot{M}_{Bondi}= 4\pi G^2M_{BH}^2 \rho/c_s^3$.
}
\end{deluxetable*}


\section{Simulations and comparison strategy}
\label{sec:simulations}
We introduce the simulation samples that are adopted in this section. All the simulation are based on the larger-scale cosmological simulations, except the Semianalytic Model (SAM) simulation (see Section~\ref{subsec:SAM}). In Table~\ref{tab:sim_sum}, we summarized the key elements for each hydrodynamic simulation being considered.

\subsection{{\tt MassiveBlackII} (MBII)}\label{subsec:MBII}
MBII is a high-resolution cosmological hydrodynamic simulation that has a box size of $(142.7~\mathrm{cMpc})^3$
and $2\times1792^3$ particles. The simulation is based on smooth particle hydrodynamic (SPH) code \texttt{P-GADGET}, a hybrid version of the parallel code {\tt GADGET}~\citep{2005MNRAS.364.1105S}. The base cosmology parameters are based on the WMAP7 results~\citep{2011ApJS..192...18K}. For dark matter and gas, the mass resolutions are $1.57\times 10^7~M_{\odot}$ and $3.14\times 10^6~M_{\odot}$, respectively. The simulation includes a full modeling of gravity plus gas hydrodynamics, with a wide range of subgrid recipes to model the star formation~\citep{2003MNRAS.339..289S}, BH growth, and the feedback process~\citep{2005Natur.433..604D}. Halos were identified using a friends-of-friends (FOF) group finder~\citep{1985ApJ...292..371D}. Galaxies are identified with the stellar matter components of subhalos; these subhalos are identified using {\tt SUBFIND} within the halos~\citep{2005MNRAS.364.1105S}.

To model supermassive black holes, a BH seed with mass $5\times 10^{5}~M_{\odot}/h$ are inserted into halos of mass $\gtrsim 5\times 10^{10}~M_{\odot}/h$. Once seeded, BH growth via gas accretion is assigned at a rate of $\dot{M}_{BH}={4\pi G^2 M_{BH}^2 \rho}/{(c_s^2+v_{BH}^2)^{3/2}}$ where $\rho$ and $c_s$ are the density and sound speed of the interstellar medium (ISM) gas at cold phase; $v_{BH}$ is the relative velocity between the BH and its surrounding gas. Note that unlike several previous works, the accretion rate in MBII adopt the prescription in~\citet{Pelupessy2007} which does not include any artificial boost factor. The accreted gas is released as radiation at a radiative efficiency of 10\%. A fraction of 5\% of the radiated energy thermally couples to the surrounding gas as black hole (or AGN) feedback~\citep{2005Natur.433..604D}. A mildly super-Eddington (two times Eddington rate) is allowed. Due to resolution limitations, BH dynamics cannot be self-consistently modeled in the simulations. Two BHs are considered to be merged when their separation distance is below the simulation spatial resolution (i.e., the SPH smoothing length) and their relative speeds are lower than the local sound speed of the medium.

As a common practice, the stellar mass is obtained by using a 3D spherical aperture of 30~kpc to represent the observed stellar mass. We adopt this definition of stellar masses for all simulations described in the following sections. Using this definition, \citet{Pillepich2018} has shown that the corresponding stellar mass function is consistent with the observational measure. Even more, the stellar mass using this 3D aperture can achieve good agreement to those measured within the Petrosian radii in observational studies~\citep{Schaye2015}. For further details of MBII simulation, we refer the reader to~\citet{Khandai2015}.

\subsection{Illustris}
The Illustris Project is another large scale hydrodynamics simulation, introduced in~\citet{2014MNRAS.444.1518V, 2014Natur.509..177V}. The simulation consist of a volume of (106.5 cMpc)$^3$~(slightly smaller than MassiveBlack II), 
and was run with the moving Voronoi mesh code {\tt Arepo}~\citep{2010MNRAS.401..791S} with a base cosmology adopted from WMAP9 results~\citep{2013ApJS..208...19H}. Besides gravity and gas hydrodynamics, the simulation calculates the astrophysical processes ~\citep{2013MNRAS.436.3031V, 2014MNRAS.438.1985T} that includes gas cooling and star formation~(with a density threshold of 0.13 cm$^{-3}$, \citealt{2003MNRAS.339..289S}), stellar evolution and chemical enrichment, kinetic stellar feedback by SNe activity, BH growth~(accretion and merging), and AGN feedback.

BHs are seeded with an initial mass of $1 \times 10^5~M_{\odot}/h$ when a halo exceeds a mass of $5 \times 10^{10}~M_{\odot}/h$. BHs then grow via accretion described by the Eddington limited Bondi-Hoyle-Lyttleton formalism ($\alpha4\pi G^2M_{BH}^2 \rho/c_s^3$), as well as mergers with other BHs. The boost factor $\alpha=100$ is introduced to account for the unresolved multiphase ISM~\citep{Springel2005, 2009MNRAS.398...53B}, which is otherwise expected to underestimate the density around the BHs. Lastly, accreting black holes radiate with a bolometric luminosity given by $\epsilon_r \dot{M}_{BH}c^2$, where $\dot{M}_{BH}$ is the mass accretion rate and $\epsilon_r=0.2$ is the radiative efficiency.

The AGN feedback consists of three components, namely quasar-mode, radio-mode and radiative feedback. In the quasar-mode which holds for BHs with Eddington ratio $>0.01$, the AGNs deposit $5\%$~(quasar-mode feedback efficiency) of their released energy into the surrounding gas as thermal energy. For Eddington ratios $<0.01$, the AGN feedback is in radio-mode where the thermal energy is released as hot bubbles with a radius of $\sim$ 100~kpc at~(irregular) intervals between which the BH mass grows by a fixed fraction. The energy of the bubbles is given by $\epsilon_m \epsilon_r \delta M_{bh} c^2$ where $\delta M_{bh}$ is the change in BH mass within the last time interval, and $\epsilon_m=0.35$ is the radio-mode feedback efficiency. Lastly, the radiative feedback mode is implemented by modifying the heating and cooling rates of the gas in the presence of radiation from all surrounding AGN. As in MBII, a 3D 30~kpc spherical aperture is used to obtained the galaxy stellar mass, assuming a \cite{2003PASP..115..763C} IMF.


\subsection{IllustrisTNG}
{\it The Next Generation Illustris Simulations} (IllustrisTNG)~\citep{2018MNRAS.475..676S, Pillepich2018} are a suite of magnetohydrodynamical simulations of galaxy formation in large cosmological volumes. It builds upon the scientific achievements of the Illustris simulation with improvements upon Illustris by 1) extending the mass range of the simulated galaxies and haloes, 2) adopting an improved numerical and astrophysical modeling, and 3) addressing the identified shortcomings of the previous generation simulations.

The TNG100 and TNG300 have a volume of (100~cMpc)$^3$ and (300~cMpc)$^3$, respectively. The adopted cosmological parameters are updates by the Planck result~\citep{2016A&A...594A..13P}.
The gas cooling and star formation prescriptions are broadly similar to the Illustris model. However significant updates have been made to the stellar feedback model~(more details in \citealt{2018MNRAS.473.4077P}).
BH seeds with initial mass of $8 \times 10^5 M_{\odot}/h$ are placed in Dark matter halos with a mass exceeding $5 \times 10^{10} M_{\odot}/h$. Notably, the seed mass is one order of magnitude higher than in the Illustris simulation. The BH accretion also follows the Bondi-Hoyle-Lyttleton formalism, but without any boost factor~(unlike Illustris). Accreting black holes release energy with a radiative efficiency of 0.2~(same as Illustris). The inclusion of the magnetic fields can affect the relationship between the BHs and their host galaxies properties; the \mbh-\smass\ mean relation is higher with magnetic fields~\citep{2018MNRAS.473.4077P}. 

The AGN feedback occurs in thermal, radio, and radiative modes. For high accretion rates, the feedback implementation is the same as Illustris i.e., thermal energy is injected in the surroundings of the accreting BHs. However, at low accretion rates, the feedback implementation is substantially different from Illustris. Instead of releasing hot bubbles, this feedback mode in TNG is purely kinetic. In particular, there is a directional injection of momentum along a randomly chosen direction~\citep{2017MNRAS.465.3291W, 2018MNRAS.479.4056W} at irregular intervals. The transition between the two feedback modes is also different from Illustris, and is set by the minimum value of 0.1 and $2 \times 10 ^{-3} \times (M_{BH} / 10^8~M_{\odot})$. Additionally, the radiative feedback implemented in Illustris~(summarized in the previous section) is also present in TNG. Lastly, the galaxy stellar mass in Illustris-TNG is obtained in a similar manner to that of Illustris.

\subsection{Horizon-AGN}\label{subsec:Horizon}
The simulation Horizon-AGN~\citep{2014MNRAS.444.1453D, 2016MNRAS.463.3948D} has a volume of 142 cMpc$^3$ and was generated using the adaptive mesh refinement code {\tt Ramses}~\citep{2002A&A...385..337T} with a $\Lambda$CDM model based on WMAP7~\citep{2011ApJS..192...18K} cosmological results. The dark matter particle mass is $8\times 10^7 M_{\odot}$. The stellar particle mass is $2\times 10^6 M_{\odot}$ and the MBH seed mass is $10^5 M_{\odot}$. Adaptive mesh refinement is permitted down to $\Delta x=1$~kpc, and, if the total mass in a cell becomes greater than 8 times the initial mass resolution, it is performed in a quasi-Lagrangian manner. Collisionless particles (dark matter and star particles) are evolved using a particle-mesh solver with a cloud-in-cell interpolation.

The simulation includes gas cooling down to $10^4\, \rm K$ \citep{sutherland&dopita93}, and stochastic star formation with a constant star formation efficiency $\epsilon_*=0.02$, which occurs in regions where the gas number density exceeds the star formation threshold $n_0 = 0.1\,\rm H\, cm^{-3}$. A Salpeter IMF is assumed. Stellar feedback is modeled as mechanical energy injection from Type Ia SNe, Type II SNe and stellar winds, with the metal enrichment from these sources.

Differing from simulations presented above, Horizon-AGN does not use a fixed threshold in the dark matter halo mass to seed BHs.  BHs are seeded with a mass of $10^5 M_\odot$ in cells, with gas density above $n_0$ and stellar velocity dispersion larger than $100 \,\rm km\,s^{-1}$. An exclusion radius is imposed so that no BH seed is formed at less than 50 ckpc from an existing BH. After $z = 1.5$, new BHs are prevented from forming. At these subsequent times, all the progenitors of the \smass$>10^{10} M_{\odot}$ galaxies at $z = 0$ should be formed and seeded with BHs~\citep{2016MNRAS.460.2979V}.  BH accretion is computed using the Bondi-Hoyle-Lyttleton formalism with a boost factor $\alpha = (\rho/\rho_0)^2$ when the density $\rho$ is higher than the resolution-dependent threshold $\rho_0$. Otherwise, the boost factor is fixed as unity~\citep{2009MNRAS.398...53B}.

Horizon-AGN includes two modes of AGN feedback. In the quasar mode ($f_{\rm Edd}>0.01$), thermal energy is isotropically released within a sphere of radius a few resolution elements. The energy deposition rate is $\dot{E}_{\rm AGN} = 0.015 \dot{M}_{\rm BH} c^2$. In the radio mode, energy is injected into a bipolar  outflow  with  a  velocity  of  $10^4\,\rm km\,s^{-1}$, to  mimic the  formation  of  a  jet.  The  energy  rate  in  this  mode is $\dot{E}_{\rm AGN} = 0.1 \dot{M}_{\rm BH} c^2$.  The  technical  details  of  BH  formation,  growth  and AGN  feedback  modeling  of  Horizon-AGN  can be found in~\citet{2012MNRAS.420.2662D}.

We identify galaxies applying the AdaptaHOP structure finder \citep{Aubert+04,Tweed+09} to the star particle distribution.  Galaxies are identified using a local threshold of 178 times the average matter density, with the local density of individual particles calculated using the 20 nearest neighbours. Only galaxies with more than 50 particles are considered. The galaxy mass corresponds to the total stellar mass of a galaxy identified with this approach. The stellar mass function is shown in \cite{2017MNRAS.467.4739K} to be in good agreement with observations.


\begin{figure}
\centering
\includegraphics[height=0.37\textwidth]{OpeningAngle.pdf}
\caption{\label{fig:SAM} 
Total gas content of galaxies as a function of AGN bolometric luminosity and jet opening angle in a new AGN feedback model incorporated into the SAM simulation.
}
\end{figure} 

\begin{figure*}
\centering
\begin{tabular}{c c}
{\includegraphics[trim = 0mm 0mm 65mm 10mm, clip, height=0.4\textwidth]{HSC_selection_MBII.png}}
{\includegraphics[trim = 0mm 0mm 20mm 10mm, clip,height=0.4\textwidth]{HST_selection_MBII.png}}
\end{tabular}
\caption{\label{fig:selection}Demonstration of AGN selection using MBII. {\it left}: Distribution of \mbh\ and Eddington ratio for the full (colored squares) MBII sample and individual objects meeting the observed selection criteria (blue circles). A matched HSC sample is shown by the orange data points. The light green background cloud shows the {\it intrinsic} simulated number density in this parameter space.
{\it right}: Similar to the panel on the left, this figure presents the impact of selection on the HST sample. For visual comparison between the HSC and HST selection, we show the region of the HST selection window in the left panel as dashed lines. 
}
\end{figure*}

\subsection{Semi-analytic Model (SAM)}\label{subsec:SAM}
We highlight the main points of the simulation with respect to our study; for more detail, a full description of the SAM can be found in~\citet{Menci2016} which is based on an earlier semi-analytic model introduced in~\citet{Menci2014}. The specific version adopted here differs from the one presented in the above papers since it implements a new, detailed description of AGN feedback, as discussed in detail below.

For dark matter halos that merge with a larger halo, the impact of dynamical friction is assessed to define whether the halo will survive as a satellite or sink to the center of the dominant galaxy which increases its mass. The binary interactions (fly-bys and mergers), among satellite sub-halos, are  also described by the model. In each halo, we compute the fraction of gas which cools because of the atomic processes and settles into a disk~\citep{Mo1998}. The stars are converted from the gas through three channels: (1) quiescent star formation with long time scales: $\sim1$~Gyr; (2) starbursts following galaxy interactions with timescales $\lesssim 100$~Myr, according to BH feeding; (3) the loss of angular momentum triggered by the internal disk instabilities causing the gas inflows to the center, resulting in stimulated star formation (as well as BH accretion). The stellar feedback is also considered by calculating the energy released by the supernovae associated with the total star formation which returns a fraction of the disk gas into a hot phase. A Salpeter IMF is adopted in the SAM simulation. A $\Lambda$CDM power spectrum of perturbations with a total matter density parameter $\Omega_0=0.3$, a baryon density parameter 
$\Omega_b=0.04$, a dark energy density parameter $\Omega_\Lambda=0.7$, and a Hubble constant $h$=0.7 is adopted.

{We assume BH seed $M_{seed}=100\,M_{\odot}$~\citep{Madau2001}  to be initially present in all galaxy progenitors at the initial redshift $z=15$. This constitutes an approximate way of rendering the effect of the collapse of PopIII stars. However, the detailed value of $M_{seed}$ has a negligible impact on the final BH masses as long as they remain in the range $M_{seed}=50-500\,M_{\odot}$.
}

The BH accretion is based on both interaction-driven and disk instability feeding modes.
The SAM adopted here implements a new and improved  model for the AGN feedback with respect to the previous versions~\citep{Menci2008}. In both versions, the basic assumption is that fast winds with velocity up to $10^{-1}c$ observed in the central regions of AGNs~\citep{Chartas2002, Pounds2003}  result in  supersonic outflows that compress the gas into a blast wave terminated by a leading shock front. This  moves outward with a lower but still supersonic speed, and sweeps out the surrounding medium. However, while in the earlier version of the SAM~\citep{Menci2016} the blast wave is assumed to expand into an isotropically distributed medium, in the new description of AGN feedback~\citep{Menci2019} the full two-dimensional structure of the gas disc and of the expanding blast wave is followed in detail. The main physical difference is that in the new model the large density of gas along the plane of the disc causes the blast wave expansion to stall in such a direction, while it expands with large velocities in the vertical direction. The resulting strong dependence of the total (integrated over directions) outflow rate on the AGN luminosity $L_{AGN}$ and on the gas content of the galaxy $M_{gas}$ is shown in Figure~\ref{fig:SAM}. Such a new AGN feedback model has been tested in detail against a state-of-the-art compilation of observed outflows in 19 galaxies with different measured gas and dynamical masses~\citep{Fiore2017}, allowing for a detailed, one-by-one comparison with the model predictions. This well tested AGN feedback model allowed us to derive, for each simulated galaxy in the SAM,  the outflow expansion and the mass outflow rates in different directions with respect to the plane of the disc.



\subsection{Application of observational measurement error and selection effects}
To make direct comparisons with observations, we add measurement errors and apply the equivalent selection to the simulated samples. We first inject random noise to the simulated catalog to mimic the scatter caused by measurement error. As mentioned above, \smass\ and \mbh\ for HSC and HST samples are measured with a similar approach; thus their uncertainty levels are expected to be equivalent. 
We assume the following measurement uncertainties that are added as random noise: $\Delta$\mbh$ = 0.4~$dex, $\Delta$\smass$ = 0.17~$dex, and $\Delta L_{\rm bol} = 0.03~$dex. 


We then apply restrictions on the noise-injected simulation to mimic selection effects as present in the observational data. Since the HSC and HST samples have their own selection function, we apply different selection criteria to the simulation as follows:

 HSC sample: (1) The observed sample consists of type-1 AGN, and thus the simulated sample should match the relationship between \mbh-$L_{\rm bol}$ as seen in the HSC sample. We use MBII to demonstrate the importance of matching the sample selection (Figure~\ref{fig:selection}--{\it top}). (2) The $i$-band magnitude of the AGN are bright (see Section~\ref{sec:hsc}). The specific selection is made as follows: for systems at $z<0.5$ and $z>0.5$, the AGN $i$-band magnitude is required to be brighter than 20.5 mag and 22.0 mag, respectively. Since the simulations do not provide the observed AGN magnitude, we adopt a simulated rest-frame magnitude or L$_{\rm 5100}$ and assume the quasar continuum as a single power-law with an index of $\alpha_\nu=-0.44$~\citep{2001AJ....122..549V} to calculate the observed $i$-band magnitude.
 3)~Following the HSC selection, we require the \smass\ value to be above a certain level (according to their redshift) to assure an accurate measurement. Finally, the HSC sample is split into three redshift bins for making comparison which are $0.2<z<0.4$, $0.4<z<0.6$, and $0.6<z<0.8$.
 
HST sample: Simulated AGN systems are selected only when they match the  \mbh-$L_{\rm bol}$ targeting window which is the same as the observational selection (see Figure~\ref{fig:selection}--{\it right} using MBII as an example). Note that the selection of the HST sample has a hard cut on the \mbh\ values (i.e., between [7.7, 8.6] $M_{\odot}$). The HST sample covers the higher redshift range $1.2<z<1.7$, which is considered as a single redshift bin to make the comparison with the simulations at $z=1.5$.


\section{Results} \label{sec:result}
In Figure~\ref{fig:comparsion}, we present the mass scaling relation \mbh--\smass\ for both the observations and simulations for direct comparison. The local scaling relation adopted by D20 (e.g., \mbh$=0.98$\smass$-2.56$, Chabrier IMF\footnote{Since different simulations adopt either a Chabrier or a Salpeter IMF, we use the local relation and \smass\ of the observational data that are based on the same IMF thus a comparison between the observations and simulations are self-consistent.}) is used as the fiducial relation to assess relative offsets and differences in dispersion. A different simulation is presented in each row and the redshift intervals increase from left to right. We note that TNG100 and TNG300 yield very similar comparison results, thus only TNG100 is presented in Figure~\ref{fig:comparsion}.

\begin{figure*}
\centering
\includegraphics[trim = 50mm 70mm 40mm 90mm, clip,height=1.1\textwidth]{MM_sum.png}
\caption{\label{fig:comparsion} 
Black hole mass versus stellar mass for both the observational (small orange circles) and simulated (small colored circles) samples. Each row pertains to a particular simulation as labelled. The panels, from left to right, show different redshift bins. The black line in each panel indicates the local relation adopted by~\citet{Ding2020}. The background cloud (in green and yellow) shows the intrinsic simulation number density before injecting random noise and applying selection effects. Only TNG100 is presented here since TNG300 presents very similar results.
}
\end{figure*} 

For each sample, the central offset and scatter of the scaling relation are estimated by calculating the mean and the standard deviation of the \mbh\ residuals for each system\footnote{The value of the slope for the local sample is close to 1, and thus if taking the \smass\ to calculate the residual for each system, the offset value remains the same.} (i.e., the offset to the local relation along the y-axis in Figure~\ref{fig:comparsion}).  To aid in visualization of the differences among the various simulations compared to the observed sample, we show the distribution of offsets (in terms of the $\Delta{\rm log}$\mbh) as histograms in Figure~\ref{fig:offsets}. Each panel presents a different redshift range. In addition, the values for the central offset and scatter in each case are given in Table~\ref{tab:sum} and shown as a function of redshift in Figure~\ref{fig:offsets_vz}.

\subsection{Dispersion}

Our results show that almost all simulations can produce scatter which is consistent with the observations across all redshifts examined (Figures~\ref{fig:comparsion}~and~\ref{fig:offsets}) --- for the simulated samples at $z<1$, this level of scatter is $\sim0.5$~dex, while at $z>1$, it is $\sim0.3$~dex. Note that the HST sample $z>1$ has a narrow selection window based on \mbh\ (see Figure~\ref{fig:selection}~bottom), causing the observed scatter to be smaller than that of the HSC sample at $z<1$. At all redshifts, we recognize that the observed scatter is dominated by measurement uncertainties in the data. 

An understanding of how much of the scatter derives from random noise can help us to determine the {\it intrinsic} scatter in the scaling relation. To this end, we measure the scatter of the simulation sample without injecting random noise but adopting the same selection window for both $z<1$ and $z>1$ samples to infer the central offset and scatter. We find that the intrinsic scatter is at a level of $\sim0.15-0.2$ dex for both $z<1$ and $z>1$ (see Table~\ref{tab:sum_no_noise}). These levels are consistent with the {\it intrinsic} scatter as estimated using observation data alone~\citep{Ding2020, 2021arXiv210902751L}. Furthermore, the intrinsic scatter appears to be independent of redshift since the observations and simulations all follow the observed trend with redshift expected to be due to selection effects (Figure~\ref{fig:offsets_vz}). This suggests that the tight scaling relation may not be the result of a pure stochastic process, i.e., random mergers. However, the scatter is affected by sample selection, and thus these levels can only be taken as an approximation of the true intrinsic scatter.

\begin{figure}
\centering
\begin{tabular}{c c c c}
\hspace*{-0.4cm} 
{\includegraphics[height=0.4\textwidth]{offset_dis_z03.pdf}}&
\hspace*{-0.4cm} 
{\includegraphics[height=0.4\textwidth]{offset_dis_z05.pdf}}\\
\hspace*{-0.4cm} 
{\includegraphics[height=0.4\textwidth]{offset_dis_z07.pdf}}&
\hspace*{-0.4cm} 
{\includegraphics[height=0.4\textwidth]{offset_dis_z15.pdf}}\\
\end{tabular}
\caption{\label{fig:offsets} 
Histograms of the offset distributions for all simulation samples and observations. The mean value and the standard derivation of the histogram are summarized in Table~\ref{tab:sum}. The vertical dashed lines show the corresponding mean value for each distribution. The mean values for observed sample (i.e., yellow lines) are also show in each simulation plots.
For the MBII simulation, the sample at redshift 0.6 is used to compare with other samples at $z=0.5$ and $z=0.7$.
}
\end{figure} 


\begin{figure}
\centering
\includegraphics[height=0.4\textwidth]{offset_summary_vz.pdf}
\caption{\label{fig:offsets_vz} 
The {\it observed} evolution of $\Delta{\rm log}$\mbh\ as a function of redshift using both observation and simulation data. The black line shows the evolution by fitting the offset as a function of redshift. The predictions from the numerical simulations, given in Table~\ref{tab:sum}, are presented by different colored symbols. The grey horizontal band illustrates the level of dispersion for the local sample.
}
\end{figure} 

\subsection{Global offsets}
We examine the offsets to understand whether the simulations deviate or not from the {\it observed} scaling relation with particular attention to changes with redshift. Considering the values given in Table~\ref{tab:sum} and shown in Figure~\ref{fig:offsets_vz}, over the lower redshift range $z<0.6$, Illustris and Horizon-AGN predict {\it observed} \mbh\ offsets consistent with the observation data (at a level of $\lesssim0.1$~dex). At higher redshift $0.6<z<1.5$, the simulations SAM, TNG100, TNG300 and Horizon-AGN follow the {\it observed} evolution. These results are consistent with the Kolmogorov-Smirnov (KS) test performed using the offset distributions between each simulated sample and the observed sample --- the p-values are given in Table~\ref{tab:pvalue} showing that Horizon-AGN and Illustris have a good statistical match to the scaling relation at $z<0.6$ (i.e., p-value $> 0.1$), while the TNG100, TNG300 and Horizon-AGN simulation do well at $z>0.6$. Overall, we find that the mass ratios between SMBHs and their host galaxies are generally consistent between observations and simulations with some subtle differences which are not at the level of concern for this present study.


\subsection{Trends with stellar mass}
In Figure~\ref{fig:deltaMM}, we investigate how the offset values are correlated with stellar mass. Here, we focus on the sample at $z\sim0.7$. The other redshift bins at $z<1$, where there is a large observation sample from HSC, show similar trends. We include the intrinsic values from the simulations in the figures to address how the observational effects (i.e., random noise and selection) change the observed scaling relations and offsets. First, considering the observed quasar sample (same in each panel), there is a trend for which black holes have masses further offset from their stellar mass with decreasing stellar mass. This trend is not seen in any of the simulations after noise and selection effects have been applied. Given the level of uncertainties in the mean offsets of the observed sample, we do not try to interpret this trend any further in this study.

Interestingly, we notice that MBII and Illustris have black holes intrinsically under-massive relative to their galaxies at the lower masses that reach the local relation at higher masses. In contrast, TNG and Horizon-AGN have black holes slightly elevated from the local relation at most masses. These differences between simulations present two different scenarios, either one where the black holes come later or co-evolution with the two growing in tandem. Considering the former scenario, Illustris show the strongest trend with stellar mass. In fact, after noise and selection is applied, the simulated sample exhibits very small offsets which agree remarkably well with the observed data. This result underscores the importance of taking into account errors and selection --- without accounting for those, one could erroneously interpret an apparent trend as evolution in the opposite sense as the true one.  The most direct way to circumvent these issues is to probe lower masses (\smass\  $<10^{10}M_{\odot}$) using a more sensitive instrument, such as JWST~\citep{Habouzit2022} across this redshift range \citep[see also][]{2011MNRAS.417.2085V}.

\begin{figure*}
\centering
\begin{tabular}{c c}
\hspace*{-0.5cm} 
{\includegraphics[trim = 0mm 0mm 0mm 0mm, clip,
height=0.4\textwidth]{DeltaMM_MBII_zs_06.png}}&
\hspace*{-0.3cm} 
{\includegraphics[trim = 36mm 0mm 0mm 0mm, clip,
height=0.4\textwidth]{DeltaMM_Illustris_zs_07.png}}\\
\hspace*{-0.5cm} 
{\includegraphics[trim = 0mm 0mm 0mm 0mm, clip,
height=0.4\textwidth]{DeltaMM_TNG100_zs_07.png}}&
\hspace*{-0.3cm} 
{\includegraphics[trim = 36mm 0mm 0mm 0mm, clip,
height=0.4\textwidth]{DeltaMM_Horizon_zs_07.png}}\\
\end{tabular}
\caption{\label{fig:deltaMM} Comparison of the offset of the \mbh\ (to the local relation) as a function of stellar mass from observation data and the simulations at $z\sim0.7$. In each stellar mass bin, we give the mean and standard derivation of the offset values. The histograms on the right indicate the offset distribution with lines marking the mean offsets for observation and simulation. The green color distributions show the intrinsic simulated sample without random noise and selection applied.
}
\end{figure*} 


\begin{deluxetable*}{lccccc}
\tablecaption{Summary of the central offsets and scatters\label{tab:sum}}
\tablewidth{0pt}
\tablehead{
\colhead{Simulation} &  \multicolumn{3}{c}{HSC comparison (offset, scatter)} & \colhead{HST comparison (offset, scatter)} \\
\cline{2-4} 
\colhead{} &  \colhead{$0.2<z<0.4$} & \colhead{$0.4<z<0.6$} & \colhead{$0.6<z<0.8$} & \colhead{$1.2<z<1.7$} & IMF
}
\startdata
Observation & (0.12, 0.51) & (0.20, 0.50)  & (0.21, 0.56)  & (0.43, 0.31) & \\
SAM & (0.73, 0.49) & (0.65, 0.46)  & (0.51, 0.45)  & (0.51, 0.36) & Salpeter \\
MBII & (-0.15, 0.48) & \multicolumn{2}{c}{(-0.16, 0.48) [$z=0.6$]}  & (0.14, 0.31) & Salpeter\\
Illustris & (0.01, 0.52) & (0.08, 0.53)  & (0.06, 0.54)  & (0.07, 0.32) &  Chabrier \\
TNG100 & (0.27, 0.48) & (0.24, 0.46)  & (0.24, 0.45)  & (0.38, 0.33) & Chabrier \\
TNG300 & (0.26, 0.48) & (0.20, 0.48)  & (0.17, 0.48)  & (0.41, 0.34) & Chabrier \\
Horizon-AGN & (0.16, 0.49) & (0.14, 0.47)  & (0.23, 0.47)  & (0.47, 0.35) & Salpeter\\
\enddata
\tablecomments{This table collects the comparison results of the \smass-\mbh\ correlations between different simulation at different redshift. The value shows the central position offset to the local relation and the scatters measured around the local relation after applying the offset. A positive offset means the \mbh\ value predicted by the simulation is higher than the local relationship measurement at fixed \smass\ value. The last column shows the corresponding IMF that adopted to the local anchor to make a fair comparison with the observation. Note that for the observational data, the relative differences between local and high-$z$ measurements are not affected by the IMF assumptions.
For the MBII sample, the simulation does not produce the sample at $z=0.5$ or $z=0.7$, but rather at $z=0.6$. We use Monte Carlo approach to infer the uncertainties of the values in the table, finding that the uncertainties are within $\pm 0.03$.
}
\end{deluxetable*}


\section{Conclusions} \label{sec:con}
We compared the observed scaling relation \mbh-\smass\ with the predictions from numerical simulations. The observation data are composed of 626 quasars at $0.2 < z < 0.8$ imaged by HSC and 32 X-ray-selected quasars at $1.2 < z < 1.7$ imaged by HST. The simulations include a semi-analytic model (SAM) and five hydrodynamic simulations, i.e., MBII, Illustris, TNG100, TNG300 and Horizon-AGN. We carried out the comparisons in the observed pararameter space to account for uncertainties and selection effects. To achieve this, we first injected random errors with the same observational uncertainty to the simulation, and then adopted the same selection condition to the simulation data (see Figure~\ref{fig:selection}). Finally, we adopted the scaling relation from the local universe as our reference and performed comparisons using the scatter of the measurements and their central offset to the local relation. Our main results are summarized as follows:

\begin{enumerate}

\item{}The {\it observed} scatter predicted by the simulations is consistent with the observational measurements, i.e., $\sim0.5$~dex at $z<1$ and $\sim0.3$~dex at $z>1$ (see Figure~\ref{fig:offsets_vz} and Table~\ref{tab:sum}). This result indicates that the simulated and observed samples have consistent {\it intrinsic} scatter.

\item{}To understand how much the {\it observed} scatter is dominated by random observational error,
we re-run the estimation without injecting noise to the simulations. The obtained scatter for both $z<1$ and $z>1$ are at a similar level (i.e.,  $\sim$0.15$-$0.2 dex, see Table~\ref{tab:sum_no_noise}), indicating that observational errors dominate the scatter.

\item{} Regarding the mass ratios and offsets from the local relation ($\Delta$\mbh\ at a given \smass\ ), all simulations generally match the observations with some subtle, yet notable, differences. While Illustris and Horizon-AGN show good correspondence with observations at $z<0.6$, the comparisons at $z>0.6$ are better for SAM, TNG100, TNG300 and Horizon-AGN. From $z\sim0.7$ to $z\sim1.5$, TNG100, TNG300 and Horizon-AGN simulations match well the observed evolution of the scaling relation, i.e., the offsets are larger at higher redshift as shown in Figure~\ref{fig:offsets_vz} and Table~\ref{tab:sum}. 

\end{enumerate}

The absolute value of stellar masses in both observation and theory have significant uncertainty (up to factor of two), which depends on the assumption of initial mass function, and possibly on the implementation of star formation in the models. In contrast, the scatter around the mean correlation is a relative quantity, which is less affected by such systematic effect. Thus, in this work, we first consider the scatter as a diagnostic criteria to see whether some simulations match the data better than others. Taking (1) and (2), our results suggest that the tightness of the scaling relations have been formed since redshift 1.7, which is in contrast with the scenario of the central limit theorem~\citep{Peng2007, Jahnke2011, Hirschmann2010} that the scaling relation is a consequence of a stochastic cloud in the early universe with subsequent random mergers thereafter. In this stochastic scenario we expect the scatter of the scaling relations to increase towards higher redshift. In fact, the scatter level in the simulation without adding random noise is consistent with the {\it intrinsic} scatter estimations reported in~\citet{Ding2020, 2021arXiv210902751L} (i.e., $\lesssim0.35$~dex). This level is also not larger than the typical scatter of the local relations reported in the literature~\citep{Kormendy13, Gul++09, Reines2015}.

The simulations studied in this work have adopted completely different numerical techniques. However, all of them can provide good agreement with the observed dispersion in the scaling relation. In fact, the tightness of the scaling relation is stem from the same physics assumed in these simulations (i.e., AGN feedback). Thus, the consistency of the scatter between simulation and observations is consistent with the hypothesis that the AGN feedback as a causal link between SMBHs and their hosts plays a key role in establishing the scaling relation. 

We can gain more insight into the role of feedback by looking at the SAM model, for which multiple feedback models have been implemented. \citet{Ding2020b} compared the scaling relations obtained with the same HST sample and the SAM simulation but with a different, isotropic, AGN feedback model, and found a larger scatter ($\sim0.7$~dex) with respect to the present SAM version ($\sim0.36$~dex). We ascribed the change to the following reasons: in the new 2D model for feedback, the wave expansions stalls along the direction of the disc, and the radius where the expansion stops depends strongly on both the gas density of the disk and the AGN luminosity. 
This means that the opening angle (and hence the fraction of expelled gas) is larger when the gas density is small (because of the lower energy that has to be spent to push the gas outwards) and when the AGN luminosity is large (because of the larger energy available to push the blast wave outwards). These dependencies are summarized in Figure~\ref{fig:SAM}. 
Both quantities depend on the merging histories and are  related, since the AGN luminosity $L_{\rm AGN}$ depends on the available cold gas reservoir $M_{\rm gas}$.
The large efficiency of feedback in galaxies with particularly small $M_{\rm gas}$ (for given $L_{\rm AGN}$) or in those with particularly large $L_{\rm AGN}$
(for given $M_{\rm gas}$) inhibits the BH growth in all the host galaxies that are outliers with respect to the average relation between $M_{\rm gas}$ and $L_{\rm AGN}$. 
This results into a smaller scatter.

In theoretical models, AGN feedback is often assumed to consist of two distinct modes:
a quasar-heating mode where the SMBH accretion rates are comparable to the Eddington rate and a radio-jet mode occurring at low accretion rates (see e.g. Section~\ref{subsec:Horizon}). In high redshift universe, the cold material in early universe leads to the vigorous accretion to the SMBH which drives the high accretion rates, and thus the quasar mode dominates the feedback. At low redshift, the star formation and feedback ejection reduce the cold material leading to a lower accretion rate and a radio-mode-dominating feedback~\citep[e.g.][]{2012MNRAS.420.2662D,2016MNRAS.460.2979V,2018MNRAS.479.4056W}. Our comparison result shows that the level of intrinsic scatter in the scaling relation at redshifts up to 1.7 is consistent with the low redshift one (see Table~\ref{tab:sum_no_noise}), which reveals the fact that at high redshift the AGN feedback described by the quasar mode has already regulated the tight correlation between SMBH and its host galaxy. After that, the radio-jet mode starts to take control at low redshift by maintaining the tightness of the scaling relation till the level we observe today.

Our work highlights the importance of applying measurement uncertainty and the effect of selection to the simulated data in order to make direct comparisons with observations. Such comparisons have been made in the local universe~\citep[e.g.,][]{Habouzit2021} where the measurements are relatively robust and the selection function is broad thus it is less crucial to ensure consistency between observations and simulations. However, beyond $z>0.2$, the scatter and the central distribution of the scaling relations are dominated by measurement uncertainty and selection effects (see Figures~\ref{fig:comparsion} and~\ref{fig:deltaMM}) and a forward modeling in the observational plane becomes essential. For example, from trends seen with stellar mass in Illustris and MBII (Figure~\ref{fig:deltaMM}), selection effects may hamper our understanding of whether BHs and their hosts co-evolve or not.

Extending this study to even higher redshift (and lower mass galaxies, \smass\ $<10^{10}M_{\odot}$) will be very beneficial, probing closer to the epoch of formation of massive galaxies and SMBHs. The understanding of how and when the tight scaling relation emerged are crucial to test theoretical models \citep{Volonteri2021}. On the observational side, the James Webb Space Telescope will provide high-quality imaging data of AGNs at redshift up to $z\sim7$ (i.e., JWST cycle 1 program GO-1967 and GO-1727). These upcoming measurements will represent stringent tests on the proposed physical mechanisms for the initial formation of supermassive black holes and of their subsequent evolution with galaxies.

\begin{acknowledgments}
This work was supported by World Premier International Research Center Initiative (WPI), MEXT, Japan. 
The authors fully appreciate input from Jingjing Shi.

Based in part on observations made with the NASA/ESA Hubble Space Telescope, obtained at the Space Telescope Science Institute, which is operated by the Association of Universities for Research in Astronomy, Inc., under NASA contract NAS 5-26555. These observations are associated with programs \#15115. Support for this work was provided by NASA through grant number HST-GO-15115 from the Space Telescope Science Institute, which is operated by AURA, Inc., under NASA contract NAS 5-26555. 
JS is supported by JSPS KAKENHI Grant Number JP18H01251 and the World Premier International Research Center Initiative (WPI), MEXT, Japan. TT acknowledges support by the Packard Foundation through a Packard Research fellowship to TT. LB acknowledges support from NSF award AST-1909933 and Cottrell Scholar Award \#27553 from the Research Corporation for Science Advancement.
\end{acknowledgments}


\begin{deluxetable*}{lcccc}
\tablecaption{Summary of the central offsets and scatters without noise\label{tab:sum_no_noise}}
\tablewidth{0pt}
\tablehead{
\colhead{Simulation} &  \multicolumn{3}{c}{HSC comparison (offset, scatter)} & \colhead{HST comparison (offset, scatter)} \\
\cline{2-4} 
\colhead{} &  \colhead{$0.2<z<0.4$} & \colhead{$0.4<z<0.6$} & \colhead{$0.6<z<0.8$} & \colhead{$1.2<z<1.7$}
}
\startdata
SAM & (0.72, 0.20) & (0.64, 0.18) & (0.56, 0.16)  & (0.08, 0.18) \\
MBII & (-0.15, 0.21) & \multicolumn{2}{c}{(-0.15, 0.22) [$z=0.6$]}  & (0.08, 0.19)\\
Illustris & (0.03, 0.32) & (0.10, 0.36) & (0.08, 0.36)  & (0.04, 0.19) \\
TNG100 &  (0.27, 0.20) & (0.27, 0.15) & (0.26, 0.16)  & (0.36, 0.15) \\
TNG300 & (0.25, 0.21) & (0.19, 0.23) & (0.20, 0.22)  & (0.32, 0.16) \\
Horizon-AGN & (0.19, 0.26) & (0.18, 0.21) & (0.28, 0.20)  & (0.37, 0.13) \\
\enddata
\tablecomments{Same as Table~\ref{tab:sum} but without inject noise to the simulation data. Note that same selection window is still adopted to infer the values.}
\end{deluxetable*}


\begin{deluxetable*}{lcccc}
\tablecaption{Summary of the p-value using KS test \label{tab:pvalue}}
\tablewidth{0pt}
\tablehead{
\colhead{Simulation} &  \multicolumn{3}{c}{HSC comparison} & \colhead{HST comparison} \\
\cline{2-4} 
\colhead{} &  \colhead{$z\sim0.3$} & \colhead{$z\sim0.5$} & \colhead{$z\sim0.7$} & \colhead{$z\sim1.5$}
}
\startdata
SAM &  3.119926e-09 & $<$1e-10  & $<$1e-10  & 7.456907e-03  \\
MBII & 4.228845e-02 & \multicolumn{2}{c}{$<$1e-10 [$z=0.6$]}  & 2.218653e-06  \\
Illustris & 2.306434e-01 & 6.144066e-02  & 9.963678e-03  & 2.330195e-07  \\
TNG100 & 1.207389e-02 & 2.530020e-01  & 3.756821e-01  & 4.708287e-01  \\
TNG300 & 1.817531e-02 & 9.192506e-01  & 1.208812e-01  & 2.118038e-01  \\
Horizon-AGN & 2.699897e-01 & 2.865927e-01  & 6.129553e-01  & 1.953735e-01  \\
\enddata
\tablecomments{These p-values are obtained by the KS test between the simulation and the observation.}
\end{deluxetable*}

%\bibliography{reference}{}
%% Beginning of file 'sample631.tex'
%%
%% Modified 2021 March
%%
%% This is a sample manuscript marked up using the
%% AASTeX v6.31 LaTeX 2e macros.
%%
%% AASTeX is now based on Alexey Vikhlinin's emulateapj.cls 
%% (Copyright 2000-2015).  See the classfile for details.

%% AASTeX requires revtex4-1.cls and other external packages such as
%% latexsym, graphicx, amssymb, longtable, and epsf.  Note that as of 
%% Oct 2020, APS now uses revtex4.2e for its journals but remember that 
%% AASTeX v6+ still uses v4.1. All of these external packages should 
%% already be present in the modern TeX distributions but not always.
%% For example, revtex4.1 seems to be missing in the linux version of
%% TexLive 2020. One should be able to get all packages from www.ctan.org.
%% In particular, revtex v4.1 can be found at 
%% https://www.ctan.org/pkg/revtex4-1.

%% The first piece of markup in an AASTeX v6.x document is the \documentclass
%% command. LaTeX will ignore any data that comes before this command. The 
%% documentclass can take an optional argument to modify the output style.
%% The command below calls the preprint style which will produce a tightly 
%% typeset, one-column, single-spaced document.  It is the default and thus
%% does not need to be explicitly stated.
%%
%% using aastex version 6.3
%\documentclass[linenumbers]{aastex631}
\documentclass[twocolumn]{aastex631}

\newcommand{\vdag}{(v)^\dagger}
\newcommand\aastex{AAS\TeX}
\newcommand\latex{La\TeX}
\newcommand{\todo}[1]{\textcolor{red}{[{\bf TODO}: #1]}}
\newcommand{\ding}[1]{\textcolor{red}{[{\bf Xuheng}: #1]}}
\newcommand{\jshi}[1]{\textcolor{orange}{#1}}
\newcommand{\blue}[1]{\textcolor{blue}{#1}}
\newcommand{\yo}[1]{\textcolor{purple}{[{\bf Yohan}: #1]}}
\newcommand{\red}[1]{\textcolor{purple}{#1}}

\def\smass{{$M_*$}}
\def\sersic{S\'ersic}
\def\halpha{${\rm H}\alpha$}
\def\hbeta{${\rm H}\beta$}
\def\mbh{$\mathcal M_{\rm BH}$}
\def\lcdm{$\Lambda$CDM}
\def\hst{{\it HST}}
\def\kms{km~s$^{\rm -1}$}


\newcommand{\aklant}[1]{\textcolor{blue}{#1}}
%%%%%%%%%%%%%%%%%%%%%%%%%%%%%%%%%%%%%%%%%%%%%%%%%%%%%%%%%%%%%%%%%%%%%%%%%%%%%%%%
%%
%% The following section outlines numerous optional output that
%% can be displayed in the front matter or as running meta-data.
%%
%% If you wish, you may supply running head information, although
%% this information may be modified by the editorial offices.
\shorttitle{Comparing simulations and observations of black hole - galaxy relations}
\shortauthors{Ding et al.}
%%%%%%%%%%%%%%%%%%%%%%%%%%%%%%%%%%%%%%%%%%%%%%%%%%%%%%%%%%%%%%%%%%%%%%%%%%%%%%%%
\graphicspath{{./}{figures/}}
%% This is the end of the preamble.  Indicate the beginning of the
%% manuscript itself with \begin{document}.

\begin{document}

%\title{Black Hole and Galaxy co-evolution at $0.2<z<1.7$ with extensive comparisons between observations and simulations}

\title{Concordance between observations and simulations in the evolution of the mass relation between supermassive black holes and their host galaxies}

\author[0000-0001-8917-2148]{Xuheng Ding}
\affiliation{Kavli Institute for the Physics and Mathematics of the Universe, The University of Tokyo, Kashiwa, Japan 277-8583 (Kavli IPMU, WPI)}

\author[0000-0002-0000-6977]{John D.Silverman}
\affiliation{Kavli Institute for the Physics and Mathematics of the Universe, The University of Tokyo, Kashiwa, Japan 277-8583 (Kavli IPMU, WPI)}

\author[0000-0002-8460-0390]{Tommaso Treu}
\affiliation{Department of Physics and Astronomy, University of California, Los Angeles, CA, 90095-1547, USA}

\author[0000-0002-1605-915X]{Junyao Li}
\affiliation{CAS Key Laboratory for Research in Galaxies and Cosmology, Department of Astronomy, University of Science and Technology of China, Hefei 230026, China}
\affiliation{School of Astronomy and Space Science, University of Science and Technology of China, Hefei 230026, China}
\affiliation{Kavli Institute for the Physics and Mathematics of the Universe, The University of Tokyo, Kashiwa, Japan 277-8583 (Kavli IPMU, WPI)}

\author[0000-0002-7080-2864]{Aklant K. Bhowmick}
\affiliation{Dept. of Physics, University of Florida, Gainesville, FL 32611, USA}

\author[0000-0002-4096-2680]{Nicola Menci}
\affiliation{INAF Osservatorio Astronomico di Roma, via Frascati 33, I-00078 Monteporzio, Italy}

\author[0000-0002-3216-1322]{Marta Volonteri}
\affiliation{Institut d'Astrophysique de Paris, Sorbonne Universit\'e, CNRS, UMR 7095, 98 bis bd Arago, 75014 Paris, France}

\author[0000-0002-2183-1087]{Laura Blecha}
\affiliation{Dept. of Physics, University of Florida, Gainesville, FL 32611, USA}

\author[0000-0002-6462-5734]{Tiziana Di Matteo}
\affiliation{McWilliams Center for Cosmology, Dept. of Physics, Carnegie Mellon University, Pittsburgh, PA 15213, USA}

\author[0000-0003-0225-6387]{Yohan Dubois}
\affiliation{Institut d'Astrophysique de Paris, Sorbonne Universit\'e, CNRS, UMR 7095, 98 bis bd Arago, 75014 Paris, France}


%\author[0000-0001-9879-4926]{Jingjing Shi}
%\affiliation{Kavli Institute for the Physics and Mathematics of the Universe, The University of Tokyo, Kashiwa, Japan 277-8583 (Kavli IPMU, WPI)}

%% Mark off the abstract in the ``abstract'' environment. 
\begin{abstract}
We carry out a comparative analysis of the relation between the mass of supermassive black holes (BHs) and the stellar mass of their host galaxies (\mbh -- \smass) at $0.2<z<1.7$ using well-matched observations and multiple state-of-the-art simulations (e.g., Massive Black II, Horizon-AGN, Illustris, TNG and a semi-analytic model). The observed sample consist of 646 uniformly-selected SDSS quasars ($0.2 < z < 0.8$) and 32 broad-line AGNs ($1.2<z<1.7$) with imaging from Hyper Suprime-Cam (HSC) for the former and Hubble Space Telescope (HST) for the latter. To make fair comparisons, we first add realistic observational uncertainties to the simulation data and then construct a simulated sample in the same manner as the observations. Over the full redshift range, our analysis demonstrates that all simulations predict a level of intrinsic scatter of the scaling relations comparable to the observations which appear to agree with the dispersion of the local relation. 
Regarding the mean scaling relation, Horizon-AGN and TNG are in closest agreement with the observational data at both low and high redshift ($z\sim$ 0.2 and 1.5, respectively) while the other simulations show subtle differences within the uncertainties. 
%TNG100, TNG300 and Horizon-AGN can successfully predict the observed evolution of the offset of the \mbh\ (i.e., $\Delta$\mbh\ increasing from $z\sim$ 0.7 to $z\sim1.5$).
For insight into the physics involved, the scatter of the scaling relation, seen in the SAM, is reduced by a factor of two and thus closer to the observations after adopting a new feedback model which considers the geometry of the AGN outflow. The consistency in the dispersion with redshift in our analysis lends support to the importance of both quasar and radio mode feedback prescriptions in the simulations. 
%In conclusion, our results are consistent with the hypothesis that AGN feedback plays a role in a causal link between the BH and its host galaxy.
%Our observational results valid the simulation prediction that, at redshift up to $z\sim1.5$, the tight scaling relation has been established by the AGN feedback in which  quasar mode (thermal bursts) essentially dominates; while at low redshift ($z<1$), the radio mode (jet outflows) starts to dominate the feedback and maintains the tightness of the scaling relation till $z=0$. 
Finally, we investigate evolution as a function of stellar and black hole mass and argue that the most direct way to probe the co-evolution of galaxies and black holes is to increase the sensitivity, e.g. with JWST, thereby pushing to lower masses and minimizing possible biases arising from selection effects.
%Finally, we investigate the offsets across the range in stellar mass; some simulations show that observational effects may lend ambiguity on whether black holes and their host are co-evolving or not.

%, finding that, after observational effect, different intrinsic offset values at lower stellar mass could result in the consistent comparison with the observation, which highlight the importance of extend this comparison work to lower stellar mass ($<10^{10}M_{\odot}$). 


\end{abstract}
\keywords{Galaxy evolution (594); Active galaxies (17); Active galactic nuclei (16)}

\section{Introduction} \label{sec:intro}

The close correlations between supermassive black holes and the properties of their host galaxies (e.g., stellar mass) indicate a physical coupling during their joint evolution~\citep{Mag++98, F+M00, M+H03, H+R04, Gul++09}. To understand the nature of this connection, considerable efforts have been focused on measuring such correlations using broad-line active galactic nucleus (AGN) over a range of redshifts with the intention to determine how and when the correlation emerges and evolves over cosmic time \citep[e.g.,][]{Tre++04,Peng2006a, Tre++07, Woo++08, Jahnke2009, Bennert11, Schramm2013, Park15, Mechtley2016, Ding2020, 2021arXiv210902751L}. While, an {\it observed} evolution has been found \blue{in many studies} in which the growth of BHs predates that of the host, \blue{other studies predict that BHs grow commensurately with galaxies}.
%equally as many studies claim a lack of evolution when considering the host total stellar mass  \aklant{\bf This sentence can be made more clear. Are you just saying the some studies predict BHs grow faster than galaxies, whereas other studies predict that BHs grow commensurately with galaxies?}. 
However, to understand the significance {\it intrinsic} evolution, it is necessary to take into account systematic uncertainties and the selection effects~\citep{Tre++07, Lauer2007, Schulze2014, Park15}. 

Various theoretical models have been proposed to explain the origin of the scaling relations. For example, AGN feedback is considered as one of the possible viable mechanisms. During this process, a fraction of the AGN energy is injected into its surrounding gas, which can then regulate the mass growth of the BH and its host. In this scenario, star formation is inhibited by the heating and unbinding of a significant amount of gas. Alternatively, the mass relations can be explained through an indirect connection in which AGN accretion and star formation are fed through a common gas supply~\citep{Cen2015, Menci2016}. Actually, even without any physical mechanisms, statistical convergence from galaxy assembly alone (i.e., dry mergers) could instill the observed correlations~\citep{Peng2007, Jahnke2011, Hirschmann2010}. However, as expected from the central limit theorem, a higher dispersion would appear in the scaling relations at high-$z$ %\aklant{\bf{do you mean "compared to what is observed today"?}} 
\blue{compared to what is observed today}. 

Numerical simulations provide an opportunity to further understand the connection between BHs and their host galaxies. For example, a comparison of scaling relations has been made using state-of-the-art cosmological hydrodynamical simulation of structure formation ({\tt MassiveBlackII}) and observational measurements at  $0.3<z<1$ \citep[e.g., ][]{DeG++15}, which show a positive evolution (i.e., the mass growth of the BH growth predates that of its host). Further efforts are using large-volume simulations to investigate the scaling relations and find good agreement with the local relation with redshift evolution, 
%\jshi{(Jingjing: the agreement is with the local relation, right? they all predict their own redshift evolution., Xuheng's reply. Yes, I have updated this information in the last sentence.)}
including the Magneticum Pathfinder smooth-particle hydrodynamics (SPH) Simulations~\citep{Steinborn2015}, the Evolution and Assembly of Galaxies and their Environments~(EAGLE) suite of SPH simulations~\citep{Schaye2015}, Illustris moving-mesh simulation~\citep{Sijacki2015, 2014MNRAS.444.1518V, Li2019} and the SIMBA simulation~\citep{Thomas2019}. %\yo{I am not sure I understand the link between this sentence and the previous one: ~\cite{Habouzit2021} includes Horizon-AGN and TNG (not cited previously), and does not include Magneticum.}
\red{In particular, the \mbh-\smass\ relation using BH populations using six large-scale cosmological simulations (i.e, Illustris, TNG100, TNG300, Horizon-AGN, EAGLE, and SIMBA) has been compared with observations in the local universe in a recent study~\citep{Habouzit2021}.} However, these comparison works are limited by the observation data in terms of the sample size ($<$100) and redshift range \blue{(i.e., limited to the local universe)}.

For such comparisons using simulations, it is crucial to consider the systematic uncertainties and selection biases. A direct means to account for these is to apply the same effects and selection to the simulation products and make a forward comparison in the observational plane. In~\citet{Ding2020b}, a direct comparison has been performed using 32 X-ray-selected AGN at $1.2<z<1.7$ and a direct comparison with two state-of-the-art simulation efforts, including {\tt MassiveBlackII} (MBII) and a Semi-analytic Model \citep[SAM,][]{Menci2014, Menci2016}. The dispersion in the mass ratio between black hole mass and stellar mass is significantly more consistent with the MBII prediction ($\sim0.3$~dex) \blue{favoring} the hypothesis of AGN feedback being responsible for a causal link between the BH and its host galaxy.

In this study, we extend our previous work \blue{by adding recent measurements of hundreds of SDSS quasars at $0.2<z<0.8$ based on wide and deep HSC imaging from the Strategic Subaru Program,} and comparing the observational measurements with that from simulations. Furthermore, we extend the simulated quasar populations by including MBII, SAM, Illustris, TNG100, TNG300 and Horizon-AGN. This paper is structured as follows. In Sections~\ref{sec:observations} and~\ref{sec:simulations}, we describe our observed and simulated samples. A direct comparison is performed and the result is presented in Section~\ref{sec:result}. The concluding remarks are presented in Section~\ref{sec:con}.

%\section{Comparison Sample: Observations and Simulations} \label{sec:sample}
%The correlations of \mbh-\smass\ between the observed data and the numerical simulations are compared in this work. 
%We introduce the comparison samples in this section, including the observational data and the numerical simulations.

\section{Observational data set}
\label{sec:observations}
The observed sample consists of 646 uniformly-selected SDSS quasars at $0.2<z<0.8$, imaged by Subaru/HSC \citep{Li2021a}, and 32 quasars at $1.2<z<1.7$ as imaged by~\hst~\citep[][hereafter D20]{Ding2020}. The latter are selected from three deep-survey fields, namely COSMOS~\citep{Civano2016}, (E)-CDFS-S~\citep{Lehmer2005, Xue2011}, and SXDS~\citep{Ueda2008}. Further details of these two samples and their measurements are given below. 

\subsection{SDSS/HSC sample}\label{sec:hsc}
A sample of $\sim$5000 type-1 SDSS quasars from the DR14 catalog~\citep{Paris2018} at $0.2<z<1$ has been imaged by the high-resolution Subaru Strategic Program (SSP) wide area survey~\citep{Aihara2019} using Hyper Suprime-Cam~\citep{Miyazaki2018}. With accurate PSF models in five optical bands {\it grizy}, two-dimensional quasar-host decompositions have been performed \citep[][hereafter L21a]{Li2021a} to obtain the flux and color of each quasar's host galaxy. The stare-of-the-art image modeling software {\tt lenstronomy} \citep{Birrer2015, Birrer2018, Birrer2021} is adopted to perform the modeling task. This approach is first developed by~\citet{Ding2020} and used to decompose the near-infrared emission of the HST sample (see next section). Having measured the host light in each band, the stellar mass of host galaxy is derived using spectral energy distribution (SED) fitting with CIGALE~\citep{Boquien2019}. Simulation tests are also performed to verify the fidelity of the \smass\ measurements. The statistical measurement error on \smass\ is at the $\sim$0.2~dex level. The values of \mbh\ are determined by~\citet{Rakshit2020} which are estimated based on the \hbeta-based measurements using the virial method~\citep{Peterson2004, Vestergaard2006}. The typical error of \mbh\ are estimated to be 0.4 dex. We refer the reader to~\citet{Li2021a} in the Section~4.2 for more details.

To avoid any potential biases related to the selection of the quasars, \citet{2021arXiv210902751L} isolated 877 sources which are uniformly selected based on their PSF-magnitudes, color cuts using single-epoch SDSS photometry and the value of the measured \smass. Specifically, we use the {\it ugri} color-selected sample (228 sources) from SDSS I/II~\citep{Richards2002}, and the CORE sample from SDSS BOSS (408 sources) and eBOSS (241 sources) surveys~\citep{Ross2013, Myers2015} (hereafter the uniform sample). These samples are initially selected based on PSF-magnitude cuts of $15 < i < 19.1$ (for {\it ugri}), and $i > 17.8$ and $g, r < 22.0$ (for CORE). Furthermore, a limit on \smass\ is set to assure the detection of the host, especially since the rate and accuracy of detection is higher when \smass\ is increasing\blue{, resulting in a final sample of 646 quasars}. These selections will be adopted in an equivalent manner to the simulated samples to mitigate selection effects thus allowing the fair comparison.

\subsection{HST sample}

A sample of 32 HST-observed AGN systems across the redshift range $1.2<z<1.7$ are selected from three deep-survey fields (COSMOS, (E)-CDFS-S, and SXDS). HSC/WFC3 IR camera is used to obtain the high-resolution imaging data (HST program GO-15115, PI: John Silverman) with six position dither pattern and a total exposure time $\sim$2348~s. The filters F125W ($1.2<z<1.44$) and F140W ($1.44<z<1.7$) were employed, according to the redshift of each target to bracket the 4000~\AA~break.  The AGN images are analyzed and decomposed to infer the host galaxies fluxes using the approach developed by D20 based on {\tt lenstronomy}. The HST ACS/F814W imaging data for 21/32 of the AGNs is also used to infer the host color. The results show that stellar templates of 1 and 0.625~Gyr can match the sample color at $z<1.44$ and $z>1.44$, respectively (see Figure 5 in D20). These best-fit models are used to estimate the stellar masses of the host galaxies. \mbh\ is determined by \citet{Schulze2018} using near-infrared spectroscopic observations of the broad \halpha\ emission line with the recipe provided by~\citet{Vestergaard2006}, in a consistent manner to that adopted for HSC sample. We refer the reader to D20 for a more detailed description of the analysis. 

The measurements of the \mbh-\smass\ relations for both the HST and HSC samples are obtained with a consistent approach. Thus, we expect the measurement errors of these two samples to be at a comparable level (i.e., $\Delta$\mbh$=0.4~$dex, $\Delta$\smass$=0.2~$dex). 
Indeed, the two samples are consistent with a lack of evolution in the mass ratio~\citep[see Figure 6 of][]{2021arXiv210902751L}, even though the sample selection is slightly different.


%\begin{deluxetable*}{ccccccc}
%%\tablenum{1}
%\tablecaption{Key characteristics of hydrodynamic simulations used in this study.\label{tab:sim_sum}}
%\tablewidth{0pt}
%\tablehead{
%%\colhead{Simulation} &  \multicolumn{3}{c}{HSC comparison (offset, scatter)} & \colhead{HST comparison (offset, scatter)} \\
%%  \cline{2-4}  \cline{5} \\
%\colhead{Simulation} & \colhead{box sizes} & particles &  \multicolumn{2}{c}{mass resolution ($M_{\odot}$)} & \colhead{feedback prescription} & \colhead{AGN fueling mechanism}  %& what else?
%\\
%\cline{4-5}
%&$(\mathrm{cMpc})^3$&& dark matter & baryonic matter && $\dot{M}_{BH}$
%}
%\startdata
%MBII & $142.7^3$ & $2\times1792^3$ & $1.57\times10^7$ & $3.14\times10^6$ &5\% & ${4\pi G^2 M_{BH}^2 \rho}/{(c_s^2+v_{BH}^2)^{3/2}}$ \\
%Illustris & $(106.5)^3$ & $2\times1820^3$ & $6.26\times10^6$ & $1.26\times10^6$ &5\%& $\alpha \dot{M}_{Bondi}$ \\
%TNG100 &$(111)^3$& $2\times1820^3$&$7.5\times10^6$ &$1.4\times10^6$& min[$2\times10^{-3}(\frac{M_{\rm BH}}{10^8M_{\odot}})^2$, 10\%] & \blue{$\dot{M}_{Bondi}$}\\
%TNG300 &$(302)^3$& $2\times2500^3$&$5.9\times10^7$&$1.1\times10^7$&
%min[$2\times10^{-3}(\frac{M_{\rm BH}}{10^8M_{\odot}})^2$, 10\%]& \blue{$\dot{M}_{Bondi}$} \\
%Horizon-AGN &$(142)^3$& $\sim2\times1024^3$ &$8\times10^7$&$2\times10^6$&1.5\% - 10\% & $\alpha\dot{M}_{Bondi}$ \\
%\enddata
%\tablecomments{%\aklant{\bf{Could you please specify the maximum accretion rate for all the simulations?}}
%\blue{For all simulations, the accretion rate on to BHs is capped at the Eddington accretion rate, expect MBII which has a mildly super-Eddington (i.e., two times Eddington rate).}
%\red{In the last column, $\dot{M}_{Bondi}= 4\pi G^2M_{BH}^2 \rho/c_s^3$. $\alpha$ is the boost factor. For Illustris, $\alpha = 100$; for Horizon-AGN, $\alpha =$ max$((\rho/\rho_0)^2$, 1).}
%}
%%\tablecomments{This table }
%\end{deluxetable*}

\begin{deluxetable*}{lccccc}
%\tablenum{1}
\tablecaption{Key characteristics of hydrodynamic simulations used in this study.\label{tab:sim_sum}}
\tablewidth{0pt}
\tablehead{
%\colhead{Simulation} &  \multicolumn{3}{c}{HSC comparison (offset, scatter)} & \colhead{HST comparison (offset, scatter)} \\
%  \cline{2-4}  \cline{5} \\
\colhead{Simulation} & MBII & Illustris &  TNG100 & TNG300 & Horizon-AGN
%&$(\mathrm{cMpc})^3$&& dark matter & baryonic matter && $\dot{M}_{BH}$
}
\startdata
box sizes $(\mathrm{cMpc})^3$  & $142.7^3$ & $(106.5)^3$ & $(111)^3$ & $(302)^3$ & $(142)^3$ \\
particles  & $2\times1792^3$ & $2\times1820^3$ & $2\times1820^3$ & $2\times2500^3$ & $\sim2\times1024^3$ \\
\cline{1-6}
{\bf mass resolution} \\
 dark matter &$1.57\times10^7$ & $6.26\times10^6$ & $7.5\times10^6$ & $5.9\times10^7$ & $8\times10^7$ \\
 baryonic matter &$3.14\times10^6$ & $1.26\times10^6$ & $1.4\times10^6$ & $1.1\times10^7$ & $2\times10^6$ \\
\cline{1-6}
{\bf AGN feedback} & (feedback efficiency $\times\epsilon_r$) \\
High acc. mode & 0.05 $\times$ 0.1 & 0.05 $\times$ 0.2 & 0.1 $\times$ 0.2 & 0.1 $\times$ 0.2 & 0.15 $\times$ 0.1 \\
Low acc. mode  & -- & 0.35 $\times$ 0.2 & $\leq0.2 \times$ 0.2 & $\leq0.2 \times$ 0.2 & 1 $\times$ 0.1 \\
Transitions btw. modes & -- & 0.05 & \multicolumn{2}{c}{min[$2\times10^{-3}(\frac{M_{\rm BH}}{10^8M_{\odot}})^2$, 10\%]} & 0.01 \\
\cline{1-6}
AGN fueling mechanism   & ${4\pi G^2 M_{BH}^2 \rho}/{(c_s^2+v_{BH}^2)^{3/2}}$ & $\alpha \dot{M}_{Bondi}$ & \blue{$\dot{M}_{Bondi}$}&\blue{$\dot{M}_{Bondi}$} & $\alpha\dot{M}_{Bondi}$ \\
maximum accretion rate & 2 $\times$Edd. acc. rate & Edd. acc. rate & Edd. acc. rate & Edd. acc. rate & Edd. acc. rate\\
\enddata
\tablecomments{%\aklant{\bf{Could you please specify the maximum accretion rate for all the simulations?}}
%\blue{For all simulations, the accretion rate on to BHs is capped at the Eddington accretion rate, expect MBII which has a mildly super-Eddington (i.e., two times Eddington rate).}
\red{In the penultimate row, $\alpha$ is the boost factor. For Illustris, $\alpha = 100$; for Horizon-AGN, $\alpha =$ max$[(\rho/\rho_0)^2, 1]$. $\dot{M}_{Bondi}= 4\pi G^2M_{BH}^2 \rho/c_s^3$.}
}
%\tablecomments{This table }
\end{deluxetable*}


\section{Simulations \blue{and comparison strategy}}
\label{sec:simulations}
We introduce the simulation samples that are adopted in this section. All the simulation are based on the larger-scale cosmological simulations, except the Semianalytic Model (SAM) simulation \blue{(see Section~\ref{subsec:SAM})}. In Table~\ref{tab:sim_sum}, we summarized the key elements for each hydrodynamic simulation being considered.
%\todo{Maybe we can also summarize the importance of AGN feedback somewhere here?}

\subsection{{\tt MassiveBlackII} (MBII)}\label{subsec:MBII}
MBII is a high-resolution cosmological hydrodynamic simulation that has a box size of %$100~\mathrm{cMpc/h}$
\red{$(142.7~\mathrm{cMpc})^3$}
and $2\times1792^3$ particles. The simulation is based on smooth particle hydrodynamic (SPH) code \texttt{P-GADGET}, a hybrid version of the parallel code {\tt GADGET}~\citep{2005MNRAS.364.1105S}. 
%\yo{if the values of the cosmological parameters are given for MBII maybe this should also be given for all the other simulations?}
The base cosmology parameters are based on the WMAP7 results~\citep{2011ApJS..192...18K}.
%, i.e., $\Omega_0=0.275$, $\Omega_l=0.725$, $\Omega_b=0.046$, $\sigma_8=0.816$, $h = 0.701$, $n_s=0.968$.
%\yo{Masses are given here in $M_\odot/h$, while in some other places this is given in $M_\odot$. The same for box lengths.}
For dark matter and gas, the mass resolutions are %$1.1\times 10^7~M_{\odot}/h$ and $2.2\times 10^6~M_{\odot}/h$
\red{$1.57\times 10^7~M_{\odot}$ and $3.14\times 10^6~M_{\odot}$}
, respectively. The simulation includes a full modeling of gravity plus gas hydrodynamics, with a wide range of subgrid recipes to model the star formation~\citep{2003MNRAS.339..289S}, BH growth, and the feedback process~\citep{2005Natur.433..604D}. Halos were identified using a friends-of-friends (FOF) group finder~\citep{1985ApJ...292..371D}. Galaxies are identified with the stellar matter components of subhalos; these subhalos are identified using {\tt SUBFIND} within the halos~\citep{2005MNRAS.364.1105S}.

To model supermassive black holes, a BH seed with mass $5\times 10^{5}~M_{\odot}/h$ are inserted into halos of mass $\gtrsim 5\times 10^{10}~M_{\odot}/h$. Once seeded, BH growth via gas accretion is assigned at a rate of $\dot{M}_{BH}={4\pi G^2 M_{BH}^2 \rho}/{(c_s^2+v_{BH}^2)^{3/2}}$ where $\rho$ and $c_s$ are the density and sound speed of the interstellar medium (ISM) gas at cold phase; $v_{BH}$ is the relative velocity between the BH and its surrounding gas. Note that unlike several previous works, the accretion rate in MBII adopt the prescription in~\citet{Pelupessy2007} which does not include any artificial boost factor. The accreted gas is released as radiation at a radiative efficiency of 10\%. A fraction of 5\% of the radiated energy \aklant{thermally} couples to the surrounding gas as black hole (or AGN) feedback~\citep{2005Natur.433..604D}. A mildly super-Eddington (two times Eddington rate) is allowed. Due to resolution limitations, BH dynamics cannot be self-consistently modeled in the simulations. Two BHs are considered to be merged when their separation distance is below the simulation spatial resolution (i.e., the SPH smoothing length) and their relative speeds are lower than the local sound speed of the medium.

%\ding{The galaxy photometry information is not used in this work (we only compare stellar mass this time). To avoid any confusing or misunderstanding, I remove the simulation SED information in Section 3.1, 3.2 and 3.3.}
%For the galaxy photometry, the initial SEDs of the BH host galaxies were determined by combining all the contributions from their constituent star particles. The stellar SEDs were derived using the PEGASE-2~\citep{1999astro.ph.12179F} stellar population synthesis code with a Salpeter IMF. Finally, the galaxy SEDs are  cooperated with the desired filter response to obtain the broadband photometry (e.g., SDSS $r$-band magnitude).

As a common practice, the stellar mass is obtained by using a 3D spherical aperture of 30~kpc to represent the observed stellar mass. \blue{We adopt this definition of stellar masses for all simulations described in the following sections.} Using this definition, \citet{Pillepich2018} has shown that the corresponding stellar mass function is consistent with the observational measure. Even more, the stellar mass using this 3D aperture can achieve good agreement to those measured within the Petrosian radii in observational studies~\citep{Schaye2015}. For further details of MBII simulation, we refer the reader to~\citet{Khandai2015}.



\subsection{Illustris}
The Illustris Project is another large scale hydrodynamics simulation, introduced in~\citet{2014MNRAS.444.1518V, 2014Natur.509..177V}. The simulation consist of a volume of (106.5 cMpc)$^3$~(slightly smaller than MassiveBlack II), 
%\yo{Here the box size is given in cMpc and in the table in cMpc/$h$ with the same numerical value.}
and was run with the moving Voronoi mesh code {\tt Arepo}~\citep{2010MNRAS.401..791S} with a base cosmology adopted from WMAP9 results~\citep{2013ApJS..208...19H}.
%\aklant{\bf{I think it'll be good to explicitly state all cosmological parameters for the simulations and SAMs}}. 
Besides gravity and gas hydrodynamics, the simulation calculates the astrophysical processes ~\citep{2013MNRAS.436.3031V, 2014MNRAS.438.1985T} that includes \aklant{gas cooling and star formation~(with a density threshold of 0.13 cm$^{-3}$, \citealt{2003MNRAS.339..289S})}, stellar evolution and chemical enrichment, kinetic stellar feedback by SNe activity, BH growth~(accretion and merging), and AGN feedback.

\aklant{BHs are seeded with an initial mass of $1 \times 10^5~M_{\odot}/h$ when a halo exceeds a mass of $5 \times 10^{10}~M_{\odot}/h$. BHs then grow via accretion described by the Eddington limited Bondi-Hoyle-Lyttleton formalism ($\alpha4\pi G^2M_{BH}^2 \rho/c_s^3$), as well as mergers with other BHs. The boost factor $\alpha=100$ is introduced to account for the unresolved multiphase ISM~\citep{Springel2005, 2009MNRAS.398...53B}, which is otherwise expected to underestimate the density around the BHs. Lastly, accreting black holes radiate with a bolometric luminosity given by $\epsilon_r \dot{M}_{BH}c^2$, where $\dot{M}_{BH}$ is the mass accretion rate and $\epsilon_r=0.2$ is the radiative efficiency.}

\aklant{The AGN feedback consists of three components, namely quasar-mode, radio-mode and radiative feedback. In the quasar-mode which holds for BHs with Eddington ratio $>0.01$, the AGNs deposit $5\%$~(quasar-mode feedback efficiency) of their released energy into the surrounding gas as thermal energy. For Eddington ratios $<0.01$, the AGN feedback is in radio-mode where the thermal energy is released as hot bubbles with a radius of $\sim$ 100~kpc at~(irregular) intervals between which the BH mass grows by a fixed fraction. The energy of the bubbles is given by $\epsilon_m \epsilon_r \delta M_{bh} c^2$ where $\delta M_{bh}$ is the change in BH mass within the last time interval, and $\epsilon_m=0.35$ is the radio-mode feedback efficiency. Lastly, the radiative feedback mode is implemented by modifying the heating and cooling rates of the gas in the presence of radiation from all surrounding AGN.} \blue{As in MBII, a 3D 30~kpc spherical aperture is used to obtained the galaxy stellar mass, assuming a \cite{2003PASP..115..763C} IMF.}

%It is highlighted in ~\citet{2014MNRAS.444.1518V} that the AGN feedback is the key process in quenching massive galaxy which is crucial to reproduce a tight stellar mass and BH mass relation. 

%\aklant{Finally, galaxy photometry was computed similar to that \texttt{MassiveBlackII}, but with SEDs derived using templates from \texttt{STARBURST99 (SB99)} catalogs~\citep{2010ApJS..189..309L} assuming a \cite{2003PASP..115..763C} IMF.}

\subsection{IllustrisTNG}
{\it The Next Generation Illustris Simulations} (IllustrisTNG)~\citep{2018MNRAS.475..676S, Pillepich2018} are 
%\yo{I am not sure why ``ambitious''}
%an ambitious 
an suite of \aklant{magneto}hydrodynamical simulations of galaxy formation in large cosmological volumes. It builds upon the scientific achievements of the Illustris simulation with improvements upon Illustris by 1) extending the mass range of the simulated galaxies and haloes, 2) adopting an improved numerical and astrophysical modeling, and 3) addressing the identified shortcomings of the previous generation simulations.

The TNG100 and TNG300 have a volume of (100~cMpc)$^3$ and (300~cMpc)$^3$, respectively. The adopted cosmological parameters are updates by the Planck result~\citep{2016A&A...594A..13P}
% \aklant{\bf{Here again, please list the cosmological parameters?}}.
\aklant{The gas cooling and star formation prescriptions are broadly similar to the Illustris model. However significant updates have been made to the stellar feedback model~(more details in \citealt{2018MNRAS.473.4077P}).
BH seeds with initial mass of $8 \times 10^5 M_{\odot}/h$ are placed in Dark matter halos with a mass exceeding $5 \times 10^{10} M_{\odot}/h$. Notably, the seed mass is one order of magnitude higher than in the Illustris simulation.} The BH accretion also follows the Bondi-Hoyle-Lyttleton formalism, but without any boost factor~(unlike Illustris). Accreting black holes release energy with a radiative efficiency of 0.2~(same as Illustris). The \aklant{inclusion} of the magnetic fields can affect the relationship between the BHs and their host galaxies properties; the \mbh-\smass\ mean relation is higher with magnetic fields~\citep{2018MNRAS.473.4077P}. %~\aklant{\bf{Aklant: Hmm I did not know this. Do they talk about what exactly causes this impact? Xuheng's reply: I adopt this sentence form the reference and Pillepich et al. It is mentioned in \citet{2018MNRAS.473.4077P} (in the texts after Figure 8) in the TNG model they modify the Bondi-Hoyle-Lyttleton formula for the gas accretion on to the BH in the presence of magnetic fields.}} %\sout{The TNG simulation also employs a kernel-weighted accretion rate over about 256 neighboring cells.} \sout{Thus, the TNG BH can build up correlation with the gas properties in the galaxy central region.}

\aklant{The AGN feedback occurs in thermal, radio, and radiative modes. For high accretion rates, the feedback implementation is the same as Illustris i.e., thermal energy is injected in the surroundings of the accreting BHs. However, at low accretion rates, the feedback implementation is substantially different from Illustris. Instead of releasing hot bubbles, this feedback mode in TNG is purely kinetic. In particular, there is a directional injection of momentum along a randomly chosen direction~\citep{2017MNRAS.465.3291W, 2018MNRAS.479.4056W} at irregular intervals.} The transition between the two feedback modes is also different from Illustris, and is set by the minimum value of 0.1 and $2 \times 10 ^{-3} \times (M_{BH} / 10^8~M_{\odot})$. \aklant{Additionally, the radiative feedback implemented in Illustris~(summarized in the previous section) is also present in TNG.} Lastly, \blue{the galaxy stellar mass} in Illustris-TNG is \blue{obtained} in a similar manner to that of Illustris.

\subsection{Horizon-AGN}\label{subsec:Horizon}

\blue{The simulation Horizon-AGN~\citep{2014MNRAS.444.1453D, 2016MNRAS.463.3948D} has a volume of 142 cMpc$^3$ and was generated using the adaptive mesh refinement code {\tt Ramses}~\citep{2002A&A...385..337T} with a $\Lambda$CDM model based on WMAP7~\citep{2011ApJS..192...18K} cosmological results. The dark matter particle mass is $8\times 10^7 M_{\odot}$. The stellar particle mass is $2\times 10^6 M_{\odot}$ and the MBH seed mass is $10^5 M_{\odot}$. Adaptive mesh refinement is permitted down to $\Delta x=1$~kpc, and, if the total mass in a cell becomes greater than 8 times the initial mass resolution, it is performed in a quasi-Lagrangian manner. Collisionless particles (dark matter and star particles) are evolved using a particle-mesh solver with a cloud-in-cell interpolation. }

\blue{The simulation includes gas cooling down to $10^4\, \rm K$ \citep{sutherland&dopita93}, and stochastic star formation with a constant star formation efficiency $\epsilon_*=0.02$, which occurs in regions where the gas number density exceeds the star formation threshold $n_0 = 0.1\,\rm H\, cm^{-3}$. A Salpeter IMF is assumed. Stellar feedback is modeled as mechanical energy injection from Type Ia SNe, Type II SNe and stellar winds, with the metal enrichment from these sources. }

\blue{Differing from simulations presented above, Horizon-AGN does not use a fixed threshold in the dark matter halo mass to seed BHs.  BHs are seeded with a mass of $10^5 M_\odot$ in cells, with gas density above $n_0$ and stellar velocity dispersion larger than $100 \,\rm km\,s^{-1}$. An exclusion radius is imposed so that no BH seed is formed at less than 50 ckpc from an existing BH. After $z = 1.5$, new BHs are prevented from forming. At these subsequent times, all the progenitors of the \smass$>10^{10} M_{\odot}$ galaxies at $z = 0$ should be formed and seeded with BHs~\citep{2016MNRAS.460.2979V}.  BH accretion is computed using the Bondi-Hoyle-Lyttleton formalism with a boost factor $\alpha = (\rho/\rho_0)^2$ when the density $\rho$ is higher than the resolution-dependent threshold $\rho_0$. Otherwise, the boost factor is fixed as unity~\citep{2009MNRAS.398...53B}.}

\blue{Horizon-AGN includes two modes of AGN feedback. In the quasar mode ($f_{\rm Edd}>0.01$), thermal energy is isotropically released within a sphere of radius a few resolution elements. The energy deposition rate is $\dot{E}_{\rm AGN} = 0.015 \dot{M}_{\rm BH} c^2$. In the radio mode, energy is injected into a bipolar  outflow  with  a  velocity  of  $10^4\,\rm km\,s^{-1}$, to  mimic the  formation  of  a  jet.  The  energy  rate  in  this  mode is $\dot{E}_{\rm AGN} = 0.1 \dot{M}_{\rm BH} c^2$.  The  technical  details  of  BH  formation,  growth  and AGN  feedback  modeling  of  Horizon-AGN  can be found in~\citet{2012MNRAS.420.2662D}.} 

\blue{We identify galaxies applying the AdaptaHOP structure finder \citep{Aubert+04,Tweed+09} to the star particle distribution.  Galaxies are identified using a local threshold of 178 times the average matter density, with the local density of individual particles calculated using the 20 nearest neighbours. Only galaxies with more than 50 particles are considered. The galaxy mass corresponds to the total stellar mass of a galaxy identified with this approach. The stellar mass function is shown in \cite{2017MNRAS.467.4739K} to be in good agreement with observations.}


%Horizon-AGN  employs  a  kinetic  SN  feedback,  including momentum, mechanical energy and metals from type~II, Type  Ia  SNe,  and  stellar  winds  (details  in  Kaviraj  et  al.2017).  The  feedback  is  modeled  as  kinetic  release  of  energy on timescale below 50 Myr, and a thermal energy after 50 Myr. The feedback is also pulsed, meaning that energy is accumulated until sufficient to propagate the blast wave to  at  least  two  cells.  The  energy  released  depends  on  theSSP modeled assumed and the metallicity of the gas, and is aboute SN $\sim10^{49}$ erg/$M_{\odot}$.





\begin{figure}
\centering
\includegraphics[height=0.35\textwidth]{OpeningAngle.pdf}
\caption{\label{fig:SAM} 
Total gas content of galaxies as a function of AGN bolometric luminosity and jet opening angle in a new AGN feedback model incorporated into the SAM simulation.
}
\end{figure} 

\begin{figure*}
\centering
\begin{tabular}{c c}
{\includegraphics[trim = 0mm 0mm 65mm 10mm, clip, height=0.45\textwidth]{HSC_selection_MBII.png}}
{\includegraphics[trim = 0mm 0mm 20mm 10mm, clip,height=0.45\textwidth]{HST_selection_MBII.png}}
\end{tabular}
%\caption{\label{fig:selection}Demonstration of the impact of AGN selection using MBII. {\it left}: Correlation between \mbh\ and %$L_{\rm bol}$ of the simulated HSC sample is used to set the selection window for the simulated sample. This region roughly bracket %the type-1 AGN sample. The light green background cloud shows the simulated number density distribution in this parameter space %which includes a random level of uncertainty to mimic observational measurement errors. {\it right}: Similar to the top plane -- %equivalent selection window adopted for the HST-observed and MBII simulated samples.
%}
\caption{\label{fig:selection}Demonstration of AGN selection using MBII. {\it left}: Distribution of \mbh\ and Eddington ratio for the full (colored squares) MBII sample and individual objects meeting the observed selection criteria (blue circles). A matched HSC sample is shown by the orange data points. The light green background cloud shows the {\it intrinsic} simulated number density in this parameter space. %which includes a random level of uncertainty to mimic observational measurement errors.
{\it right}: Similar to the panel on the left, this figure presents the impact of selection on the HST sample. For visual comparison between the HSC and HST selection, we show the region of the HST selection window in the left panel as dashed lines. 
}
\end{figure*}

\subsection{Semi-analytic Model (SAM)}\label{subsec:SAM}
We highlight the main points of the simulation with respect to our study; for more detail, a full description of the SAM can be found in~\citet{Menci2016} which is based on an earlier semi-analytic model introduced in~\citet{Menci2014}. \blue{The specific version adopted here differs from the one presented in the above papers since it implements a new, detailed description of AGN feedback, as discussed in detail below}.

For dark matter halos that merge with a larger halo, the impact of dynamical friction is assessed to define whether the halo will survive as a satellite or sink to the center of the dominant galaxy which increases its mass. The binary interactions (fly-bys and mergers), among satellite sub-halos, are  also described by the model. In each halo, we compute the fraction of gas which cools because of the atomic processes and settles into a disk~\citep{Mo1998}. The stars are converted from the gas through three channels: (1) quiescent star formation with long time scales: $\sim1$~Gyr; (2) starbursts following galaxy interactions with timescales $\lesssim 100$~Myr, 
%(CHECK THIS-JDS)\ding{It's mentioned in Menci 2016 section 2.} 
according to BH feeding; (3) the loss of angular momentum triggered by the internal disk instabilities causing the gas inflows to the center, resulting in stimulated star formation (as well as BH accretion). The stellar feedback is also considered by calculating the energy released by the supernovae associated with the total star formation which returns a fraction of the disk gas into a hot phase. A Salpeter IMF is adopted in the SAM simulation. \blue{A $\Lambda$CDM power spectrum of perturbations with a total matter density parameter $\Omega_0=0.3$, a baryon density parameter 
$\Omega_b=0.04$, a dark energy density parameter $\Omega_\Lambda=0.7$, and a Hubble constant $h$=0.7 is adopted. }

{We assume BH seed $M_{seed}=100\,M_{\odot}$~\citep{Madau2001}  to be initially present in all galaxy progenitors at the initial redshift $z=15$. This constitutes an approximate way of rendering the effect of the collapse of PopIII stars. However, the detailed value of $M_{seed}$ has a negligible impact on the final BH masses as long as they remain in the range $M_{seed}=50-500\,M_{\odot}$.
}

The BH accretion is based on both interaction-driven and disk instability feeding modes.
%\newline(1)~{\it triggered by interactions.} The interaction rate $\tau_r^{-1}=n_T\,\Sigma (r_t,v_c,V)\,V_{rel} (V)$ for galaxies with relative velocity $V_{rel}$ and number density $n_T$ in a common DM halo determines the probability for encounters, either fly-by or merging, through the corresponding cross sections $\Sigma$ given in~\citet{Menci2014}. The fraction of gas destabilized in each interaction corresponds to the loss $\Delta j$ of orbital angular momentum $j$, and depends on the mass ratio of the merging partners $M'/M$ and on the impact factor $b$.
%\newline(2)~{\it induced by disc instabilities.} We assume these to arise  in  galaxies with disc mass exceeding~\citep{Efstathiou1982} $M_{crit} =  {v_{max}^2 R_{d}/ G \epsilon}$ with $\epsilon=0.75$, where $v_{max}$ is the maximum circular velocity associated to each halo ~\citep{Mo1998}.  Such a criterion strongly suppresses the probability for disc instabilities to occur not only in massive, gas-poor galaxies, but also in dwarf galaxies characterized by small values of the gas-to-DM mass ratios. The instabilities induce loss of angular momentum resulting into strong inflows that we compute following the description in~\citet{Hopkins2011}, recast and extended as in~\citet{Menci2014}. 
%\todo{This part should be updated for the new SAM version? No, please leave as it is}
\blue{The SAM adopted here implements a new and improved  model for the AGN feedback with respect to the previous versions~\citep{Menci2008}. In both versions, the basic assumption is that fast winds with velocity up to $10^{-1}c$ observed in the central regions of AGNs~\citep{Chartas2002, Pounds2003}  result in  supersonic outflows that compress the gas into a blast wave terminated by a leading shock front. This  moves outward with a lower but still supersonic speed, and sweeps out the surrounding medium. However, while in the earlier version of the SAM~\citep{Menci2016} the blast wave is assumed to expand into an isotropically distributed medium, in the new description of AGN feedback~\citep{Menci2019} the full two-dimensional structure of the gas disc and of the expanding blast wave is followed in detail. The main physical difference is that in the new model the large density of gas along the plane of the disc causes the blast wave expansion to stall in such a direction, while it expands with large velocities in the vertical direction. The resulting strong dependence of the total (integrated over directions) outflow rate on the AGN luminosity $L_{AGN}$ and on the gas content of the galaxy $M_{gas}$ is shown in Figure~\ref{fig:SAM}. Such a new AGN feedback model has been tested in detail against a state-of-the-art compilation of observed outflows in 19 galaxies with different measured gas and dynamical masses~\citep{Fiore2017},
%(Fiore et al. 2019)\ding{Hi Nicola, I can't find the reference Fiore et al. 2019, I guess you mean~\citep{Fiore2017}, right? I will update the reference before submit. Let me know if I am wrong.}
allowing for a detailed, one-by-one comparison with the model predictions. Such a new, well tested AGN feedback model  allowed us to derive, for each simulated galaxy in the SAM,  the outflow expansion and the mass outflow rates in different directions with respect to the plane of the disc.}



\begin{figure*}
\centering
\includegraphics[trim = 50mm 70mm 40mm 90mm, clip,height=1.1\textwidth]{MM_sum.png}
\caption{\label{fig:comparsion} 
Black hole mass versus stellar mass for both the observational (small orange circles) and simulated (small colored circles) samples. Each row pertains to a particular simulation as labelled. The panels, from left to right, show different redshift bins. The black line in each panel indicates the local relation adopted by~\citet{Ding2020}. The background cloud (in green and yellow) shows the intrinsic simulation number density before injecting random noise and applying selection effects. Only TNG100 is presented here since TNG300 presents very similar results.
}
\end{figure*} 

\subsection{Application of observational measurement error and selection effects}

To make direct comparisons with observations, we add measurement errors and apply the equivalent selection to the simulated samples. We first inject random noise to the simulated catalog to mimic the scatter caused by measurement error. As mentioned above, \smass\ and \mbh\ for HSC and HST samples are measured with a similar approach; thus their uncertainty levels are expected to be equivalent. 
We assume the following measurement uncertainties that are added as random noise: $\Delta$\mbh$ = 0.4~$dex, $\Delta$\smass$ = 0.17~$dex, and $\Delta L_{\rm bol} = 0.03~$dex. 



We then apply restrictions on the noise-injected simulation to mimic selection effects as present in the observational data. Since the HSC and HST samples have their own selection function, we apply different selection criteria to the simulation as follows:
%\begin{itemize}

 HSC sample: (1) The observed sample consists of type-1 AGN, and thus the simulated sample should match the relationship between \mbh-$L_{\rm bol}$ as seen in the HSC sample. We use MBII to demonstrate the importance of matching the sample selection (Figure~\ref{fig:selection}--{\it top}). (2) The $i$-band magnitude of the AGN are bright (see Section~\ref{sec:hsc}). The specific selection is made as follows: for systems at $z<0.5$ and $z>0.5$, the AGN $i$-band magnitude is required to be brighter than 20.5 mag and 22.0 mag, respectively. Since the simulations do not provide the observed AGN magnitude, we adopt a simulated rest-frame magnitude or L$_{\rm 5100}$ and assume the quasar continuum as a single power-law with an index of $\alpha_\nu=-0.44$~\citep{2001AJ....122..549V} to calculate the observed $i$-band magnitude.
 3)~Following the HSC selection, we require the \smass\ value to be above a certain level (according to their redshift) to assure an accurate measurement. Finally, the HSC sample is split into three redshift bins for making comparison which are $0.2<z<0.4$, $0.4<z<0.6$, and $0.6<z<0.8$.
 
HST sample: Simulated AGN systems are selected only when they match the  \mbh-$L_{\rm bol}$ targeting window which is the same as the observational selection (see Figure~\ref{fig:selection}--{\it right} using MBII as an example). Note that the selection of the HST sample has a hard cut on the \mbh\ values (i.e., between [7.7, 8.6] $M_{\odot}$). The HST sample covers the higher redshift range $1.2<z<1.7$, which is considered as a single redshift bin to make the comparison with the simulations at $z=1.5$.
%\end{itemize}



\section{Results} \label{sec:result}
In Figure~\ref{fig:comparsion}, we present the mass scaling relation \mbh--\smass\ for both the observations and simulations for direct comparison. The local scaling relation adopted by D20 (e.g., \mbh$=0.98$\smass$-2.56$, Chabrier IMF\footnote{Since different simulations adopt either a Chabrier or a Salpeter IMF, we use the local relation and \smass\ of the observational data that are based on the same IMF thus a comparison between the observations and simulations are self-consistent.}) is used as the fiducial relation to assess relative offsets and differences in dispersion. A different simulation is presented in each row and the redshift intervals increase from left to right. We note that TNG100 and TNG300 yield very similar comparison results, thus only TNG100 is presented in Figure~\ref{fig:comparsion}.

For each sample, the central offset and scatter of the scaling relation are estimated by calculating the mean and the standard deviation of the \mbh\ residuals for each system\footnote{The value of the slope for the local sample is close to 1, and thus if taking the \smass\ to calculate the residual for each system, the offset value remains the same.} (i.e., the offset to the local relation along the y-axis in Figure~\ref{fig:comparsion}).  To aid in visualization of the differences among the various simulations compared to the observed sample, we show the distribution of offsets (in terms of the $\Delta{\rm log}$\mbh) as histograms in Figure~\ref{fig:offsets}. Each panel presents a different redshift range. In addition, the values for the central offset and scatter in each case are given in Table~\ref{tab:sum} and shown as a function of redshift in Figure~\ref{fig:offsets_vz}.



\begin{figure*}
\centering
\begin{tabular}{c c c c}
\hspace*{-0.4cm} 
{\includegraphics[height=0.4\textwidth]{offset_dis_z03.pdf}}&
\hspace*{-0.4cm} 
{\includegraphics[height=0.4\textwidth]{offset_dis_z05.pdf}}&
\hspace*{-0.4cm} 
{\includegraphics[height=0.4\textwidth]{offset_dis_z07.pdf}}&
\hspace*{-0.4cm} 
{\includegraphics[height=0.4\textwidth]{offset_dis_z15.pdf}}\\
\end{tabular}
\caption{\label{fig:offsets} 
Histograms of the offset distributions for all simulation samples and observations. The mean value and the standard derivation of the histogram are summarized in Table~\ref{tab:sum}. The vertical dashed lines show the corresponding mean value for each distribution. The mean values for observed sample (i.e., yellow lines) are also show in each simulation plots.
For the MBII simulation, the sample at redshift 0.6 is used to compare with other samples at $z=0.5$ and $z=0.7$.
}
\end{figure*} 

\begin{figure}
\centering
%\begin{tabular}{c c}
%{
\includegraphics[height=0.4\textwidth]{offset_summary_vz.pdf}%}&
%{\includegraphics[height=0.4\textwidth]{offset_int_summary_vz.pdf}}\\
%\end{tabular}
\caption{\label{fig:offsets_vz} 
%{\it left}: 
The {\it observed} evolution of $\Delta{\rm log}$\mbh\ as a function of redshift using both observation and simulation data. The black line shows the evolution by fitting the offset as a function of redshift. The predictions from the numerical simulations, given in Table~\ref{tab:sum}, are presented by different colored symbols. The grey horizontal band illustrates the level of dispersion for the local sample.
%\ding{The right panel can be confusing and hard to explain. Maybe we can remove?}
}
\end{figure} 

\subsection{Dispersion}

Our results show that almost all simulations can produce scatter which is consistent with the observations across all redshifts examined (Figures~\ref{fig:comparsion}~and~\ref{fig:offsets}) --- for the simulated samples at $z<1$, this level of scatter is $\sim0.5$~dex, while at $z>1$, it is $\sim0.3$~dex. Note that the HST sample $z>1$ has a narrow selection window based on \mbh\ (see Figure~\ref{fig:selection}~bottom), causing the observed scatter to be smaller than that of the HSC sample at $z<1$. At all redshifts, we recognize that the observed scatter is dominated by measurement uncertainties in the data. 

An understanding of how much of the scatter derives from random noise can help us to determine the {\it intrinsic} scatter in the scaling relation. To this end, we measure the scatter of the simulation sample without injecting random noise but adopting the same selection window for both $z<1$ and $z>1$ samples to infer the central offset and scatter. We find that the intrinsic scatter is at a level of $\sim0.15-0.2$ dex for both $z<1$ and $z>1$ (see Table~\ref{tab:sum_no_noise}). These levels are consistent with the {\it intrinsic} scatter as estimated using observation data alone~\citep{Ding2020, 2021arXiv210902751L}. Furthermore, the intrinsic scatter appears to be independent of redshift since the observations and simulations all follow the observed trend with redshift expected to be due to selection effects (Figure~\ref{fig:offsets_vz}). This suggests that the tight scaling relation may not be the result of a pure stochastic process, i.e., random mergers. However, the scatter is affected by sample selection, and thus these levels can only be taken as an approximation of the true intrinsic scatter.

\subsection{Global offsets}

We examine the offsets to understand whether the simulations deviate or not from the {\it observed} scaling relation with particular attention to changes with redshift. Considering the values given in Table~\ref{tab:sum} and shown in Figure~\ref{fig:offsets_vz}, over the lower redshift range $z<0.6$, Illustris and Horizon-AGN predict {\it observed} \mbh\ offsets consistent with the observation data (at a level of $\lesssim0.1$~dex). At higher redshift $0.6<z<1.5$, the simulations SAM, TNG100, TNG300 and Horizon-AGN follow the {\it observed} evolution. These results are consistent with the Kolmogorov-Smirnov (KS) test performed using the offset distributions between each simulated sample and the observed sample --- the p-values are given in Table~\ref{tab:pvalue} showing that Horizon-AGN and Illustris have a good statistical match to the scaling relation at $z<0.6$ (i.e., p-value $> 0.1$), while the TNG100, TNG300 and Horizon-AGN simulation do well at $z>0.6$. Overall, we find that the mass ratios between SMBHs and their host galaxies are generally consistent between observations and simulations with some subtle differences which are not at the level of concern for this present study.

\blue{
\subsection{Trends with stellar mass}
In Figure~\ref{fig:deltaMM}, we investigate how the offset values are correlated with stellar mass. Here, we focus on the sample at $z\sim0.7$. The other redshift bins at $z<1$, where there is a large observation sample from HSC, show similar trends. We include the intrinsic values from the simulations in the figures to address how the observational effects (i.e., random noise and selection) change the observed scaling relations and offsets. First, considering the observed quasar sample (same in each panel), there is a trend for which black holes have masses further offset from their stellar mass with decreasing stellar mass. This trend is not seen in any of the simulations after noise and selection effects have been applied. Given the level of uncertainties in the mean offsets of the observed sample, we do not try to interpret this trend any further in this study.}   

\blue{
Interestingly, we notice that MBII and Illustris have black holes intrinsically under-massive relative to their galaxies at the lower masses that reach the local relation at higher masses. In contrast, TNG and Horizon-AGN have black holes slightly elevated from the local relation at most masses. These differences between simulations present two different scenarios, either one where the black holes come later or co-evolution with the two growing in tandem. Considering the former scenario, Illustris show the strongest trend with stellar mass. In fact, after noise and selection is applied, the simulated sample exhibits very small offsets which agree remarkably well with the observed data. {\bf TT edits: This result underscores the importance of taking into account errors and selection --- without accounting for those, one could erroneously interpret an apparent trend as evolution in the opposite sense as the true one.  The most direct way to circumvent these issues is to probe lower masses (\smass\  $<10^{10}M_{\odot}$) using a more sensitive instrument, such as JWST~\citep{Habouzit2022} across this redshift range \citep[see also][]{2011MNRAS.417.2085V}.}}
{\bf TT: the following sentences seem out of place} \ding{OK, let's remove the following red sentences can be removed}
\red{In the local universe, a delayed black hole growth scenario seems to be supported \citep{Reines2015}, %\yo{
which is understood to be the result of the quenching of BH growth by efficient feedback from SNe in low-mass galaxies~\citep[e.g.][]{dubois_snreg_2015,habouzit_blossom_2017,bower_dark_2017,angles_black_2017}.}


%the intrinsic distribution actually shows an opposite trend in which the host galaxy is more massive at low mass end. While the TNG100 and Horizon-AGN also present good comparisons in the {\it observed} plane, they predict the different intrinsic trend unlike Illustris. Yet, limited by the stellar mass range of our observed data, our comparison result could not rule out any simulation at this moment. To this end, extending the study to lower mass ($<10^{10}M_{\odot}$) using more sensitive instrument, such as JWST, will be the key~\citep{Habouzit2022}. 


%For instance, if overlook the observational effect, the HSC observational measurement show that, when host stellar mass is low, the SMBH trend to be more massive taking local scaling relation as reference. This could mislead to a wrong conclusion that SMBH grow up first and the host galaxy catch up later to match up with the scaling relation in local universe. 

%In fact, taking the Illustris as example who well predicts the {\it observed} scaling relation to the HSC observation, 


\begin{figure*}
\centering
\begin{tabular}{c c}
\hspace*{-0.5cm} 
{\includegraphics[trim = 0mm 0mm 0mm 0mm, clip,
height=0.4\textwidth]{DeltaMM_MBII_zs_06.png}}&
\hspace*{-0.3cm} 
{\includegraphics[trim = 36mm 0mm 0mm 0mm, clip,
height=0.4\textwidth]{DeltaMM_Illustris_zs_07.png}}\\
\hspace*{-0.5cm} 
{\includegraphics[trim = 0mm 0mm 0mm 0mm, clip,
height=0.4\textwidth]{DeltaMM_TNG100_zs_07.png}}&
\hspace*{-0.3cm} 
{\includegraphics[trim = 36mm 0mm 0mm 0mm, clip,
height=0.4\textwidth]{DeltaMM_Horizon_zs_07.png}}\\
\end{tabular}
\caption{\label{fig:deltaMM} Comparison of the offset of the \mbh\ (to the local relation) as a function of stellar mass from observation data and the simulations at $z\sim0.7$. In each stellar mass bin, we give the mean and standard derivation of the offset values. The histograms on the right indicate the offset distribution with lines marking the mean offsets for observation and simulation. The green color distributions show the intrinsic simulated sample without random noise and selection applied.
%Since the selection of HST sample has a hard cut on the \mbh\ values, an {\it observed} trend of offset value as a function of stellar mass appears; this trend doesn't appear on the intrinsic ones. 
%{\bf[MV: If I understand correctly, mean and standard deviation of the offset values are shown only for the simulation once you have included measurement errors and selection effects. If this is the case, can you add them also for the intrinsic simulated sample?]}
}
\end{figure*} 


\section{Conclusions} \label{sec:con}
%A brief introduce of the result. Scatter, central, which is best...
We compared the observed scaling relation \mbh-\smass\ with the predictions from numerical simulations. The observation data are composed of 626 quasars at $0.2 < z < 0.8$ imaged by HSC and 32 X-ray-selected quasars at $1.2 < z < 1.7$ imaged by HST. The simulations include a semi-analytic model (SAM) and five hydrodynamic simulations, i.e., MBII, Illustris, TNG100, TNG300 and Horizon-AGN. We carried out the comparisons in the observed pararameter space to account for uncertainties and selection effects. To achieve this, we first injected random errors with the same observational uncertainty to the simulation, and then adopted the same selection condition to the simulation data (see Figure~\ref{fig:selection}). Finally, we adopted the scaling relation from the local universe as our reference and performed comparisons using the scatter of the measurements and their central offset to the local relation. Our main results are summarized as follows:

\begin{enumerate}

\item{}The {\it observed} scatter predicted by the simulations is consistent with the observational measurements, i.e., $\sim0.5$~dex at $z<1$ and $\sim0.3$~dex at $z>1$ (see Figure~\ref{fig:offsets_vz} and Table~\ref{tab:sum}). This result indicates that the simulated and observed samples have consistent {\it intrinsic} scatter.

\item{}To understand how much the {\it observed} scatter is dominated by random observational error,
we re-run the estimation without injecting noise to the simulations. The obtained scatter for both $z<1$ and $z>1$ are at a similar level (i.e.,  $\sim$0.15$-$0.2 dex, see Table~\ref{tab:sum_no_noise}), indicating that observational errors dominate the scatter.

\item{} Regarding the mass ratios and offsets from the local relation ($\Delta$\mbh\ at a given \smass\ ), all simulations generally match the observations with some subtle, yet notable, differences. While Illustris and Horizon-AGN show good correspondence with observations at $z<0.6$, the comparisons at $z>0.6$ are better for SAM, TNG100, TNG300 and Horizon-AGN. From $z\sim0.7$ to $z\sim1.5$, TNG100, TNG300 and Horizon-AGN simulations match well the observed evolution of the scaling relation, i.e., the offsets are larger at higher redshift as shown in Figure~\ref{fig:offsets_vz} and Table~\ref{tab:sum}. 
%\ding{In the simulation, is there any observational data that be used to give extra constrain on the recipe/assumption? Xuheng and John are surprised to see that all of them have very good scatter matches. Maybe we need provide some details on how the scaling relations in the simulations are calibrated. For example, why Illustris and TNG sample predict different center for the scaling relation. Horizon-AGN could predict well at both low and high z, maybe we can add some discussions.}
%{\bf[MV: This is very interesting. When comparing with BH-galaxy correlations at $z=0$ we noticed in \cite{2016MNRAS.460.2979V} that the simulations showed less scatter wrt the observations, and we thought that this meant that some stochastic processes on small scale were not captured. Now I wonder whether instead the lower scatter in the simulated results was because we did not apply any observational measurement error or selection effects to the simulation. You note in the Conclusions that this may not be important, but also that, indeed, no strong efforts have been made to include such effects.]}
\end{enumerate}

The absolute value of stellar masses in both observation and theory have significant uncertainty (up to factor of two), which depends on the assumption of initial mass function, and possibly on the implementation of star formation in the models. In contrast, the scatter around the mean correlation is a relative quantity, which is less affected by such systematic effect. Thus, in this work, we first consider the scatter as a diagnostic criteria to see whether some simulations match the data better than others. Taking (1) and (2), our results suggest that the tightness of the scaling relations have been formed since redshift 1.7, which is in contrast with the scenario of the central limit theorem~\citep{Peng2007, Jahnke2011, Hirschmann2010} that the scaling relation is a consequence of a stochastic cloud in the early universe with subsequent random mergers thereafter. In this stochastic scenario we expect the scatter of the scaling relations to increase towards higher redshift. In fact, the scatter level in the simulation without adding random noise is consistent with the {\it intrinsic} scatter estimations reported in~\citet{Ding2020, 2021arXiv210902751L} (i.e., $\lesssim0.35$~dex). This level is also not larger than the typical scatter of the local relations reported in the literature~\citep{Kormendy13, Gul++09, Reines2015}.

The simulations studied in this work have adopted completely different numerical techniques. However, all of them can provide good agreement with the observed dispersion in the scaling relation. In fact, the tightness of the scaling relation is stem from the same physics assumed in these simulations (i.e., AGN feedback). Thus, the consistency of the scatter between simulation and observations is consistent with the hypothesis that the AGN feedback as a causal link between SMBHs and their hosts plays a key role in establishing the scaling relation. 

%So far, a control experiment has not been performed yet in which one numerical model provide the 
\blue{
%hydrodynamical} simulation altering the AGN feedback prescription, while fixing all other conditions.
We can gain more insight into the role of feedback by looking at the SAM model, for which multiple feedback models have been implemented. \citet{Ding2020b} compared the scaling relations obtained with the same HST sample and the SAM simulation but with a different, isotropic, AGN feedback model, and found a larger scatter ($\sim0.7$~dex) with respect to the present SAM version ($\sim0.36$~dex). We ascribed the change to the following reasons: in the new 2D model for feedback, the wave expansions stalls along the direction of the disc, and the radius where the expansion stops depends strongly on both the gas density of the disk and the AGN luminosity. 
This means that the opening angle (and hence the fraction of expelled gas) is larger when the gas density is small (because of the lower energy that has to be spent to push the gas outwards) and when the AGN luminosity is large (because of the larger energy available to push the blast wave outwards). These dependencies are summarized in Figure~\ref{fig:SAM}. 
Both quantities depend on the merging histories and are  related, since the AGN luminosity $L_{\rm AGN}$ depends on the available cold gas reservoir $M_{\rm gas}$.
The large efficiency of feedback in galaxies with particularly small $M_{\rm gas}$ (for given $L_{\rm AGN}$) or in those with particularly large $L_{\rm AGN}$
(for given $M_{\rm gas}$) inhibits the BH growth in all the host galaxies that are outliers with respect to the average relation between $M_{\rm gas}$ and $L_{\rm AGN}$. 
This results into a smaller scatter.}

\blue{
In theoretical models, AGN feedback is often assumed to consist of two distinct modes:
%In theory, it is a common knowledge that two different modes are responsible for AGN feedback: 
a quasar-heating mode where the SMBH accretion rates are comparable to the Eddington rate and a radio-jet mode occurring at low accretion rates (see e.g. Section~\ref{subsec:Horizon}). In high redshift universe, the cold material in early universe leads to the vigorous accretion to the SMBH which drives the high accretion rates, and thus the quasar mode dominates the feedback. At low redshift, the star formation and feedback ejection reduce the cold material leading to a lower accretion rate and a radio-mode-dominating feedback~\citep[e.g.][]{2012MNRAS.420.2662D,2016MNRAS.460.2979V,2018MNRAS.479.4056W}. Our comparison result shows that the level of intrinsic scatter in the scaling relation at redshifts up to 1.7 is consistent with the low redshift one (see Table~\ref{tab:sum_no_noise}), which reveals the fact that at high redshift the AGN feedback described by the quasar mode has already regulated the tight correlation between SMBH and its host galaxy. After that, the radio-jet mode starts to take control at low redshift by maintaining the tightness of the scaling relation till the level we observe today.
}

Our work highlights the importance of applying measurement uncertainty and the effect of selection to the simulated data in order to make direct comparisons with observations. Such comparisons have been made in the local universe~\citep[e.g.,][]{Habouzit2021} where the measurements are relatively robust and the selection function is broad thus it is less crucial to ensure consistency between observations and simulations. However, beyond $z>0.2$, the scatter and the central distribution of the scaling relations are dominated by measurement uncertainty and selection effects (see Figures~\ref{fig:comparsion} and~\ref{fig:deltaMM}) and a forward modeling in the observational plane becomes essential. For example, from trends seen with stellar mass in Illustris and MBII (Figure~\ref{fig:deltaMM}), selection effects may hamper our understanding of whether BHs and their hosts co-evolve or not.

Extending this study to even higher redshift (and lower mass galaxies, \smass\ $<10^{10}M_{\odot}$) will be very beneficial, probing closer to the epoch of formation of massive galaxies and SMBHs. The understanding of how and when the tight scaling relation emerged are crucial to test theoretical models \citep{Volonteri2021}. On the observational side, the James Webb Space Telescope will provide high-quality imaging data of AGNs at redshift up to $z\sim7$ (i.e., JWST cycle 1 program GO-1967 and GO-1727). These upcoming measurements will represent stringent tests on the proposed physical mechanisms for the initial formation of supermassive black holes and of their subsequent evolution with galaxies.

\begin{acknowledgments}
This work was supported by World Premier International Research Center Initiative (WPI), MEXT, Japan. 
The authors fully appreciate input from Jingjing Shi.

Based in part on observations made with the NASA/ESA Hubble Space Telescope, obtained at the Space Telescope Science Institute, which is operated by the Association of Universities for Research in Astronomy, Inc., under NASA contract NAS 5-26555. These observations are associated with programs \#15115. Support for this work was provided by NASA through grant number HST-GO-15115 from the Space Telescope Science Institute, which is operated by AURA, Inc., under NASA contract NAS 5-26555. 
JS is supported by JSPS KAKENHI Grant Number JP18H01251 and the World Premier International Research Center Initiative (WPI), MEXT, Japan.
TT acknowledges support by the Packard Foundation through a Packard Research fellowship to TT. LB acknowledges support from NSF award AST-1909933 and Cottrell Scholar Award \#27553 from the Research Corporation for Science Advancement.


\end{acknowledgments}


\begin{deluxetable*}{lccccc}
%\tablenum{1}
\tablecaption{Summary of the central offsets and scatters\label{tab:sum}}
\tablewidth{0pt}
\tablehead{
\colhead{Simulation} &  \multicolumn{3}{c}{HSC comparison (offset, scatter)} & \colhead{HST comparison (offset, scatter)} \\
\cline{2-4} 
%  \cline{2-4}  \cline{5} \\
\colhead{} &  \colhead{$0.2<z<0.4$} & \colhead{$0.4<z<0.6$} & \colhead{$0.6<z<0.8$} & \colhead{$1.2<z<1.7$} & IMF
}
\startdata
Observation & (0.12, 0.51) & (0.20, 0.50)  & (0.21, 0.56)  & (0.43, 0.31) & \\
SAM & (0.73, 0.49) & (0.65, 0.46)  & (0.51, 0.45)  & (0.51, 0.36) & Salpeter \\
MBII & (-0.15, 0.48) & \multicolumn{2}{c}{(-0.16, 0.48) [$z=0.6$]}  & (0.14, 0.31) & Salpeter\\
Illustris & (0.01, 0.52) & (0.08, 0.53)  & (0.06, 0.54)  & (0.07, 0.32) &  Chabrier \\
TNG100 & (0.27, 0.48) & (0.24, 0.46)  & (0.24, 0.45)  & (0.38, 0.33) & Chabrier \\
TNG300 & (0.26, 0.48) & (0.20, 0.48)  & (0.17, 0.48)  & (0.41, 0.34) & Chabrier \\
Horizon-AGN & (0.16, 0.49) & (0.14, 0.47)  & (0.23, 0.47)  & (0.47, 0.35) & Salpeter\\
\enddata
\tablecomments{This table collects the comparison results of the \smass-\mbh\ correlations between different simulation at different redshift. The value shows the central position offset to the local relation and the scatters measured around the local relation after applying the offset. A positive offset means the \mbh\ value predicted by the simulation is higher than the local relationship measurement at fixed \smass\ value. The last column shows the corresponding IMF that adopted to the local anchor to make a fair comparison with the observation. Note that for the observational data, the relative differences between local and high-$z$ measurements are not affected by the IMF assumptions.
For the MBII sample, the simulation does not produce the sample at $z=0.5$ or $z=0.7$, but rather at $z=0.6$. We use Monte Carlo approach to infer the uncertainties of the values in the table, finding that the uncertainties are within $\pm 0.03$.
}
\end{deluxetable*}





\begin{deluxetable*}{lcccc}
%\tablenum{1}
\tablecaption{Summary of the central offsets and scatters without noise\label{tab:sum_no_noise}}
\tablewidth{0pt}
\tablehead{
\colhead{Simulation} &  \multicolumn{3}{c}{HSC comparison (offset, scatter)} & \colhead{HST comparison (offset, scatter)} \\
\cline{2-4} 
%  \cline{2-4}  \cline{5} \\
\colhead{} &  \colhead{$0.2<z<0.4$} & \colhead{$0.4<z<0.6$} & \colhead{$0.6<z<0.8$} & \colhead{$1.2<z<1.7$}
}
\startdata
SAM & (0.72, 0.20) & (0.64, 0.18) & (0.56, 0.16)  & (0.08, 0.18) \\
MBII & (-0.15, 0.21) & \multicolumn{2}{c}{(-0.15, 0.22) [$z=0.6$]}  & (0.08, 0.19)\\
Illustris & (0.03, 0.32) & (0.10, 0.36) & (0.08, 0.36)  & (0.04, 0.19) \\
TNG100 &  (0.27, 0.20) & (0.27, 0.15) & (0.26, 0.16)  & (0.36, 0.15) \\
TNG300 & (0.25, 0.21) & (0.19, 0.23) & (0.20, 0.22)  & (0.32, 0.16) \\
Horizon-AGN & (0.19, 0.26) & (0.18, 0.21) & (0.28, 0.20)  & (0.37, 0.13) \\
%Previous are ones not adding noise but same selection, belows are not add noise and not select
%SAM & (0.75, 0.26) & (0.74, 0.26) & (0.94, 0.26)  & (0.59, 0.24) \\
%MBII & (-0.24, 0.22) & \multicolumn{2}{c}{(-0.26, 0.22) [$z=0.6$]}  & (-0.29, 0.22)\\
%Illustris & (-0.78, 0.40) & (-0.73, 0.40) & (-0.69, 0.40)  & (-0.55, 0.36) \\
%TNG100 &  (0.33, 0.31) & (0.31, 0.32) & (0.29, 0.33)  & (0.22, 0.37) \\
%TNG300 & (0.42, 0.58) & (0.40, 0.57) & (0.39, 0.56)  & (0.35, 0.54) \\
%Horizon & (0.12, 0.30) & (0.13, 0.30) & (0.14, 0.28)  & (-0.00, 0.48) \\
\enddata
\tablecomments{Same as Table~\ref{tab:sum} but without inject noise to the simulation data. Note that same selection window is still adopted to infer the values.}
\end{deluxetable*}


\begin{deluxetable*}{lcccc}
%\tablenum{1}
\tablecaption{Summary of the p-value using KS test \label{tab:pvalue}}
\tablewidth{0pt}
\tablehead{
\colhead{Simulation} &  \multicolumn{3}{c}{HSC comparison} & \colhead{HST comparison} \\
\cline{2-4} 
%  \cline{2-4}  \cline{5} \\
\colhead{} &  \colhead{$z\sim0.3$} & \colhead{$z\sim0.5$} & \colhead{$z\sim0.7$} & \colhead{$z\sim1.5$}
}
%\decimalcolnumbers
\startdata
SAM &  3.119926e-09 & $<$1e-10  & $<$1e-10  & 7.456907e-03  \\
MBII & 4.228845e-02 & \multicolumn{2}{c}{$<$1e-10 [$z=0.6$]}  & 2.218653e-06  \\
Illustris & 2.306434e-01 & 6.144066e-02  & 9.963678e-03  & 2.330195e-07  \\
TNG100 & 1.207389e-02 & 2.530020e-01  & 3.756821e-01  & 4.708287e-01  \\
TNG300 & 1.817531e-02 & 9.192506e-01  & 1.208812e-01  & 2.118038e-01  \\
Horizon-AGN & 2.699897e-01 & 2.865927e-01  & 6.129553e-01  & 1.953735e-01  \\
\enddata
\tablecomments{These p-values are obtained by the KS test between the simulation and the observation.}
\end{deluxetable*}


\vspace{5mm}
%\facilities{HST, HSC}

%% Similar to \facility{}, there is the optional \software command to allow 
%% authors a place to specify which programs were used during the creation of 
%% the manuscript. Authors should list each code and include either a
%% citation or url to the code inside ()s when available.

%\software{astropy \citep{2013A&A...558A..33A,2018AJ....156..123A},  }

%% Appendix material should be preceded with a single \appendix command.
%% There should be a \section command for each appendix. Mark appendix
%% subsections with the same markup you use in the main body of the paper.

%% Each Appendix (indicated with \section) will be lettered A, B, C, etc.
%% The equation counter will reset when it encounters the \appendix
%% command and will number appendix equations (A1), (A2), etc. The
%% Figure and Table counter will not reset.

%\appendix



%\section{Comparison of the observed mass function}\label{app:function}
%Our comparison also provides the opportunity to compare the {\it observed} mass distribution of the BH mass and the stellar mass. The histogram distribution of the {\it observed} \mbh\ and \smass\ are shown in Figure~\ref{fig:MB_fun} and Figure~\ref{fig:Mstar_fun}, respectively. All the simulations have good agreements to the observed mass function, in terms of the standard derivation. However, focus on the central point of the distribution, the TNG100 and TNG300 give the best prediction, indicating that their parent sample are most similar to the HSC sample. \todo{Xuheng will talk with Jingjing and add more discussion to this part.}

%\begin{figure*}
%\centering
%\begin{tabular}{c c}
%%\vspace*{-0.2cm} 
%{\includegraphics[height=0.35\textwidth]{MBH_dis_z03.pdf}}&
%%\hspace*{-0.4cm} 
%{\includegraphics[height=0.35\textwidth]{MBH_dis_z05.pdf}}\\
%%\hspace*{-0.4cm} 
%{\includegraphics[height=0.35\textwidth]{MBH_dis_z07.pdf}}&
%%\hspace*{-0.4cm} 
%{\includegraphics[height=0.35\textwidth]{MBH_dis_z15.pdf}}\\
%\end{tabular}
%\caption{\label{fig:MB_fun} 
%Illustration of the histogram of the observed \mbh\ function. For MBII simulation, the sample at redshift 0.6 is used to compare with other samples at $z=0.5$ and $z=0.7$. Note that for sample at $z\sim1.5$, the selection cut on \mbh\ is used in the simulation, thus the distribution is almost flat. \todo{update hist format as Figure 4}
%}
%\end{figure*} 
%
%\begin{figure*}
%\centering
%\begin{tabular}{c c}
%%\vspace*{-0.2cm} 
%{\includegraphics[height=0.35\textwidth]{Mstar_dis_z03.pdf}}&
%%\hspace*{-0.4cm} 
%{\includegraphics[height=0.35\textwidth]{Mstar_dis_z05.pdf}}\\
%%\hspace*{-0.4cm} 
%{\includegraphics[height=0.35\textwidth]{Mstar_dis_z07.pdf}}&
%%\hspace*{-0.4cm} 
%{\includegraphics[height=0.35\textwidth]{Mstar_dis_z15.pdf}}\\
%\end{tabular}
%\caption{\label{fig:Mstar_fun} 
%Illustration of the histogram of the observed MBH function. For MBII simulation, the sample at redshift 0.6 is used to compare with other samples at $z=0.5$ and $z=0.7$.
%}
%\end{figure*} 
\bibliography{reference}{}
\bibliographystyle{aasjournal}

%% This command is needed to show the entire author+affiliation list when
%% the collaboration and author truncation commands are used.  It has to
%% go at the end of the manuscript.
%\allauthors

%% Include this line if you are using the \added, \replaced, \deleted
%% commands to see a summary list of all changes at the end of the article.
%\listofchanges

\end{document}

% End of file `sample631.tex'.

\bibliographystyle{aasjournal}

\end{document}

