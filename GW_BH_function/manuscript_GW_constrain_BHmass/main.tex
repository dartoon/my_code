\documentclass[twocolumn]{aastex62}

\newcommand{\vdag}{(v)^\dagger}
\newcommand\aastex{AAS\TeX}
\newcommand\latex{La\TeX}
\newcommand\s{$\sim$}
\newcommand{\e}{\'{e}}
\newcommand{\amen}[1]{\textbf{\textit{#1}}}

\usepackage{amsmath}
\usepackage{xcolor}
\usepackage{mathrsfs}

\usepackage{float}
\usepackage{multirow}
\usepackage{enumitem} 

\shorttitle{Black hole mass function and redshift evolution by ET}
\shortauthors{Ding et al.}

\begin{document}

\newcommand{\mbh}{$\mathcal M_{\rm BH}$}
\newcommand{\cmass}{${\cal M}_0$}
\newcommand{\dl}{$d_L$}
\newcommand{\mone}{$m_1$}
\newcommand{\mtwo}{$m_2$}
\newcommand{\snr}{$\rho$}

\title{The first prediction of Black hole mass function and its evolution by ET}

\correspondingauthor{Xuheng Ding}
\email{dingxh@whu.edu.cn}

\author%[0000-0001-8917-2148]
{Xuheng Ding}
\affiliation{School of Physics and Technology, Wuhan University, Wuhan 430072, China}
\affiliation{Department of Physics and Astronomy, University of California, Los Angeles, CA, 90095-1547, USA}

%\author{Lilan Yang}
%\affiliation{School of Physics and Technology, Wuhan University, Wuhan 430072, China}
%\affiliation{Department of Physics and Astronomy, University of California, Los Angeles, CA, 90095-1547, USA}

\author%[0000-0003-1308-7304]
{Marek Biesiada}
\affiliation{Department of Astronomy, Beijing Normal University, Beijing 100875, China}
\affiliation{Department of Astrophysics and Cosmology, Institute of Physics, University of Silesia, 75 Pu{\l}ku Piechoty 1, 41-500 Chorz{\'o}w,
Poland}

\author%[0000-0002-4359-5994]
{Kai Liao}
\affiliation{School of Science, Wuhan University of Technology, Wuhan 430070, China}

\author%[0000-0002-3567-6743]
{Zong-Hong Zhu}
\affiliation{School of Physics and Technology, Wuhan University, Wuhan 430072, China}
\affiliation{Department of Astronomy, Beijing Normal University, Beijing 100875, China}


\begin{abstract}
%discover of GW
%Measure of BH mass and thus fit the BH mass.
%The discovery of double compact objects by GW provide a the measurement of their BH masses. Recent works by LIGO group have constrained 10 GW event and constrain the BHs. The ET would detect more GW events with DL and BH mass up to redshift $z\sim10$, and thus enable one to constrain the mass function and study how it evolution with redshift. In this work, test the ability of 1000 GWs events by ET. We tested, 1. how well are they able to recover the BH mass function as power law distribution. 2. If the GW events are able to tell a redshifted-related BH mass function.
We demonstrate the ability of future gravitational-waves (GW) measured by third generation detector Einstein Telescope (ET) to infer the slope of the BH mass function (BHMF) and its evolution with redshift. %The measurement of BH binary systems enable the inference of the BH mass (\mone\ and \mtwo\ of each component) together with the luminosity distance (\dl).
We performed the Monte Carlo approach to simulate the measurements of binary BH (BBH) inspiral systems signals as detected by ET, including the BH masses and their luminosity distances (\dl). Following the previous works, we consider the mass of  primary black hole in each binary and model the BHMF as a function of the power law with slope parameter as $\alpha$. We take into account the bias that would be raised by the uncertainty of the measurement and the selection effect and find that a thousand GW events measured by ET ($\sim1\%$ amount of the yearly detection rate) could measure the $\alpha$ to $3\%$ accuracy level. Furthermore, we investigate a scenario, that  $\alpha$ is evolving with redshift as $\alpha(z) = \alpha_0 + \alpha_1\frac{z}{1+z}$. Our result is that, taking the \dl\ as redshift estimator, with a thousand GW events one is capable of obtaining an unbiased $\alpha_1$ at the uncertainty level of $\pm0.5$. The ability of knowing the evolution scenario of BHMF would help to understand the origin of the BHs and how BH binaries formed at different stage of the Universe.

\end{abstract}

\keywords{keywords TBD}

\section{Introduction} \label{sec_intro}
As first identified as a solution to Einstein's field equations by Schwarzschild in 1916 \citep{Schwarzschild1999}, black holes (BHs) have been predicted to remain a bunches of fundamental questions in unifying GR with quantum physics \citep{Hawking1976, Giddings2017}. The mass of BHs is known to cover a wide range from stellar-mass to supermassive ($\sim10^{10} M_{\odot}$). The recent discovery of gravitational-wave (GW) measurement is a substantial evidence of stellar-mass BHs \citep{Abbott2016}, while supermassive BHs are thought to exist in the centers of almost all the galaxies \citep{Lynden-Bell1969, Kormendy1995}. More recently, the first image of supermassive BH in the center of the giant elliptical galaxy M87 has been reconstructed by radio-wave observations \citep{Alberdi2019}. Nevertheless, in theory, it is still unclear of how the BHs are formed \citep{Fryer1999, Fryer2001, Mirabel2016}. The number and mass distribution of stellar-mass BHs in the Universe is still needed to be more clear.

The recent GW detection activities have brought us to a new era of astronomy \citep[e.g.,][]{Abbott2016, Abbott2016_sum, Abbott2018}. To date, the LIGO and Virgo have achieved two successful runs (O1 and O2) and now is performing the third run (O3). In the first run O1 (from September 12th, 2015 to January 19th, 2016), the first catalog of binary BH (BBH) inspiral systems have been measured including three BBH systems. In the O2 run (from November 30th, 2016 to August 25th, 2017), a total of seven binary black hole mergers have been detected\footnote{Together with the first detection of GW from detection of binary neutron star inspiral system.}. In the next decade, the volume of the events is expected to be increasing rapidly with the improvements in the detector sensitivities.

The GWs provide a direct way to hear the inspiralling BBH systems, enable one to derive their information including mass, spin, luminosity distance\citep{Abbott2017phy}. This brings us the opportunity to not only measure the population of BHs \citep{Abbott2018b}, but also answer some fundamental questions such as the cosmography\citep{Liao2017, Ding2019, Cai2017}, the GW speed\citep{Fan2017, Collett2017}, the strong lensing effects~\citep{Ola2013, Biesiada2014, Ding2015}. 

Focusing on the BH mass function (BHMF), the recent work \citet{Kovetz2017PhRvD} has investigated that further LIGO measurements from thousands of BBHs can be taken to infer the  BHMF as power law and constrain the slope parameter $\alpha$ at 10\% accuracy. More recently, the LIGO collaboration has used ten BBH merger events and constrained the BHMF as power law index to $\alpha~=~1.6\substack{+1.5\\-1.7}$ (90\% credibility) \citep{Abbott2018b}. In this study, we pay extra attention on simulating the GW events that could be measured by third-generation gravitational wave detector, Einstein Telescope (ET) \citep{Abernathy2011} based Monte Carlo (MC) approach. This provides us a mock BBHs catalog with the realistic dataset to exam their ability to constrain the BHMFs, taking into the effects by the data noise level, selection bias in a more realistic way. Moreover, the ET would detect the GW event at distant Universe up to $z\sim17$ \citep{Abernathy2011}. The wide redshift range of the events enable us to exam whether it is possible to detect the $\alpha$ as a function of redshift.

This paper is organized as follows. In Sec~\ref{sec_simulation} we describe the simulation of the BBHs events that detected by ET using the Monte Carlo approach. In this section, we assume the initial assumptions for the BH mass function as the truth. In Sec~\ref{sec_theory}, we introduce the theoretical framework to reconstruct the BHMFs, considering the noise realization and the selection effects. Furthermore, we make a further step by considering the power law index $\alpha$ as a function of redshift and explore the way to use luminosity distance as redshift estimator and detect such evolution. We present our results in the Section~\ref{sec_result}. The discussion and conclusions are given in the last Section~\ref{sec_summary}.

Throughout this paper, we assume a standard concordance cosmology with $H_0= 70$ km s$^{-1}$ Mpc$^{-1}$, $\Omega{_m} = 0.30$, and $\Omega{_\Lambda} = 0.70$.

\vspace{1cm}
\section{Data simulation} \label{sec_simulation}
%\subsection{The simulation of GW events and the detection by ET} \label{subsec_BHMF_redshift}
%Rather than numerical calculation, we use Monto Carlo to simulate the GW signal.
%1. We randomly generate the DCOs with redshift number density as the Dominique's startrack.
%2. We consider for each event,  we randomly assuming its properties, including  $\Theta$, with BH mass as power law.
%3. Knowing the DL, $\Theta$, $\cal M$, we are able to simulation their $\rho$ and test whether could be detected by ET.
%4. Add noise level. We assuming the Dl, $\cal M$, $m_1$, $m_2$ as lognormal distribution.
%\begin{itemize}
%%\renewcommand\labelitemi{--}
%\item We randomly generate the DCOs with redshift number density as the Dominique's startrack.
%\item We consider for each event,  we randomly assuming its properties, including  $\Theta$, with BH mass as power law.
%\item Knowing the DL, $\Theta$, $\cal M$, we are able to simulation their $\rho$ and test whether could be detected by ET.
%\item Add noise level. We assuming the Dl, $\cal M$, $m_1$, $m_2$ as lognormal distribution.
%\end{itemize}
We simulate the GW signal of BBHs that would be obtained in the future ET detection in this section. The numerical predictions of the events heard by ET have been discussed in many works, and forecasted that the yearly detection rate of BBHs would be $\sim10^{4-5}$ \citep{Abernathy2011, Ola2013, Biesiada2014}. More recently, \citet{Yang2019} developed the approach the Monte Carlo (MC) simulation to predict the detection rate by explicitly considering each BBHs event, which provides the way to mimic a realistic BBH GW catalog. The backbone is to build up a mock universe which includes a sufficient volume of BBH events to represent the overall sample. We refer the reader for the details in \citet[][Section 2, therein]{Yang2019}  and briefly recall the key points here.

\subsection{Detection Criteria} \label{subsec_criteria}
We randomly generated the key parameters to determine the intensity (i.e., signal-to-noise ratio, SNR) of each GW system. For a specific BBH event at redshift $z_s$, the ET's corresponding reaction is defined as \citep{Abernathy2011}:

\begin{equation} \label{SNR}
\rho = 8 \Theta \frac{r_0}{d_L(z_s)} \left( \frac{(1+z){\cal M}_0}{1.2 M_{\odot}} \right)^{5/6}
\sqrt {\zeta(f_{max})},
\end{equation}
where $r_0$ is the detector's characteristic distance parameter and $\zeta(f_{max})$ is the dimensionless function reflecting the overlap between the GW signal and the ET's effective bandwidth which is usually simplified as unity. ${\cal M}_0$ is the intrinsic chirp mass which is calculated as $ {\cal M}_0 = \frac{(m_1m_2)^{3/5}}{(m_1+m_2)^{1/5}}$, where the \mone\ and \mtwo\ are the each BH mass in the binary. $\Theta$ is the orientation factor determined by four angles as \citep{Finn93}:
 \begin{equation} \label{Theta}
 \Theta = 2 [ F_{+}^2(1 + \cos^2{\iota} )^2 + 4 F_{\times}^2 \cos^2{\iota} ]^{1/2},
 \end{equation}
where: $F_{+} = \frac{1}{2} (1 + \cos^2{\theta}) \cos{2\phi} \cos{2 \psi} - \cos{\theta} \sin{2 \phi} \sin{ 2 \psi}$, and
$F_{\times} = \frac{1}{2} (1 + \cos^2{\theta}) \sin{2\phi} \cos{2 \psi} + \cos{\theta} \sin{2 \phi} \cos{ 2 \psi}$ are so-called antenna patterns. The four angles ($\theta, \phi, \psi, \iota$) are independent, with $(\cos\theta, \phi/\pi, \psi/\pi, \cos\iota)$ distributed uniformly over the range $[-1, 1]$. Note that the GW signals are considered as detectable if their \snr\ are over the detecting threshold, i.e., $\rho > \rho_0 = 8$.

\subsection{Monte Carlo Approach} \label{MC}
We build up a sufficient volume of BBHs systems in the mock universe, and randomly generate the key parameters for each BBH systems. To build up this volume, we follow previous steps by adopting the intrinsic BBH merger rates $\dot{n}_{0}(z_{s})$ predicted by the population synthesis model (using {\tt StarTrack} code\footnote{The data is taken from the website \url{http://www.syntheticuniverse.org}.}) in \citet{Dominik13}. Therefore, the yearly  merging rate of BBH sources in a redshift interval  $[z_{s}, z_{s}+dz_{s}]$ is 
 \begin{equation}
 d\dot{N} (z_s)=4\pi\left(\frac{c}{H_{0}}\right)^3\frac{\dot{n}_{0}(z_{s})}{1+z_{s}}\frac{\tilde{r}^2(z_{s})}{E(z_{s})}dz_{s}. 
 \end{equation}
Hence, the overall amount of BBH events in the universe can be calculated by ${\dot N} = \int_0^{ \infty} \frac{d {\dot N (z)}}{dz} dz$.

For each BBH systems, we following the possibility distribution of the key parameters to randomly generate their values. These key parameters include the four angles (i.e., $\theta, \phi, \psi, \iota$ in Equation~\ref{Theta}) and the masses of each BHs in the binary systems (i.e., \mone\ and \mtwo).

To randomly generate the BH masses, we follow the previous works \citep{Kovetz2017PhRvD, Abbott2018b, Fishbach2018} and assume the \mone\ follows a power law distribution with hard cut at both maximum and minimum mass:
 \begin{equation} \label{equ_powlaw}
p(m_1|\alpha, M_{max}, M_{min}) = m_1^{\alpha} \mathcal{H}(m_1-M_{min}) \mathcal{H}(M_{max}-m_1),
 \end{equation}
where $\mathcal{H}$ is the Heaviside step function. Then, the secondary mass, \mtwo, is fixed as uniform between $[M_{min}, m_1]$. Note that, we only take the \mone\ to reconstruct the BHMF, thus the assumption of the distribution for \mtwo\ actually does not affect the inference for the shape of BHMF.
The generation of these parameters would determine the value of $\Theta$ and ${\cal M}_0$ in Equation~\ref{SNR}. We combine them with their redshift to realize the $\rho$ of each BBH system. We collect the events which have $\rho > \rho_0 = 8$, meaning that those events with $\rho < 8$ are to faint to be detected. 

We consider the BHMF contains two scenarios. In the first scenario, the $\alpha$ is a constant value and the shape of the BHMF is fix throughout the redshfit. In the second scenario, we consider that $\alpha$ varies as a function of redshift in formalism as:
 \begin{equation} \label{equ_alphaz}
\alpha(z) = \alpha_0 + \alpha_1\frac{z}{1+z} , 
 \end{equation}
so that the $\alpha(z)$ would transform gradually from $\alpha_0$ to $\alpha_0+\alpha_1$ through low$-z$ to high$-z$.

\subsection{Measurement Error} \label{sec_noiselevel}
We aim to produce the mock dataset of the future GW events as measured by ET. To consider the measurement error in a realistic way, we inject the random statistical errors into the mock data in this section.

The measurement of the BBH inspiral parameters including the \dl, chirp mass, m1, m2, and SNR. In practice, the \mone\ and \mtwo\ are derived from the combination of the chirp mass and the total mass (i.e., \mone+\mtwo, which is more sensitive to the ringdown waveform). We adopt the implications by \citet{Ghosh2016}, which explore the statistical errors of the measurement with Bayesian parameter estimation, as our recipe to set up the noise level of our simulated data.  In addition, we note that the measurements of the GW are ${\it skewed}$. Other than the traditional Gaussian distribution, we assume the properties of the measurements follow the Log-Normal distribution (i.e. their $log$ value follows the Gaussian) with the standard deviation as 0.2, 0.35 and 0.17 for \mone\ \dl, and ${\cal M}_0$, respectively. For instance, if the $m_{1,fid}$ is the true value for \mone, its possibility distribution of measured value follows:
 \begin{equation} \label{equ_lognorm}
p(m) = \frac{1}{m\sigma\sqrt{2\pi}} exp \big(- \frac{log(m)-log(m_{1,fid})}{2\sigma} \big) 
 \end{equation}

Having clarified the MC approach and defined data the uncertainty level, we are capable of producing the mock GW dataset. As a demonstration, we list a thousand-like simulated data in Table~\ref{tab_GW_mock_data}, which we would take to infer the BHMFs in per realization.

\begin{deluxetable}{lcccc}
\tablecolumns{5}
\tabletypesize{\footnotesize}
\tablewidth{0pt}
\tablecaption{Demonstration of the mock GW catalog} 
\tablehead{ 
\colhead{Object ID} &
\colhead{\mone}&
\colhead{Luminosity Distance} & 
\colhead{Chirp Mass} &
\colhead{SNR}
\\ 
\colhead{} &
\colhead{($M_\odot$)}&
\colhead{(Mpc)} & 
\colhead{($M_\odot$)} &
\colhead{(\snr)}
\\
\colhead{(1)} &
\colhead{(2)} &
\colhead{(3)} &
\colhead{(4)} &
\colhead{(5)}
} 
\startdata
%\multicolumn{5}{c}{Sample presented in \citet{Treu+07}}\\
%\\
ID1 & $95.85\substack{+21.22\\-17.37}$  & $87120.7\substack{+19288.8\\-15792.3}$  & $61.87\substack{+13.70\\-11.21}$ & 28.707 \\
ID2 & $13.31\substack{+2.95\\-2.41}$  & $81476.5\substack{+18039.1\\-14769.2}$  & $11.68\substack{+2.59\\-2.12}$ & 10.468 \\
ID3 & $7.40\substack{+1.64\\-1.34}$  & $7456.8\substack{+1651.0\\-1351.7}$  & $6.82\substack{+1.51\\-1.24}$ & 39.673 \\
ID4 & $19.02\substack{+4.21\\-3.45}$  & $96201.9\substack{+21299.4\\-17438.4}$  & $19.11\substack{+4.23\\-3.46}$ & 12.227 \\
ID5 & $15.12\substack{+3.35\\-2.74}$  & $47645.3\substack{+10548.8\\-8636.6}$  & $14.24\substack{+3.15\\-2.58}$ & 12.672 \\
ID6 & $18.95\substack{+4.20\\-3.43}$  & $23937.4\substack{+5299.8\\-4339.1}$  & $13.86\substack{+3.07\\-2.51}$ & 16.027 \\
ID7 & $8.65\substack{+1.92\\-1.57}$  & $44053.0\substack{+9753.4\\-7985.4}$  & $6.99\substack{+1.55\\-1.27}$ & 8.383 \\
ID8 & $40.03\substack{+8.86\\-7.26}$  & $60432.5\substack{+13379.9\\-10954.6}$  & $25.53\substack{+5.65\\-4.63}$ & 17.917 \\
ID9 & $32.58\substack{+7.21\\-5.91}$  & $4293.5\substack{+950.6\\-778.3}$  & $18.58\substack{+4.11\\-3.37}$ & 34.190 \\
ID10 & $9.88\substack{+2.19\\-1.79}$  & $50294.8\substack{+11135.4\\-9116.9}$  & $4.65\substack{+1.03\\-0.84}$ & 9.608 \\
... & $...$ & ... & ... & ...\\
ID991 & $7.90\substack{+1.75\\-1.43}$  & $10175.7\substack{+2252.9\\-1844.5}$  & $7.31\substack{+1.62\\-1.32}$ & 9.354 \\
ID992 & $15.48\substack{+3.43\\-2.81}$  & $8058.1\substack{+1784.1\\-1460.7}$  & $17.14\substack{+3.80\\-3.11}$ & 11.149 \\
ID993 & $6.11\substack{+1.35\\-1.11}$  & $9566.9\substack{+2118.1\\-1734.2}$  & $4.11\substack{+0.91\\-0.75}$ & 10.703 \\
ID994 & $17.41\substack{+3.85\\-3.16}$  & $232095.1\substack{+51386.5\\-42071.7}$  & $14.46\substack{+3.20\\-2.62}$ & 8.444 \\
ID995 & $23.76\substack{+5.26\\-4.31}$  & $127615.4\substack{+28254.4\\-23132.7}$  & $16.97\substack{+3.76\\-3.08}$ & 14.972 \\
ID996 & $10.43\substack{+2.31\\-1.89}$  & $65546.6\substack{+14512.2\\-11881.6}$  & $6.96\substack{+1.54\\-1.26}$ & 9.497 \\
ID997 & $5.79\substack{+1.28\\-1.05}$  & $24315.0\substack{+5383.4\\-4407.6}$  & $6.93\substack{+1.53\\-1.26}$ & 14.177 \\
ID998 & $12.48\substack{+2.76\\-2.26}$  & $98731.2\substack{+21859.3\\-17896.9}$  & $6.85\substack{+1.52\\-1.24}$ & 13.340 \\
ID999 & $6.95\substack{+1.54\\-1.26}$  & $75187.9\substack{+16646.8\\-13629.2}$  & $3.40\substack{+0.75\\-0.62}$ & 9.437 \\
ID1000 & $6.53\substack{+1.45\\-1.18}$  & $27801.3\substack{+6155.3\\-5039.5}$  & $9.79\substack{+2.17\\-1.77}$ & 8.390 \\
%... & $...$ & ... & ... \\
\enddata
\label{tab_GW_mock_data}
\tablecomments{
A demonstration of a thousand-like GW catalog with properties are adopted to estimate the BHMF.
}
\end{deluxetable}

\vspace{1cm}
\section{Theoretical Framework}  \label{sec_theory}
In this section, we describe the fitting for the parameterized BHMFs. 
In principle, the modeling for a dataset which follows power-law distribution as Equation~\ref{equ_powlaw} is very straightforward. To derive the posterior of the parameters, one only needs to combine all the measured points together and joint the likelihood by:
 \begin{equation} \label{equ_lik_powlaw}
 P(\alpha, M_{max}, M_{min}|m_{1}) \propto  \prod_{i=1}^{total} P(m_{1,i}|\alpha, M_{max}, M_{min}).
 \end{equation}
However, the obtained \mone\ in Table~\ref{tab_GW_mock_data} actually deviate from the initial power-law distribution. This deviation stems from several steps as exist in the realistic. In Section~\ref{sec_likelihood_noise} and \ref{sec_likelihood_sf}, we introduce them and explore the ways to account for.

\subsection{Measurement Uncertainty}\label{sec_likelihood_noise}
The intrinsic \mone\ follows a power-law distribution, however the measured  \mone\ are disturbed by Log-Normal distribution and thus they do not follow a power-law function anymore \citep{Koen2009}. In theory, if event $X$ follows a power-law distribution and is observed subject to the Gaussian error, then $X + e$ is distributed as the convolution of the power-law and Gaussian distributions. Likewise, given that the nosied data follows the Log-Normal distribution, the intrinsic power-law should be convolved in the likelihood as:
 \begin{equation} \label{equ_lik_conv}
 P(\alpha, M_{max}, M_{min}) \propto  \prod_{i=1}^{total} \hat{P}(m_{1,i}|\alpha, M_{max}, M_{min}),
 \end{equation}
where the $\hat{P}$ is the convolved power-law function convolved with the Log-Normal distribution using the standard deviation as 0.2 as we assumed. We illustrate the effect of such convolving in this Figure~\ref{fig:result_slope}.

\begin{figure}%[!b]
\includegraphics[width=1.05\linewidth]{convolving.pdf}
\caption{
Figure illustrate the convolving of a power-law distribution by a Log-Normal distribution with $\sigma = 0.2$. One can see that the convolution make slope is shallower and the smooth the breaking edge at $m_1 = 5 M_{\odot}$.
}
\label{fig:result_slope}
\end{figure}

\subsection{Selection Effect}\label{sec_likelihood_sf}
The observation has tendency to discover more significant events, known as Malmquist bias. The GW systems with higher values \mone\ trend to stronger signals and thus have a higher possibility to be detected. If not well taking into account, the final BHMFs would be biased the high mass end in particular.

To overcome the selection effects, we introduced the selection factor $\eta$ for the GW event, which is the detecting possibility of one event, if simulating again. The physical meaning of this factor $\eta$ is straightforward -- if one GW event has $\eta=0.2$ indicates this event has 80\% possibility would be missed. In other words, there are four equivalent events that would have been missed. Thus, for this event, one needs to re-calibrate its emergency by manually enhance likelihood by a power of 5. Hence, to account for the global selection effect, 
 \begin{equation} \label{equ_lik_sf}
 P(\alpha, M_{max}, M_{min}) \propto  \prod_{i=1}^{total} \hat{P}(m_{1,i}|\alpha, M_{max}, M_{min})^{1/\eta},
 \end{equation}
$\eta$ is directly determined by the possibility distribution of $\rho$, i.e. $\eta = P(\rho>8)$. In Equation~\ref{SNR}, the distribution function of $\Theta$ has been known by the MC approach, the \cmass\ and \dl\ have been provided in the mock dataset as demonstrated in Table~\ref{tab_GW_mock_data}. Furthermore, the redshift is the unknown parameter since it is non-measurable in the GW detection. We thus take the \dl\ as redshift estimator.

Note that the \dl\ and \cmass\ has asymmetry distribution with large scatter, result in the inferred $\eta$ distributed around its true value in a skewed-biased way. We find that this skewness could be well described by a Log-Normal distribution, with multiplicative standard deviation\footnote{In Log-Normal distribution, the multiplicative standard deviation is the exponent value of the standard deviation, i.e., $\sigma^* = exp(\sigma)$.} as $\sigma^*=log(\eta)/3$. Thus, we manually correct this skewness and obtain the non-biased $\eta$.

\subsection{Luminosity Distance as Redshift Estimator} 
\label{sec_dl_z}
%The redshift is not detectable by GW, however, can be inferred from the DL. 
In the previous section, we take \dl\ as the redshift estimator to derive the redshift and hence selection factor $\eta$. The way to derive the $z$ is quite straightforward: simply an inverse solution of integral function when knowing the \dl$(z)$ and fixing the cosmological model.

Moreover, inferring the $z$ of the source provide an opportunity to model the slope BHMF as a function of redshift. Therefore, we are able to investigate the second scenario (i.e. Equation~\ref{equ_alphaz}) by:
 \begin{equation} \label{equ_lik_alphaz}
 \begin{split}
 P&(\alpha_0, \alpha_1, M_{max}, M_{min}) \propto \\
  &\prod_{i=1}^{total} \hat{P}(m_{1i}, z_{inf,i} |\alpha_0, \alpha_1, M_{max}, M_{min})^{1/\eta}.
  \end{split}
 \end{equation}
 
 We present our inference for the BHMF using the mock data in the next section. 


\vspace{1cm}
\section{Result}\label{sec_result}
%We present the result in this section.
%One thousand of data points.
%An appendix to introduce the correction for the skewness for the $\eta$.
We fit the mock data to the BHMF model to infer the distribution of the best-fitted parameter. To infer a non-biased distribution, we adopt the realization approach. The detail of the simulation is the following. In each realization, we simulate a thousand GW events and infer the best-fit parameters using minimization. We do not perform the realizations until the inferred best-fit parameters have converged and informative. 

In the first scenario, we consider the slope $\alpha$ as a constant. We generated the mock data by taking $\alpha~=~2.35$, $M_{min}~=~5M_{\odot}$, $M_{max}~=~80M_{\odot}$ as the fiducial value. We calculate the likelihood by Equation~\ref{equ_lik_sf} to infer the best-fit parameters in each realization. In Figure~\ref{fig_result_a}, we present the distribution of the inference. We find that the non-biased results have inferred, with  best-fit parameters distributed as $\alpha = 2.36\substack{+0.09\\-0.08}$, $M_{max} = 80.54\substack{+5.45\\-5.60}$,  $M_{min} = 5.03\substack{+0.17\\-0.15}$. Our inference of $\alpha$ indicates that a thousand GW events could constrain the slope within $3.5\%$ level.

\begin{figure}%[!b]
\includegraphics[width=1.0\linewidth]{fig_results_3para.pdf}
\caption{
One- and two-dimensional distributions for the best-fit parameters of the scenario A. The BHMF is assumed as a power-law with hard cut at the $M_{min}$ and $M_{max}$, with a constant slope ($\alpha$) across all the redshift.
}
\label{fig_result_a}
\end{figure}

In the second scenario, the $\alpha$ is evolution with redshift as $\alpha(z) = \alpha_0 + \alpha_1\frac{z}{1+z}$. We take the fiducial parameters as $\alpha_0~=~2.35$, $\alpha_1~=~0.75$, $M_{min}~=~5M_{\odot}$, $M_{max}~=~80M_{\odot}$ to generate a thousand datasets in each realization. We present the inference in Figure~\ref{fig_result_b}. Since the second scenario have one more parameters, the distribution of the best-fit parameters have larger scatter with inference as $\alpha_0 = 2.32\substack{+0.35\\-0.36}$, $\alpha_1 = 0.73\substack{+0.52\\-0.50}$, $M_{max} = 78.03\substack{+10.03\\-8.40}$ and  $M_{min} = 5.02\substack{+0.14\\-0.14}$. Not surprisingly, the convergency exsits between the $\alpha_0$ and $\alpha_1$.

We highlight that in this test, it is the inferred uncertainty of $\alpha_1$ that matters. Our result show that, with a volume of one thousands GW measurements, the inferred value of $\alpha_1$ would have a precision of $\pm0.5$. That is to say, this volume of measurements is able to distinguish the evolution of BHMF, when $\alpha_1$ is deviated from 0 by a value of 0.5. Limited by the computing power, we haven't tested to use a larger sample of measurements. However, conjecturing that the precision of inference is increasing with the volume of sample as a function of $\sqrt{N}$, one year measurements of ET ($\sim10^5$ in total) would result in the $\alpha_1$ with precision down to $\sim0.05$.

We also note that the distribution of the best-fit parameters ($\alpha_0$, $\alpha_1$) do not follows the Gaussian distribution, but rather a large fraction is converge at the center. As a result, the 1-$\sigma$ confidence interval in 2-D parameter space are narrower than the ones in 1-D space. \textcolor{blue}{(What does this suggest?)} 

\begin{figure}%[!b]
\includegraphics[width=1.0\linewidth]{fig_results_4para.pdf}
\caption{
Same as Figure~\ref{fig_result_a} but for the scenario B, where the $\alpha$ of BHMF is evolving with redshift as $\alpha(z) = \alpha_0 + \alpha_1\frac{z}{1+z}$. 
}
\label{fig_result_b}
\end{figure}

\vspace{1cm}
\section{Conclusion \& Discussion} \label{sec_summary}
The third-generation gravitational wave detector, Einstein Telescope, is very powerful and capable of detecting $\sim10^5$ GW events per year, with redshift up to $z\sim17$. In this study, we investigated how the detections of the merger of BBHs improve our knowledge of the black hole mass function.

We performed the Monte Carlo simulation to generate the mock measurements of GW signals of BBHs that would be detected by ET. As a starting point, we assume the BHMF for the primary BH mass follows power law distribution. Based on the BBH intrinsic merger rate predicted by {\tt StarTrack}, we randomly simulate the key parameters of the BBH systems, including the chirp mass, redshift and orientation factor and calculate the corresponding their signal-to-noise ratio \snr\ to ET. We collected the events whose \snr\ exceeds the detecting threshold and injected Log-Normal noise to the detected parameter, including BH mass, chirp mass, luminosity distance to simulate the mock data.

We built up a theoretical framework and explore to use the mock measurements to infer the BHMF. We take into account the measurement error and the selection effect which would lead bias in the inference. We use the realization approach, one thousand GW events per realization, and estimate the distribution of the best-fit parameters in the BHMF, including the power law slope $\alpha$, the maximum BH mass $M_{max}$ and the minimum BH mass $M_{min}$. Furthermore, we use the luminosity distance as redshift estimator and test whether one can infer the $\alpha$ as a function of redshift. We summarize our main results as follows:
\begin{enumerate}
\item Assuming the $\alpha$ is invariable with redshift, one could infer the $\alpha$, $M_{max}$ and $M_{min}$ with precisely and accurately. With one thousand measurements of BBHs events, the $\alpha$ would reach 3\% level, i.e., Figure~\ref{fig_result_a}.
\item Testing the scenario that $\alpha$ is evolving with redshift as $\alpha(z) = \alpha_0 + \alpha_1\frac{z}{1+z}$, our results show that a volume of one thousand measurements of BBHs events would derive the evolving parameter $\alpha_1$ with an uncertainty level of $\pm0.5$. Meaning that with one this volume of sample is able to recognize the evolution of BHMF when $\alpha_1$ is deviate from $0$ by 0.5.
\item According to the fact that the precision of the inference increases with the sample size, as a function of $\sqrt{N}$, we conclude that one year BBH sample by ET would be able to deliver the value of $\alpha_1$ with precision down to 0.05.
\end{enumerate}

We point out a few limitations of this work. First, we have adopted a template of intrinsic BBH merger rate based on the predictions by a standard model in {\tt StarTrack}, which must be different from the realistic one. Of course, the intrinsic BBH merger rate is unknown yet, which is related to the different elements such as BBH masses, explosion mechanism, the metallicity history and the time delay distribution. Also, to simplify, we simulate the value of secondary BH mass \mtwo\ by assuming the two masses of BBHs have independent distribution, which probably not right. However, we address that these limitations affect the prediction of the yearly detection rate of the GW events and their distribution as the function of redshift, but possibly trivial for our case. The fact is that using a sample with the same volume but different redshift distribution, the inference on the BHMF parameters could be very similar. 

In this work, we focus on the inferring the BHMF using the mass properties by the BBH, it is worth to note its application to other approaches. For example, one could infer the spin of BH \citep{Abbott2018b}, the mass function for the binary of NS-NS, NS-BH system, though these events are detectable at lower redshift ($z<4$). In addition, using the luminosity as redshift estimator, one should be able to reconstruct the BBH intrinsic merger rate \citep{Fishbach2018}, the cosmological parameter.


\acknowledgments
We thank Hosek Jr., M.W for the useful discussion.

This work was supported by the National Natural Science Foundation of China under grant Nos. 11633001 and 11373014, the Strategic Priority Research Program of the Chinese Academy of Sciences, grant No. XDB23000000, and the Interdiscipline Research Funds of Beijing Normal University.

X. Ding acknowledges support by China Postdoctoral Science Foundation Funded Project (No. 2017M622501).
%L. Yang is supported from the China Scholarship Council. 
M.B. was supported by the Key Foreign Expert Program for the Central Universities No. X2018002.
%K. Liao was supported by the National Natural Science Foundation of China (NSFC) No. 11603015.
\software{
         {\sc corner}, \citep{corner},
        Matplotlib \citep{Matplotlib},
        and standard Python libraries.
        }

\bibliography{reference}
%\input{reference.bbl}

%\newpage
%\appendix
%\section{section}\label{appendix_section}
%TEXTTEXTTEXTTEXTTEXTTEXTTEXTTEXTTEXT
%\newpage

%\newpage
%\appendix
%\section{Correction for the skewness of the selection factor}\label{appA}
%1. Describe the bias of the selection.
%2. The correction by $\eta/3.$


\end{document}