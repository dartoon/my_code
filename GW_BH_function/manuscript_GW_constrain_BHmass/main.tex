\documentclass[twocolumn]{aastex62}

\newcommand{\vdag}{(v)^\dagger}
\newcommand\aastex{AAS\TeX}
\newcommand\latex{La\TeX}
\newcommand\s{$\sim$}
\newcommand{\e}{\'{e}}
\newcommand{\amen}[1]{\textbf{\textit{#1}}}

\usepackage{amsmath}
\usepackage{xcolor}
\usepackage{mathrsfs}

\usepackage{float}
\usepackage{multirow}
\usepackage{enumitem} 

\shorttitle{Black hole mass function and redshift evolution by ET}
\shortauthors{Marian et al.}

\begin{document}

\newcommand{\mbh}{$\mathcal M_{\rm BH}$}
\newcommand{\cmass}{${\cal M}_0$}
\newcommand{\dl}{$d_L$}
\newcommand{\mone}{$m_1$}
\newcommand{\mtwo}{$m_2$}
\newcommand{\snr}{$\rho$}

\title{Black hole mass function and redshift evolution by ET}

\correspondingauthor{Xuheng Ding}
\email{dingxh@whu.edu.cn}

\author%[0000-0001-8917-2148]
{Xuheng Ding}
\affiliation{School of Physics and Technology, Wuhan University, Wuhan 430072, China}
\affiliation{Department of Physics and Astronomy, University of California, Los Angeles, CA, 90095-
1547, USA}

\author{Lilan Yang}
\affiliation{School of Physics and Technology, Wuhan University, Wuhan 430072, China}
\affiliation{Department of Physics and Astronomy, University of California, Los Angeles, CA, 90095-
1547, USA}

\author%[0000-0003-1308-7304]
{Marek Biesiada}
\affiliation{Department of Astronomy, Beijing Normal University, Beijing 100875, China}
\affiliation{Department of Astrophysics and Cosmology, Institute of Physics, University of Silesia, 75 Pu{\l}ku Piechoty 1, 41-500 Chorz{\'o}w,
Poland}

\author%[0000-0002-4359-5994]
{Kai Liao}
\affiliation{School of Science, Wuhan University of Technology, Wuhan 430070, China}

\author%[0000-0002-3567-6743]
{Zong-Hong Zhu}
\affiliation{School of Physics and Technology, Wuhan University, Wuhan 430072, China}
\affiliation{Department of Astronomy, Beijing Normal University, Beijing 100875, China}


\begin{abstract}
%discover of GW
%Measure of BH mass and thus fit the BH mass.
%The discovery of double compact objects by GW provide a the measurement of their BH masses. Recent works by LIGO group have constrained 10 GW event and constrain the BHs. The ET would detect more GW events with DL and BH mass up to redshift $z\sim10$, and thus enable one to constrain the mass function and study how it evolution with redshift. In this work, test the ability of 1000 GWs events by ET. We tested, 1. how well are they able to recover the BH mass function as power law distribution. 2. If the GW events are able to tell a redshifted-related BH mass function.
We demonstrate the ability of future gravitational-wave (GW) measured by third generation detector Einstein telescope (ET) to infer the slope of the BH mass function (BHMF) and its evolution with redshift. %The measurement of BH binary systems enable the inference of the BH mass (\mone\ and \mtwo\ of each component) together with the luminosity distance (\dl).
We performed Monte Carlo approach to simulate the measurements of BH binaries GW signals as detected by ET, including the BH masses and their luminosity distance (\dl). Following the recent works by Kovetz et al. (2017) and LIGO-Virgo (2018), we consider the more massive black hole in each binary and model the BHMF as a function of power law with slope parameter as $\alpha$. We find that only a thousand GW events ($\sim1\%$ amount of the yearly detection rate) could measure the $\alpha$ to $3\%$ accuracy level. Furthermore, we investigate a scenario that  $\alpha$ is evolving with redshift. We find that using the \dl\ as redshift estimator, one is able to distinguish the evolution trend between positive and negative. This may help to understand the origin of the BHs and how BH binaries formed at different stage of the Universe.

\end{abstract}

\keywords{keywords TBD}

\section{Introduction} \label{sec_intro}
As first identified as a solution to Einstein's field equations by Schwarzschild in 1916, black holes (BHs) have been predicted to remain a bunches of fundamental questions in unifying GR with quantum physics (Hawking 1976; Giddings 2017). The mass of BHs is known to cover a wide range from stellar-mass to supermassive ($\sim10^{10} M_{\odot}$). The recent discovery of gravitational-wave (GW) measurement is a substantial evidence of stellar-mass BHs (Abbott et al. 2016), while supermassive BHs are thought to be existed in the centers of most all the galaxies. More recently, the first image of supermassive BH in the center of the giant elliptical galaxy M87 has been reconstructed by radio-wave observations. Nevertheless, in theory, it is still unclear of how the BHs are formed. The prediction for the number and mass distribution of stellar-mass BHs is still in the Universe is needed to be more clear.

The recent detections by LIGO bring us to a new era of GW astronomy. \textcolor{blue}{COPY THE INTRODUCE THE RECENT LIGO RESULT: Overall, advanced LIGO in its 2015 O1 run observed three1 BH coalescence events, adding nine additional measured black hole masses to the current data (the six premerger masses range from roughly 7 to 36 Mstar). Over the next decade, improvements in detector sensitivities are expected to usher in a wave of newly detected events. LIGO itself is scheduled to perform two more runs (O2,3) with increasing sensitivity before commencing a multiyear run at its design sensitivity at the turn of the decade.} 
The GWs provide a direct way to hear the inspiralling binary BH (BBH) systems, enable one to derive their information including mass, spin, luminosity distance.
This bring us the opportunity to measure the population of BHs and help to answer some fundamental questions such as the cosmography, the GW speed. the strong lensing effects. 
%The detection of GWs are able one to measure the BH mass thus understand the BH mass function. Introducing the importance of BH mass function.

More recently, the LIGO collaboration has used ten BBH merger events and constrained the BHMF as power law index to $\alpha~=~1.6\substack{+1.5\\-1.7}$ (90\% credibility). The previous work (Kovetz 2017) has investigated that further LIGO measurements from thousands of BBHs can be taken to infer the power law index $\alpha$ at 10\% accuracy. In this study, we pay extra attentions on simulating the GW events that could be measured by ET based Monte Carlo (MC) approach. This provides us a BBHs catalog with realistic dataset to exam their ability to constrain the BHMFs, putting the data noise level, selection effects in a more realistic way. Moreover, the ET would detect the GW event at distant Universe up to $z\sim17$; we take the luminosity distance as the redshift estimator and exam whether one could detect the $\alpha$ as a function of redshift.

This paper is organized as follows. In Sec~\ref{sec_simulation} we describe the simulation of the BBHs events that detected by ET using MC approach. In this section, we adopt our initial assumptions for the BH mass function as the truth. In Sec~\ref{sec_theory}, we introduce the theoretical framework as used to reconstruct the BHMFs, including the noise realization and the selection effects. Furthermore, we make a further step by considering the power law index $\alpha$ as a function of redshift and explore the way to use luminosity distance as redshift estimator and detect such evolution. We present our results in the Section. The discussion and conclusions are given in the final Section.

Throughout this paper, we assume a standard concordance cosmology with $H_0= 70$ km s$^{-1}$ Mpc$^{-1}$, $\Omega{_m} = 0.30$, and $\Omega{_\Lambda} = 0.70$.

\section{Data simulation} \label{sec_simulation}
%\subsection{The simulation of GW events and the detection by ET} \label{subsec_BHMF_redshift}
%Rather than numerical calculation, we use Monto Carlo to simulate the GW signal.
%1. We randomly generate the DCOs with redshift number density as the Dominique's startrack.
%2. We consider for each event,  we randomly assuming its properties, including  $\Theta$, with BH mass as power law.
%3. Knowing the DL, $\Theta$, $\cal M$, we are able to simulation their $\rho$ and test whether could be detected by ET.
%4. Add noise level. We assuming the Dl, $\cal M$, $m_1$, $m_2$ as lognormal distribution.
%\begin{itemize}
%%\renewcommand\labelitemi{--}
%\item We randomly generate the DCOs with redshift number density as the Dominique's startrack.
%\item We consider for each event,  we randomly assuming its properties, including  $\Theta$, with BH mass as power law.
%\item Knowing the DL, $\Theta$, $\cal M$, we are able to simulation their $\rho$ and test whether could be detected by ET.
%\item Add noise level. We assuming the Dl, $\cal M$, $m_1$, $m_2$ as lognormal distribution.
%\end{itemize}
We simulate the GW signal of BBHs that would be obtained in the future ET detection in this section. The nurmecial predictions of the events heard by ET have been discussed in many works, and forecasted that the yearly detection rate of BBHs would be $\sim10^{4-5}.$ More recently, Yang et al. (2019) developed the MC approach to predict the detection rate by explicitly considering each BBHs event, which provides the way to mimic a realistic BBH GW catalog for us to test. The backbone is to build up a mock universe which includes a sufficient volume of BBH events to represent the overall.  We refer the reader for the details in Yang et al. (section 2, 2019) and briefly recall the key points here.

\subsection{Detection Criteria} \label{subsec_criteria}
The values of the key parameters that determined the GW signal-to-noise ratio (\snr) are randomly generated. For a specific BBG event at redshift $z_s$, the ET's reaction of this event is:

\begin{equation} \label{SNR}
\rho = 8 \Theta \frac{r_0}{d_L(z_s)} \left( \frac{(1+z){\cal M}_0}{1.2 M_{\odot}} \right)^{5/6}
\sqrt {\zeta(f_{max})},
\end{equation}
where $r_0$ is the detector's characteristic distance parameter and $\zeta(f_{max})$ is the dimensionless function reflecting the overlap between the GW signal and the ET's effective bandwidth which is usually simplified as unity. ${\cal M}_0$ is the intrinsic chirp mass which is calculated as $ {\cal M}_0 = \frac{(m_1m_2)^{3/5}}{(m_1+m_2)^{1/5}}$, where the \mone\ and \mtwo\ are the each BH mass in the binary. $\Theta$ is the orientation factor determined by four angles as:
 \begin{equation} \label{Theta}
 \Theta = 2 [ F_{+}^2(1 + \cos^2{\iota} )^2 + 4 F_{\times}^2 \cos^2{\iota} ]^{1/2},
 \end{equation}
where: $F_{+} = \frac{1}{2} (1 + \cos^2{\theta}) \cos{2\phi} \cos{2 \psi} - \cos{\theta} \sin{2 \phi} \sin{ 2 \psi}$, and
$F_{\times} = \frac{1}{2} (1 + \cos^2{\theta}) \sin{2\phi} \cos{2 \psi} + \cos{\theta} \sin{2 \phi} \cos{ 2 \psi}$ are so-called antenna patterns. The four angles ($\theta, \phi, \psi, \iota$) are independent, with $(\cos\theta, \phi/\pi, \psi/\pi, \cos\iota)$ distributed uniformly over the range $[-1, 1]$. 

The GW signals are considered as detectable if their \snr\ are over the detecting threshold, i.e., $\rho > \rho_0 = 8$.

\subsection{Monte Carlo Approach} \label{MC}
We build up a sufficient volume of BBHs systems in the mock universe, and randomly generate the key parameters for each BBH systems. To build up this volume, we follow previous steps by adopting the intrinsic BBH merger rates $\dot{n}_{0}(z_{s})$ predicted by the population synthesis model (using {\tt StarTrack} code\footnote{The data is taken from the website \url{http://www.syntheticuniverse.org}.}) in Dominik et al. (2013). Therefore, the yearly  merging rate of BBH sources in a redshift interval  $[z_{s}, z_{s}+dz_{s}]$ is 
 \begin{equation}
 d\dot{N} (z_s)=4\pi\left(\frac{c}{H_{0}}\right)^3\frac{\dot{n}_{0}(z_{s})}{1+z_{s}}\frac{\tilde{r}^2(z_{s})}{E(z_{s})}dz_{s}. 
 \end{equation}
Hence, the overall BBHs in the universe can be calculated as ${\dot N} = \int_0^{ \infty} \frac{d {\dot N (z)}}{dz} dz$.

For each BBH systems, we following the possibility distribution of the key parameters to randomly generate their values. The key parameters includes the four angles (i.e., $\theta, \phi, \psi, \iota$, see the Equation~\ref{Theta} and its description) and the each BH mass in the binary systems (i.e., \mone\ and \mtwo).

To randomly simulate the BH masses, we follow the previous works by LIGO 2019 and assuming the BHMF of the more massive BH mass, \mone, following a power law distribution as:
 \begin{equation} \label{equ_powlaw}
p(m_1|\alpha, M_{max}, M_{min}) = m_1^{\alpha} \mathcal{H}(m_1-M_{min}) \mathcal{H}(M_{max}-m_1),
 \end{equation}
where $\mathcal{H}$ is the Heaviside step function. Then, the secondary mass, \mtwo, is fix as uniform between $[M_{min}, m_1]$. Note that, in the next section, we only take the \mone\ to reconstruct the BHMF, thus the assumption of the distribution for \mtwo\ would not affect our inference for the shape of BHMF.

The generation of these parameters would determine the value of $\Theta$ and ${\cal M}_0$. We combine them with their redshift to realize the $\rho$ of each BBH system.
We aim to collect the events which have $\rho > \rho_0 = 8$ so as to estimate the yearly detection rate. 

In this paper, we consider the BHMF as two scenarios. In the first scenario, the $\alpha$ is a constant value and we adopt the values $\alpha~=~2.35$, $M_{min}~=~5M_{\odot}$, $M_{max}~=~80M_{\odot}$ as our fiducial model. In the second scenario, we consider that $\alpha$ varies as a function of redshift in formalism as:
 \begin{equation} \label{equ_alphaz}
\alpha(z) = \alpha_0 + \alpha_1\frac{z}{1+z} , 
 \end{equation}

so that the $\alpha(z)$ would transform gradually from $\alpha_0$ to $\alpha_0+\alpha_1$ through low$-z$ to high$-z$. We adopt the values $\alpha_0~=~2.35$, $\alpha_1~=~0.75$, $M_{min}~=~5M_{\odot}$, $M_{max}~=~80M_{\odot}$ as the truth.

\subsection{Measurement Error} \label{sec_noiselevel}
Our simulation provides a mock dataset of the future GW events as measured by ET. To consider the measurements in a realistic way, we take the random statistical errors into account here.

We assume the observed probability distribution with their uncertainty level of the data, including the \mone, \mtwo, and \dl. In theory, the \mone\ and \mtwo\ are constrained from the combining of the chirp mass (a direct measurement by the inspiral) and the total mass (i.e., \mone+\mtwo, which is more sensitive to the ringdown waveform).

In Archisman et al. (2015), the expected statistical errors of the measurement has been estimated with Bayesian parameter estimation. We adopt their implications as our guidance to setup noise level of our simulated data.  Nevertheless, we note that the measurements of the GW are ${\it skewed}$. Other than the traditional Gaussian distribution, we assume the properties of the measurements follow the Log-Normal distribution (i.e. their $log$ value follows the Gaussian). As a consequence, we assume the standard deviation of \mone\ \dl, and ${\cal M}_0$ as 0.2, 0.35 and 0.17, respectively. For instance, if the $m_{1,fid}$ is the true value for \mone, its possibility distribution of measured value follows:
 \begin{equation} \label{equ_lognorm}
p(m) = \frac{1}{m\sigma\sqrt{2\pi}} exp \big(- \frac{log(m)-log(m_{1,fid})}{2\sigma} \big) 
 \end{equation}

As a demonstration, we list a thousand-like simulated data in Table~\ref{tab_GW_mock_data}, which we would take to infer the BHMFs in one realization.

\begin{deluxetable}{lccc}
\tablecolumns{4}
\tablewidth{0pt}
\tablecaption{Details of Observation} 
\tablehead{ 
\colhead{Object ID} &
\colhead{\mone}&
\colhead{Luminosity Distance} & 
\colhead{Chirp Mass}
\\ 
\colhead{} &
\colhead{($M_\odot$)}&
\colhead{Mpc} & 
\colhead{($M_\odot$)}
\\
\colhead{(1)} &
\colhead{(2)} &
\colhead{(3)} &
\colhead{(4)}
} 
\startdata
%\multicolumn{5}{c}{Sample presented in \citet{Treu+07}}\\
%\\
ID1 & $11.05\substack{+0.15\\-0.11}$ & $11.05\substack{+0.15\\-0.11}$ & $11.05\substack{+0.15\\-0.11}$ \\
ID2 & $11.05\substack{+0.15\\-0.11}$ & $11.05\substack{+0.15\\-0.11}$ & $11.05\substack{+0.15\\-0.11}$ \\
ID3 & $11.05\substack{+0.15\\-0.11}$ & $11.05\substack{+0.15\\-0.11}$ & $11.05\substack{+0.15\\-0.11}$ \\
ID4 & $11.05\substack{+0.15\\-0.11}$ & $11.05\substack{+0.15\\-0.11}$ & $11.05\substack{+0.15\\-0.11}$ \\
ID5 & $11.05\substack{+0.15\\-0.11}$ & $11.05\substack{+0.15\\-0.11}$ & $11.05\substack{+0.15\\-0.11}$ \\
ID6 & $11.05\substack{+0.15\\-0.11}$ & $11.05\substack{+0.15\\-0.11}$ & $11.05\substack{+0.15\\-0.11}$ \\
ID7 & $11.05\substack{+0.15\\-0.11}$ & $11.05\substack{+0.15\\-0.11}$ & $11.05\substack{+0.15\\-0.11}$ \\
ID8 & $11.05\substack{+0.15\\-0.11}$ & $11.05\substack{+0.15\\-0.11}$ & $11.05\substack{+0.15\\-0.11}$ \\
ID9 & $11.05\substack{+0.15\\-0.11}$ & $11.05\substack{+0.15\\-0.11}$ & $11.05\substack{+0.15\\-0.11}$ \\
ID10 & $11.05\substack{+0.15\\-0.11}$ & $11.05\substack{+0.15\\-0.11}$ & $11.05\substack{+0.15\\-0.11}$ \\
... & $...$ & ... & ... \\
ID901 & $11.05\substack{+0.15\\-0.11}$ & $11.05\substack{+0.15\\-0.11}$ & $11.05\substack{+0.15\\-0.11}$ \\
ID902 & $11.05\substack{+0.15\\-0.11}$ & $11.05\substack{+0.15\\-0.11}$ & $11.05\substack{+0.15\\-0.11}$ \\
ID903 & $11.05\substack{+0.15\\-0.11}$ & $11.05\substack{+0.15\\-0.11}$ & $11.05\substack{+0.15\\-0.11}$ \\
ID904 & $11.05\substack{+0.15\\-0.11}$ & $11.05\substack{+0.15\\-0.11}$ & $11.05\substack{+0.15\\-0.11}$ \\
ID905 & $11.05\substack{+0.15\\-0.11}$ & $11.05\substack{+0.15\\-0.11}$ & $11.05\substack{+0.15\\-0.11}$ \\
ID906 & $11.05\substack{+0.15\\-0.11}$ & $11.05\substack{+0.15\\-0.11}$ & $11.05\substack{+0.15\\-0.11}$ \\
ID907 & $11.05\substack{+0.15\\-0.11}$ & $11.05\substack{+0.15\\-0.11}$ & $11.05\substack{+0.15\\-0.11}$ \\
ID908 & $11.05\substack{+0.15\\-0.11}$ & $11.05\substack{+0.15\\-0.11}$ & $11.05\substack{+0.15\\-0.11}$ \\
ID909 & $11.05\substack{+0.15\\-0.11}$ & $11.05\substack{+0.15\\-0.11}$ & $11.05\substack{+0.15\\-0.11}$ \\
ID1000 & $11.05\substack{+0.15\\-0.11}$ & $11.05\substack{+0.15\\-0.11}$ & $11.05\substack{+0.15\\-0.11}$ \\
\enddata
\label{tab_GW_mock_data}
\tablecomments{
A demonstration of a thousand-like GW catalog.
}
\end{deluxetable}

\section{Theoretical Framework}  \label{sec_theory}
In this section, we describe the fitting for the parameterized BHMFs. 
In principle, the modeling for a dataset which follows power-law distribution as Equation~\ref{equ_powlaw} is very straightforward. One just needs to combining all the measured points together by:
 \begin{equation} \label{equ_likeli_simple}
 P(\alpha, M_{max}, M_{min}) =  \prod_{i=1}^{total} P(m_{1,i}|\alpha, M_{max}, M_{min}).
 \end{equation}
However, the observed \mone\ are deviated from a power-law by several processes. In the rest of this section, we introduce them and explore the ways to reconstruct the non-biased likelihood.

\subsection{Measurement Errors}\label{sec_likelihood_noise}
We assumed the intrinsic \mone\ follows a power-law distribution, however the measured  \mone\ are disturbed by Log-Normal distribution and thus they do not follow a power-law function any more (C. Koen 2009). 

In theory, if event $X$ follows a power-law distribution and is observed subject to the Gaussian error, then $X + e$ is distributed as the convolution of the power-law and Gaussian distributions. Likewise, given that our data are observed as Log-Normal distribution, the intrinsic power-law should be convolved in the likelihood as:
 \begin{equation} \label{equ_likelihood_conv}
 P(\alpha, M_{max}, M_{min}) =  \prod_{i=1}^{total} \hat{P}(m_{1,i}|\alpha, M_{max}, M_{min}),
 \end{equation}
where the $\hat{P}$ is the convolved power-law function convolved with the Log-Normal distribution using the standard deviation as 0.2 as we assumed. We illustrate the effect of such convolving in this figure.

\begin{figure}%[!b]
\includegraphics[width=1.05\linewidth]{convolving.pdf}
\caption{
Figure illustrate the convolving of a power-law distribution by a Log-Normal distribution with $\sigma = 0.2$. One can see that the convolution make slope is shallower and the smooth the breaking edge at $m_1 = 5 M_{\odot}$.
}
\label{fig:result_slope}
\end{figure}

\subsection{Selection Effect}\label{sec_likelihood_sf}
The observation has tendency to discover more significant events. Those GW systems with higher values \mone\ trend to stronger signals and thus more easier to be detected. If not well handled, the final BHMFs would be biased with higher fraction at the high mass end.

To overcome the selection effects, we calculated the selection factor $\eta$ for the GW event, which is the detecting possibility of one event, if simulating again. The physical meaning of this factor $\eta$ is straightforward -- if one GW event has $\eta=0.2$ indicates this event has 80\% possibility would be missed. In other words, there are four equivalent events that has been missed. Thus, for this event, one needs to re-calibrate its emergency by manually enhance likelihood by a power of 5. Hence, to account for the global selection effect, 
 \begin{equation} \label{equ_powlaw_like}
 P(\alpha, M_{max}, M_{min}) =  \prod_{i=1}^{total} \hat{P}(m_{1,i}|\alpha, M_{max}, M_{min})^{1/\eta}.
 \end{equation}

Note that the $\eta$ is directly estimated by the possibility distribution of $\rho$, i.e. Equation~\ref{SNR}, which is determined by the distribution function of $\Theta$. In Equation~\ref{SNR}, the \cmass\ and \dl\ have been provided in the mock dataset as demonstrated in Table~\ref{tab_GW_mock_data}. However, the redshift is the unknown parameter. Though its non-measurable in the GW detection, we introduce to take the \dl\ as redshift estimator in the following section and test our re-calibration for selection effect in Section~\ref{sec_result}.

\subsection{Luminosity Distance as Redshift Estimator} 
%The redshift is not detectable by GW, however, can be inferred from the DL. 
We take \dl\ as the redshift estimator so as to derive the selection factor $\eta$. The way to derive the $z$ is quite straightforward: simply an inverse solution of integral function when knowing the \dl$(z)$ and fixing the cosmological model.

Moreover, inferring the $z$ provide a way to model the slope BHMF as a function of redshift. Thus, we consider the second scenario (i.e. Equation~\ref{equ_alphaz}) to model the $\alpha$ as a function of redshift:
 \begin{equation} \label{equ_likelihood_alphaz}
 \begin{split}
 P&(\alpha_0, \alpha_1, M_{max}, M_{min}) = \\
  &\prod_{i=1}^{total} \hat{P}(m_{1i}, z_{inf,i} |\alpha_0, \alpha_1, M_{max}, M_{min})^{1/\eta}.
  \end{split}
 \end{equation}


\vspace{1cm}

\section{Result}\label{sec_result}
We present the result in this section.

An appendix to introduce the correction for the skewness for the $\eta$.

\begin{figure*}%[!b]
\begin{tabular}{c c}
\includegraphics[width=0.5\linewidth]{fig_results_3para.pdf} &
\includegraphics[width=0.5\linewidth]{fig_results_4para.pdf} \\
\end{tabular}
\caption{
The main figure... 
}
\label{fig:result_slope}
\end{figure*}

\section{Summary \& Conclusions} \label{sec:summary}

In this study, we simulated the GWs event and test it ability on constrain the BH. We find that:


Our results are consistent with...



\acknowledgments

This work was supported by the National Basic Science Program (Project 973) of China under (Grant No. 2014CB845800), the National Natural Science Foundation of China under Grants Nos. 11633001 and 11373014, the Strategic Priority Research Program of the Chinese Academy of Sciences, Grant No. XDB23000000 and the Interdiscipline Research Funds of Beijing Normal University.

X. Ding acknowledges support by China Postdoctoral Science Foundation Funded Project (No. 2017M622501).
L. Yang is supported from the China Scholarship Council. 
M.B. was supported by the Key Foreign Expert Program for the Central Universities No. X2018002.
K. Liao was supported by the National Natural Science Foundation of China (NSFC) No. 11603015.
\software{
        corner,
        Matplotlib \citep{matplotlib},
        }

\newpage

%\appendix

%\section{section}\label{appendix_section}
%TEXTTEXTTEXTTEXTTEXTTEXTTEXTTEXTTEXT
%\newpage

\bibliography{reference}
%\input{reference.bbl}

\end{document}